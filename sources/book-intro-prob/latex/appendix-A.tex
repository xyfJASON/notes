\section{常见随机变量}
\label{sec:random-variables}

本节总结常见的随机变量及其期望、方差、矩母函数和性质,其中离散随机变量包括伯努利、二项、泊松、几何和超几何随机变量,连续随机变量包括均匀、指数和正态随机变量。


\paragraph{伯努利随机变量}

\begin{itemize}[itemsep=1ex]
    \item 记号:$X\sim B(1,p)$
    \item 实例:抛掷一枚硬币,硬币向上概率为 $p$,$X$ 为是否向上。
    \item PMF:$p_X(k)=\begin{cases}p,&k=1\\1-p,&k=0\end{cases}$
    \item 期望:$\E X=p$
    \item 方差:$\var X=p(1-p)$
    \item 矩母函数:$M_X(s)=1-p+pe^s$
\end{itemize}

\begin{proof}[矩母函数的推导]
\[
M_X(s)=\E[e^{sX}]=(1-p)\cdot e^{0}+p\cdot e^{s}=1-p+pe^s
\]
\end{proof}

\paragraph{二项随机变量}

\begin{itemize}[itemsep=1ex]
    \item 记号:$X\sim B(n,p)$
    \item 实例:抛掷 $n$ 枚硬币,每枚硬币向上概率均为 $p$,$X$ 为向上次数。
    \item PMF:$p_X(k)=\displaystyle\binom{n}{k}p^k(1-p)^{n-k},\quad k=0,1,2,\ldots,n$
    \item 期望:$\E X=np$
    \item 方差:$\var X=np(1-p)$
    \item 矩母函数:$M_X(s)=(1-p+pe^s)^n$
\end{itemize}

\begin{proof}[期望的推导]
\begin{align*}
\E X&=\sum_{k=0}^{n}k\binom{n}{k}p^k(1-p)^{n-k}=\sum_{k=1}^nn\binom{n-1}{k-1}p^k(1-p)^{n-k}\\
&=np\sum_{k=0}^{n-1}\binom{n-1}{k}p^k(1-p)^{n-1-k}=np(p+1-p)^{n-1}=np
\end{align*}
其中用到了恒等式 $\displaystyle\binom{n}{k}=\binom{n-1}{k-1}\frac{n}{k}$ 和二项式定理。
\end{proof}
\begin{proof}[二阶矩的推导]
\begin{align*}
\E X^2&=\sum_{k=0}^{n}k^2\binom{n}{k}p^k(1-p)^{n-k}=\sum_{k=1}^nnk\binom{n-1}{k-1}p^k(1-p)^{n-k}\\
&=\sum_{k=1}^nn\binom{n-1}{k-1}p^k(1-p)^{n-k}+\sum_{k=1}^nn(k-1)\binom{n-1}{k-1}p^k(1-p)^{n-k}\\
&=np+np\sum_{k=0}^{n-1}k\binom{n-1}{k}p^{k}(1-p)^{n-1-k}=np+np(n-1)p
\end{align*}
\end{proof}
\begin{proof}[方差的推导]
\[
\var X=\E X^2-(\E X)^2=np+n(n-1)p^2-n^2p^2=np-np^2=np(1-p)
\]
\end{proof}
\begin{proof}[矩母函数的推导]
\[
M_X(s)=\E[e^{sX}]=\sum_{k=0}^n\binom{n}{k}p^k(1-p)^{n-k}e^{sk}=(1-p+pe^s)^n
\]
\end{proof}

\paragraph{泊松随机变量}

\begin{itemize}[itemsep=1ex]
    \item 记号:$X\sim P(\lambda)$
    \item 实例:一个城市一天中发生车祸次数。
    \item PMF:$p_X(k)=e^{-\lambda}\dfrac{\lambda^k}{k!},\quad k=0,1,\ldots$
    \item 期望:$\E X=\lambda$
    \item 方差:$\var X=\lambda$
    \item 矩母函数:$M_X(s)=e^{\lambda(e^s-1)}$
    \item 性质:取 $\lambda=np$,则当 $n\to\infty$ 时,泊松分布近似二项分布。
\end{itemize}

\begin{proof}[期望的推导]
\[
\E X=\sum_{k=0}^\infty ke^{-\lambda}\frac{\lambda^k}{k!}=e^{-\lambda}\sum_{k=1}^{\infty}\frac{\lambda^k}{(k-1)!}=\lambda e^{-\lambda}\sum_{k=0}^{\infty}\frac{\lambda^k}{k!}=\lambda
\]
其中用到了 $e^x$ 的泰勒展开。
\end{proof}
\begin{proof}[二阶矩的推导]
\begin{align*}
\E X^2&=\sum_{k=0}^\infty k^2e^{-\lambda}\frac{\lambda^k}{k!}=e^{-\lambda}\sum_{k=1}^{\infty}k\frac{\lambda^k}{(k-1)!}\\
&=e^{-\lambda}\sum_{k=1}^{\infty}\frac{\lambda^k}{(k-1)!}+e^{-\lambda}\sum_{k=2}^{\infty}\frac{\lambda^k}{(k-2)!}\\
&=e^{-\lambda}\lambda e^\lambda+e^{-\lambda}\lambda^2 e^\lambda=\lambda+\lambda^2
\end{align*}
\end{proof}
\begin{proof}[方差的推导]
\[
\var X=\E X^2-(\E X)^2=\lambda+\lambda^2-\lambda^2=\lambda
\]
\end{proof}
\begin{proof}[矩母函数的推导]
\[
M_X(s)=\E[e^{sX}]=\sum_{k=0}^\infty e^{-\lambda}\frac{\lambda^k}{k!}e^{sk}=e^{-\lambda}\sum_{k=0}^\infty \frac{(\lambda e^s)^k}{k!}=e^{\lambda(e^s-1)}
\]
\end{proof}
\begin{proof}[证明泊松分布近似二项分布]
取 $\lambda=np$,则:
\begin{align*}
\lim_{n\to\infty}\binom{n}{k}p^k(1-p)^{nk}&=\lim_{n\to\infty}\frac{n^{\underline{k}}}{k!}\cdot\frac{\lambda^k}{n^k}\left(1-\frac{\lambda}{n}\right)^{n-k}\\
&=\frac{\lambda^k}{k!}\lim_{n\to\infty}\frac{n^{\underline{k}}}{n^k}\cdot\left(1-\frac{\lambda}{n}\right)^{n-k}\\
&=\frac{\lambda^k}{k!}\cdot 1\cdot\lim_{n\to\infty}\left[\left(1-\frac{\lambda}{n}\right)^{-\frac{n}{\lambda}}\right]^{-\frac{\lambda(n-k)}{n}}\\
&=\frac{\lambda^k}{k!}e^{-\lambda}
\end{align*}
\end{proof}

\paragraph{几何随机变量}

\begin{itemize}[itemsep=1ex]
    \item 记号:$X\sim G(p)$
    \item 实例:抛掷一枚硬币直至向上,硬币向上概率为 $p$,$X$ 为抛掷次数。
    \item PMF:$p_X(k)=(1-p)^{k-1}p,\quad k=1,2,\ldots$
    \item 期望:$\E X=\dfrac{1}{p}$
    \item 方差:$\var X=\dfrac{1-p}{p^2}$
    \item 矩母函数:$M_X(s)=\dfrac{pe^s}{1-(1-p)e^s}$
    \item 无记忆性:$\Pb(X>n+m\vert X>n)=\Pb(X>m)$
\end{itemize}

\begin{proof}[期望的推导]
设:
\[
f(x)=\sum_{k=1}^\infty k x^{k-1}=\sum_{k=1}^\infty (x^k)'=\left(\sum_{k=1}^\infty x^k\right)'=\left(\frac{x}{1-x}\right)'=\frac{1}{(1-x)^2}
\]
则:
\[
\E X=\sum_{k=1}^\infty k(1-p)^{k-1}p=pf(1-p)=\frac{1}{p}
\]
\end{proof}
\begin{proof}[二阶矩的推导]
设:
\[
g(x)=\sum_{k=1}^\infty k^2x^{k-1}=\sum_{k=1}^\infty k(x^k)'=\left(\sum_{k=1}^\infty kx^k\right)'=(xf(x))'=\left(\frac{x}{(1-x)^2}\right)'=\frac{1+x}{(1-x)^3}
\]
则:
\[
\E X^2=\sum_{k=1}^\infty k^2(1-p)^{k-1}p=pg(1-p)=p\cdot\frac{2-p}{p^3}=\frac{2-p}{p^2}
\]
\end{proof}
\begin{proof}[方差的推导]
\[
\var X=\E X^2-(\E X)^2=\frac{2-p}{p^2}-\frac{1}{p^2}=\frac{1-p}{p^2}
\]
\end{proof}
\begin{proof}[矩母函数的推导]
\[
M_X(s)=\E[e^{sX}]=\sum_{k=1}^\infty(1-p)^{k-1}pe^{sk}=pe^s\sum_{k=0}^\infty((1-p)e^s)^{k}=\frac{pe^s}{1-(1-p)e^s}
\]
\end{proof}
\begin{proof}[证明无记忆性]
首先计算尾概率:
\[
\Pb(X>n)=\sum_{k=n+1}^\infty(1-p)^{k-1}p=p\frac{(1-p)^n}{1-(1-p)}=(1-p)^n
\]
于是:
\[
\Pb(X>n+m\vert X>n)=\frac{P(X>n+m)}{P(X>n)}=\frac{(1-p)^{n+m}}{(1-p)^n}=(1-p)^m=\Pb(X>m)
\]
\end{proof}

\paragraph{超几何随机变量}

\begin{itemize}[itemsep=1ex]
    \item 实例:一盒内有 $N$ 个球,其中 $M$ 个白球 $N-M$ 个黑球,从中无放回地取 $n$ 个球且每次取球独立,$X$ 表示取出白球个数。
    \item PMF:$p_X(k)=\dfrac{\binom{M}{k}\binom{N-M}{n-k}}{\binom{N}{n}},\quad k=0,1,\ldots,n$
    \item 期望:$\E X=\dfrac{nM}{N}$
    \item 方差:$\var X=\dfrac{nM}{N}\left(1-\dfrac{M}{N}\right)\dfrac{N-n}{N-1}$
    \item 性质:当 $N\to\infty$ 时,超几何分布近似二项分布。
\end{itemize}

\begin{proof}[期望的推导]
\[
\E X=\sum_kk\frac{\binom{M}{k}\binom{N-M}{n-k}}{\binom{N}{n}}=\frac{1}{\binom{N}{n}}\sum_k M\binom{M-1}{k-1}\binom{N-M}{n-k}=\frac{M}{\binom{N}{n}}\binom{N-1}{n-1}=\frac{nM}{N}
\]
其中用到了恒等式 $\displaystyle\binom{N}{n}=\binom{N-1}{n-1}\frac{N}{n}$ 和范德蒙德卷积式 $\displaystyle\sum_k\binom{r}{k}\binom{s}{n-k}=\binom{r+s}{n}$.
\end{proof}
\begin{proof}[二阶矩的推导]
\begin{align*}
\E X^2&=\sum_kk^2\frac{\binom{M}{k}\binom{N-M}{n-k}}{\binom{N}{n}}=\frac{M}{\binom{N}{n}}\sum_k k\binom{M-1}{k-1}\binom{N-M}{n-k}\\
&=\frac{M}{\binom{N}{n}}\sum_k\binom{M-1}{k-1}\binom{N-M}{n-k}+\frac{M}{\binom{N}{n}}\sum_k (k-1)\binom{M-1}{k-1}\binom{N-M}{n-k}\\
&=\frac{M}{\binom{N}{n}}\binom{N-1}{n-1}+\frac{M}{\binom{N}{n}}(M-1)\binom{N-2}{n-2}=\frac{nM}{N}+\frac{M(M-1)n(n-1)}{N(N-1)}
\end{align*}
\end{proof}
\begin{proof}[方差的推导]
\[
\var X=\E X^2-(\E X)^2=\frac{nM}{N}+\frac{M(M-1)n(n-1)}{N(N-1)}-\frac{n^2M^2}{N^2}=\frac{nM}{N}\left(1-\frac{M}{N}\right)\frac{N-n}{N-1}
\]
\end{proof}


\paragraph{均匀随机变量}

\begin{itemize}[itemsep=1ex]
    \item 记号:$X\sim U(a,b)$
    \item PDF:$f_X(x)=\dfrac{1}{b-a},\quad a\leqslant x\leqslant b$
    \item 期望:$\E X=\dfrac{a+b}{2}$
    \item 方差:$\var X=\dfrac{(b-a)^2}{12}$
    \item 矩母函数:$M_X(s)=\dfrac{1}{b-a}\cdot\dfrac{e^{sb}-e^{sa}}{s}$
\end{itemize}

\begin{proof}[矩母函数的推导]
\[
M_X(s)=\E[e^{sX}]=\int_a^b\frac{e^{sx}}{b-a}\mathrm dx=\frac{1}{b-a}\cdot\frac{1}{s}\int_{a/s}^{b/s}e^{sx}\mathrm d(sx)=\frac{1}{b-a}\cdot\frac{e^{sb}-e^{sa}}{s}
\]
\end{proof}

\paragraph{指数随机变量}

\begin{itemize}[itemsep=1ex]
    \item 记号:$X\sim E(\lambda)$
    \item PDF:$f_X(x)=\lambda e^{-\lambda x},\quad x\geqslant0$
    \item 期望:$\E X=\dfrac{1}{\lambda}$
    \item 方差:$\var X=\dfrac{1}{\lambda^2}$
    \item 矩母函数:$M_X(s)=\dfrac{\lambda}{\lambda-s}\;(s<\lambda)$
    \item 无记忆性:$\Pb(X>x+y\vert X>x)=\Pb(X>y)$
\end{itemize}

\begin{proof}[期望的推导]
\[
\E X=\int_0^{+\infty}x\lambda e^{-\lambda x}\mathrm dx=-\int_0^{+\infty}x\mathrm d e^{-\lambda x}=\int_0^{+\infty}e^{-\lambda x}\mathrm dx=-\frac{1}{\lambda}\left.e^{-\lambda x}\right|_{0}^{+\infty}=\frac{1}{\lambda}
\]
\end{proof}
\begin{proof}[二阶矩的推导]
\[
\E X^2=\int_0^{+\infty} x^2\lambda e^{-\lambda x}\mathrm dx=-\int_0^{+\infty}x^2\mathrm de^{-\lambda x}=2\int_0^{+\infty}x e^{-\lambda x}\mathrm dx=\frac{2}{\lambda^2}
\]
\end{proof}
\begin{proof}[方差的推导]
\[
\var X=\E X^2-(\E X)^2=\frac{2}{\lambda^2}-\frac{1}{\lambda^2}=\frac{1}{\lambda^2}
\]
\end{proof}
\begin{proof}[矩母函数的推导]
\[
M_X(s)=\E[e^{sX}]=\int_0^\infty\lambda e^{-\lambda x}e^{sx}\mathrm dx=\lambda\int_0^\infty e^{-(\lambda-s)x}\mathrm dx=\frac{\lambda}{\lambda-s}\quad(s<\lambda)
\]
\end{proof}
\begin{proof}[证明无记忆性]
首先计算尾概率:
\[
\Pb(X>x)=\int_x^{+\infty}\lambda e^{-\lambda t}\mathrm dt=\left.e^{-\lambda t}\right|_{+\infty}^x=e^{-\lambda x}
\]
于是:
\[
\Pb(X>x+y\vert X>x)=\frac{\Pb(X>x+y)}{\Pb(X>x)}=\frac{e^{-\lambda(x+y)}}{e^{-\lambda x}}=e^{-\lambda y}=\Pb(X>y)
\]
\end{proof}

\paragraph{正态随机变量}

\begin{itemize}[itemsep=1ex]
    \item 记号:$X\sim N(\mu,\sigma^2)$
    \item PDF:$f_X(x)=\dfrac{1}{\sqrt{2\pi}\sigma}\exp\left({-\dfrac{(x-\mu)^2}{2\sigma^2}}\right)$
    \item 期望:$\E X=\mu$
    \item 方差:$\var X=\sigma^2$
    \item 矩母函数:$M_X(s)=\exp\left({\dfrac{\sigma^2s^2}{2}+\mu s}\right)$
\end{itemize}

\begin{proof}[证明标准正态分布的归一性]
由于:
\begin{align*}
\left[\int_{-\infty}^{+\infty}\frac{1}{\sqrt{2\pi}}\exp\left({-\frac{x^2}{2}}\right)\mathrm dx\right]^2&=\frac{1}{2\pi}\int_{-\infty}^{+\infty}\int_{-\infty}^{+\infty}\exp\left({-\frac{x^2+y^2}{2}}\right)\mathrm dx\mathrm dy\\
&=\frac{1}{2\pi}\int_0^{2\pi}\mathrm d\theta\int_{0}^{+\infty}\exp\left({-\frac{r^2}{2}}\right)r\mathrm dr\\
&=\int_0^{+\infty}\exp\left({-\frac{r^2}{2}}\right)\mathrm d\left(\frac{r^2}{2}\right)\\
&=\left.\exp\left({-\frac{r^2}{2}}\right)\right|_{+\infty}^0=1
\end{align*}
故:
\[
\int_{-\infty}^{+\infty}\frac{1}{\sqrt{2\pi}}\exp\left({-\frac{x^2}{2}}\right)\mathrm dx=1
\]
\end{proof}
% \begin{proof}[矩母函数的推导]
% \begin{align*}
% M_X(s)=\E[e^{sX}]&=\int_{-\infty}^{+\infty}\frac{1}{\sqrt{2\pi}\sigma}e^{-\frac{(x-\mu)^2}{2\sigma^2}}e^{sx}\mathrm dx=\frac{1}{\sqrt{2\pi}\sigma}\int_{-\infty}^{+\infty}e^{-\frac{(x-\mu)^2}{2\sigma^2}+sx}\mathrm dx\\
% &=\frac{1}{\sqrt{2\pi}\sigma}\int_{-\infty}^{+\infty}e^{-\frac{(x-\mu-s\sigma^2)^2}{2\sigma^2}}\mathrm dx
% \end{align*}
% \end{proof}
