\section{正态分布的三个导出分布}
\label{sec:normal-derive}

\begin{definition}[$\chi^2$ 分布]
设 $X_1,X_2,\cdots,X_n$ 为 $n$ 个独立的服从 $N(0,1)$ 的随机变量,则称
\[
Z=\sum\limits_{i=1}^nX_i^2
\]
的分布为自由度为 $n$ 的 $\chi^2$ 分布,记作 $Z\sim \chi^2(n)$. 
\end{definition}

\noindent\textbf{期望与方差}:
\[
\E Z=n,\quad\var Z=2n
\]
\begin{proof}
由于
\[
\E X_i^2=\var X_i+(\E X_i)^2=1+0=1
\]
故
\[
\E Z=\E \left[\sum_{i=1}^nX_i^2\right]=\sum_{i=1}^n\E X_i^2=n
\]
又由于
\begin{align*}
\E X_i^4&=\int_{-\infty}^{+\infty}x^4\varphi(x)\mathrm dx\\
&=\frac{1}{\sqrt{2\pi}}\int_{-\infty}^{+\infty}x^4e^{-\frac{x^2}{2}}\mathrm dx
=-\frac{1}{\sqrt{2\pi}}\int_{-\infty}^{+\infty}x^3\mathrm d e^{-\frac{x^2}{2}}\\
&=\frac{3}{\sqrt{2\pi}}\int_{-\infty}^{+\infty}x^2e^{-\frac{x^2}{2}}\mathrm dx
=-\frac{3}{\sqrt{2\pi}}\int_{-\infty}^{+\infty}x\mathrm de^{-\frac{x^2}{2}}\\
&=\frac{3}{\sqrt{2\pi}}\int_{-\infty}^{+\infty}e^{-\frac{x^2}{2}}\mathrm dx=3
\end{align*}
故
\[
\var X_i^2=\E X_i^4-(\E X_i^2)^2=3-1=2
\]
故
\[
\var Z=\var \sum_{i=1}^nX_i^2=\sum_{i=1}^n\var X_i=2n
\]
\end{proof}


\begin{definition}[$t$ 分布]
设 $X\sim N(0,1)$,$Y\sim \chi^2(n)$,$X,Y$ 相互独立,则称
\[
t=\frac{X}{\sqrt{Y/n}}
\]
的分布为自由度为 $n$ 的 $t$ 分布,记作 $t\sim t(n)$. 
\end{definition}


\begin{definition}[$F$ 分布]
设 $X\sim\chi^2(n),Y\sim\chi^2(m)$,$X,Y$ 独立,则称
\[
Z=\frac{X/n}{Y/m}
\]
的分布为自由度为 $n,m$ 的 $F$ 分布,记作 $Z\sim F(n,m)$. 
\end{definition}


\begin{theorem}
设 $X_i\overset{\text{i.i.d.}}{\sim} N(\mu,\sigma^2)$,则样本均值服从期望相同、方差更小的正态分布:
\[
\bar X=\frac{1}{n}\sum\limits_{i=1}^nX_i\sim N\left(\mu,\frac{\sigma^2}{n}\right)
\]
\end{theorem}

\begin{theorem}
\label{thm:standard-normal}
设 $X_i\overset{\text{i.i.d.}}{\sim} N(\mu,\sigma^2)$,有样本均值的标准化:
\[
\frac{\bar X-\mu}{\sigma/\sqrt{n}}\sim N(0,1)
\]
\end{theorem}

\begin{theorem}
\label{thm:s2chi2}
设 $X_i\overset{\text{i.i.d.}}{\sim} N(\mu,\sigma^2)$,$S^2=\frac{1}{n-1}\sum\limits_{i=1}^n(X_i-\bar X)^2$ 为样本方差,则:
\[
\frac{(n-1)S^2}{\sigma^2}\sim \chi^2(n-1)
\]
\end{theorem}

\begin{theorem}
设 $X_i\overset{\text{i.i.d.}}{\sim} N(\mu,\sigma^2)$,$S^2=\frac{1}{n-1}\sum\limits_{i=1}^n(X_i-\bar X)^2$ 为样本方差,则 $\bar X$ 与 $S^2$ 独立。
\end{theorem}

\begin{theorem}
设 $X_i\overset{\text{i.i.d.}}{\sim} N(\mu,\sigma^2)$,$S^2=\frac{1}{n-1}\sum\limits_{i=1}^n(X_i-\bar X)^2$ 为样本方差,则:
\[
\frac{\sqrt n(\bar X-\mu)}{S}\sim t(n-1)
\]
\end{theorem}
\begin{proof}
根据定理 \ref{thm:standard-normal} 和定理 \ref{thm:s2chi2} 可知;
\[
\frac{\bar X-\mu}{\sigma/\sqrt{n}}\sim N(0,1),\quad
\frac{(n-1)S^2}{\sigma^2}\sim \chi^2(n-1)
\]
于是,根据 $t$ 分布的定义有:
\[
\cfrac{\cfrac{\bar X-\mu}{\sigma/\sqrt{n}}}{\sqrt{\frac{(n-1)S^2}{\sigma^2}/(n-1)}}=\frac{\sqrt n(\bar X-\mu)}{S}\sim t(n-1)
\]
\end{proof}

\begin{theorem}
设 $X_i\overset{\text{i.i.d.}}{\sim} N(\mu_1,\sigma_1^2)$,$i=1,2,\cdots,n$,样本方差为 $S_1^2$;$Y_i\overset{\text{i.i.d.}}{\sim} N(\mu_2,\sigma_2^2)$,$i=1,2,\cdots,m$,样本方差为 $S_2^2$,且 $X_i,Y_i$ 相互独立,则:
\[
\frac{S_1^2/\sigma_1^2}{S_2^2/\sigma_2^2}\sim F(n-1,m-1)
\]
\end{theorem}
\begin{proof}
由定理 \ref{thm:s2chi2} 知:
\[
\frac{(n-1)S_1^2}{\sigma_1^2}\sim\chi^2(n-1),\quad
\frac{(m-1)S_2^2}{\sigma_2^2}\sim \chi^2(m-1)
\]
于是根据 $F$ 分布的定义有:
\[
\frac{\dfrac{(n-1)S_1^2}{\sigma_1^2}/(n-1)}{\dfrac{(m-1)S_2^2}{\sigma_2^2}/(m-1)}=\frac{S_1^2/\sigma_1^2}{S_2^2/\sigma_2^2}\sim F(n-1,m-1)
\]
\end{proof}
