\PassOptionsToPackage{quiet}{fontspec}
\documentclass{article}


%%%%%%%%%%%%%%%%%%% usepackage %%%%%%%%%%%%%%%%%%%

\usepackage[a4paper, margin=1in]{geometry}

\usepackage{ctex}
\setCJKmainfont[Mapping = fullwidth-stop]{FandolSong-Regular}

\usepackage{amsthm, amsfonts, amsmath, amssymb}
\usepackage{mathtools}
\usepackage{centernot}
\usepackage{arydshln}

\usepackage{booktabs}
\usepackage{graphicx}
\usepackage{setspace}
\usepackage[hidelinks]{hyperref}
\usepackage{framed}
\usepackage{float}
\usepackage[bottom]{footmisc}
\usepackage{caption}

\usepackage{tcolorbox}
\tcbuselibrary{breakable}

\usepackage{xcolor}
\usepackage{multicol}
\usepackage{multirow}

\usepackage{enumitem}
\setlist{noitemsep}


%%%%%%%%%%%%%%%%%%%%%% math %%%%%%%%%%%%%%%%%%%%%%

\theoremstyle{definition}
\newtheorem{theorem}{定理}[section]
\newtheorem{corollary}[theorem]{推论}
\newtheorem{lemma}[theorem]{引理}
\newtheorem*{property}{性质}

\theoremstyle{definition}
\newtheorem{definition}{定义}[section]
\newtheorem{example}{例}[section]
\newtheorem{exercise}{练习}

\theoremstyle{remark}
\newtheorem*{remark}{注解}
\newtheorem*{note}{注意}
\newtheorem*{com}{注释}

\tcolorboxenvironment{example}{colback=blue!5!white, boxrule=0.5pt, breakable}
\tcolorboxenvironment{remark}{colback=green!10!white, boxrule=0.5pt, breakable}
\tcolorboxenvironment{note}{colback=yellow!10!white, boxrule=0.5pt, breakable}
\tcolorboxenvironment{proof}{colback=black!5!white, boxrule=0.5pt, breakable}


%%%%%%%%%%%%%%%%%% new command %%%%%%%%%%%%%%%%%%%
\newcommand{\ind}{\mathrel{\perp\!\!\!\!\perp}}
\newcommand{\cov}{\mathrm{cov}}
\newcommand{\var}{\mathrm{var}}
\newcommand{\E}{\mathbb{E}}
\newcommand{\R}{\mathbb{R}}
\newcommand{\Pb}{\mathrm{P}}
\newcommand\norm[1]{\Vert#1\Vert}
\renewcommand{\leq}{\leqslant}
\renewcommand{\geq}{\geqslant}
\newcommand{\argmin}{\mathop{\arg\min}}
\newcommand{\argmax}{\mathop{\arg\max}}

\newcommand{\TODO}[1]{{\color{red}{TODO:#1}}}


%%%%%%%%%%%%%%%%%%%%%% main %%%%%%%%%%%%%%%%%%%%%%

\begin{document}

\title{
    \rule{\linewidth}{2.0pt} \\
    \LARGE \textbf{\uppercase{笔记·《概率导论》}}
    \rule{\linewidth}{1.5pt} \\
    \vspace*{14\baselineskip}
}
\author{
    Yifeng Xu \\
    中国科学院计算技术研究所 \\
    中国科学院大学
}
\date{}

\maketitle \newpage

\nocite{*}
\section*{前言}

这是《概率导论》\cite{bertsekas2008introduction} 一书的笔记,但不包括第 6 章与第 7 章关于随机过程的部分(《随机过程》相关内容将单独记录)。另外,原书将离散随机变量和连续随机变量分别写在第 2 章和第 3 章中,导致许多概念需要重复书写,因此本笔记将这两章合并为了一章。最后,本笔记附录 \ref{sec:random-variables} 整理总结了常用随机变量及其期望、方差、矩母函数及性质,附录 \ref{sec:two-normal} 整理了二元正态分布相关内容,附录~\ref{sec:normal-derive} 整理了正态分布的三个导出分布相关内容。
 \newpage
\tableofcontents \newpage
\section{线性空间和线性变换}

\subsection{线性空间}

\begin{definition}[映射,变换,函数]
设有集合 $S,S'$,定义规则 $\sigma:S\to S'$,使得集合 $S$ 中元素 $a$ 与 $S'$ 中唯一元素对应,记作:
\[
a'=\sigma(a)\quad\text{或}\quad \sigma:a\mapsto a'
\]
则称 $\sigma$ 为一个映射。特别地,若 $S$ 与 $S'$ 相同,则称 $\sigma$ 为变换;若 $S'$ 为数域,则称 $\sigma$ 为函数。
\end{definition}
\begin{com}
映射最本质的特征在于对于 $S$ 中的任意一个元素在 $S'$ 中仅有唯一的一个元素和它对应。
\end{com}

\begin{definition}[集值映射]
由单点映射 $\sigma:S\to S'$ 可导出集值映射 $\sigma:2^S\to 2^{S'}$,
\begin{align*}
    \sigma(\Omega)&=\{y:y=\sigma(x),\exists\,x\in\Omega\},\quad\forall\,\Omega\subset S\\
    \sigma^{-1}(\Omega')&=\{x:y=\sigma(x),\exists\,y\in\Omega'\},\quad\forall\,\Omega'\subset S'
\end{align*}
如图所示:
\begin{figure}[H]
    \centering
    \includegraphics[width=0.25\linewidth]{figs/set-proj.png}
\end{figure}

\end{definition}

\begin{property}[集值映射的基本性质]
集值映射满足如下基本性质:
\begin{itemize}
    \item 任给子集 $\Omega\subset S$,则 $\sigma^{-1}(\sigma(\Omega))=\Omega$.
    \item 任给子集 $\Omega'\subset S'$,则 $\sigma(\sigma^{-1}(\Omega'))=\Omega'\cap \sigma(S)$.
\end{itemize}
如果用包含关系定义子集间的偏序关系,那么由映射导出的集值映射是保持这种偏序关系的,即:
\begin{align*}
&\Omega_1\subset\Omega_2\subset S\implies \sigma(\Omega_1)\subset \sigma(\Omega_2)\subset \sigma(S)\\
&\Omega'_1\subset\Omega'_2\subset S\implies \sigma^{-1}(\Omega'_1)\subset \sigma^{-1}(\Omega'_2)\subset \sigma^{-1}(S)
\end{align*}
\end{property}

\begin{definition}[线性空间/向量空间]
设 $V$ 是一个非空集合,它的元素用 $x,y,z$ 等表示,称为向量;$\mathbb F$ 是一个数域,它的元素用 $k,l,m$ 等表示,称为标量。定义加法与数乘:
\begin{gather*}
    +:V\times V\to V,\;(x,y)\mapsto x+y\\
    \cdot:\mathbb F\times V\to V,\;(k,x)\mapsto k\cdot x
\end{gather*}
如果加法和数乘满足下列 8 条性质:
\begin{itemize}
    \item 结合律:$x+(y+z)=(x+y)+z$
    \item 交换律:$x+y=y+x$
    \item 存在零元素:$x+0=x$
    \item 存在负元素:$x+(-x)=0$
    \item 数因子分配律:$k(x+y)=kx+ky$
    \item 分配律:$(k+l)x=kx+lx$
    \item 结合律:$k(lx)=(kl)x$
    \item 存在单位元:$1x=x$
\end{itemize}
则称 $(V,\mathbb F,+,\cdot)$ 为数域 $\mathbb F$ 上的线性空间或向量空间,简记作 $V$. 特别地,当 $\mathbb F$ 为实数域 $\mathbb R$ 时,称之为实线性空间;当 $\mathbb F$ 为复数域 $\mathbb C$ 时,称之为复线性(酉)空间。
\end{definition}

\begin{remark}
集合中的元素无需一定是列向量,可以是矩阵、多项式等;加法和数乘也不一定是我们熟悉的加法和数乘,只要满足上述 8 条性质即可。因此线性空间是多种多样的。
\end{remark}

\begin{example}
\label{ex:linearspace-pn}
次数不超过$n−1$的多项式$P_n$全体按照通常的多项式加法和数乘构成一个线性的多项式函数空间。
\end{example}
\begin{example}
$n$维实列向量的全体按照通常的向量加法和数乘构成一个实线性空间,称为实向量空间,记作 $\mathbb R^n$.
\end{example}
\begin{example}
所有$m\times n$实矩阵的全体按照通常的矩阵加法和数乘构成一个实线性空间,称为矩阵空间。
\end{example}
\begin{example}
\label{ex:linearspace}
取 $V=\mathbb R^n$,对 $x,y\in V,\,k\in\mathbb R$,定义加法 $\oplus$ 和数乘 $\odot$:
\begin{align*}
    &x\oplus y=\left((x_1^3+y_1^3)^{1/3},(x_2^3+y_2^3)^{1/3}\ldots,(x_n^3+y_n^3)^{1/3}\right)^T\\
    &k\odot x=k^{1/3}x
\end{align*}
则 $(\mathbb R^n,\mathbb R,\oplus,\odot)$ 构成一个线性空间。
\end{example}
\begin{example}
\label{ex:linearspace2}
取 $V=\mathbb R^+$,对 $x,y\in V,\,k\in\mathbb R$,定义加法 $\oplus$ 和数乘 $\odot$:
\begin{align*}
    &x\oplus y=x\cdot y\\
    &k\odot x=x^k
\end{align*}
则 $(\mathbb R^+,\mathbb R,\oplus,\odot)$ 构成一个线性空间。
\end{example}

\begin{definition}[线性相关,线性无关]
若存在一组不全为零的数$c_1,c_2,\ldots,c_m$,使得 $c_1x_1+c_2x_2+\cdots+c_mx_m=0$,则称向量组$x_1,x_2,\ldots,x_m$线性相关,否则称为线性无关。
\end{definition}

\begin{definition}[极大线性无关组]
若对一个线性无关的向量组,再往里添加向量就无法保持它们的线性无关性,那么称该向量组为极大线性无关组。
\end{definition}

\begin{theorem}[有限维空间的基本假设]
线性无关组总是可以扩充为极大线性无关组。
\end{theorem}

\begin{lemma}
\label{lemma:maxlinear}
对一个线性空间中的任两个极大线性无关组,若它们的所含向量个数都有限,则所含向量个数一定相同。
\end{lemma}

\begin{proof}
设 $x_1,x_2,\ldots,x_m$ 和 $y_1,y_2,\ldots,y_n$ 为线性空间 $V$ 中的两个极大线性无关组,则存在矩阵 $A,B$ 使得:
\begin{align*}
    &(x_1,x_2,\ldots,x_m)=(y_1,y_2,\ldots,y_n)A\\
    &(y_1,y_2,\ldots,y_n)=(x_1,x_2,\ldots,x_m)B
\end{align*}
联立上述二式得:
\[
    (y_1,y_2,\ldots,y_n)=(y_1,y_2,\ldots,y_n)AB
\]
由于 $y_1,y_2,\ldots,y_n$ 是极大线性无关组,所以只能有 $AB=I_n$,其中 $I_n$ 为 $n$ 阶单位阵,于是:
\[
    \text{tr}(AB)=\text{tr}(I_n)=n
\]
同理可得:
\[
    \text{tr}(BA)=\text{tr}(I_m)=m
\]
根据迹算子的性质,有:
\[
    n=\text{tr}(AB)=\text{tr}(BA)=m
\]
即这两个极大线性无关组所含向量个数相同。
\end{proof}

\begin{definition}[线性空间的维数]
定义为线性空间 $V$ 的维数为其中极大线性无关组的所含向量的个数,记作 $\dim V$. 维数有限的称为有限维空间,否则称为无穷维空间。
\end{definition}

\begin{com}
这个定义之所以是良定义的,是因为引理 \ref{lemma:maxlinear} 说明了有限维空间中,不同极大线性无关组所含向量的个数是相同的。
\end{com}

\begin{remark}
本课程只讨论有限维空间,无穷维空间属于泛函分析的范畴。有限维空间下得到的结论有些可以直接推广到无穷维空间,但有些却不可能。必须小心!
\end{remark}

\begin{definition}[基]
若线性空间 $V$ 的向量 $x_1,x_2,\ldots,x_n$ 满足:
\begin{enumerate}
    \item $x_1,x_2,\ldots,x_n$ 线性无关;
    \item $V$ 中任意向量都是 $x_1,x_2,\ldots,x_n$ 的线性组合。
\end{enumerate}
则称 $x_1,x_2,\ldots,x_r$ 为 $V$ 的一个基或基底,相应地称 $x_i$ 为基向量。
\end{definition}

\begin{note}
组成基的向量排列是\textbf{有顺序}的!这是因为向量在这个基下的坐标表示是有顺序的,例如 $(1,2)\neq (2,1)$.
\end{note}

\begin{corollary}
线性空间中任意一个极大无关组构成它的一个基。
\end{corollary}

\begin{definition}[坐标表示]
设 $X=(x_1,\ldots,x_n)$ 是一个基,若向量 $x$ 在这个基下的线性表示为:
\[x=\xi_1 x_1+\cdots+\xi_nx_n=X\xi\]
则称 $x$ 在 $X$ 下的坐标表示为 $\xi=(\xi_1,\ldots,\xi_n)^{T}$.
\end{definition}

\begin{remark}
式 $x=X\xi$ 非常重要,日后将经常使用。其中 $X=(x_1,\ldots,x_n)$ 表示向量组而非矩阵,$X\xi$ 也并非矩阵乘法,只是可以按照矩阵乘法来理解。
\end{remark}

\begin{example}
考虑例 \ref{ex:linearspace-pn} 中介绍的多项式空间 $P_n$,选择 $P_n$ 中的一个基:
\[
    x_1=1,\,x_2=x,x_3=x^2,\ldots,x_n=x^{n-1}
\]
则任意次数不超过 $n-1$ 的多项式 $f(x)=a_0x^{n-1}+a_1x^{n-2}+\cdots+a_{n-2}x+a_{n-1}$ 可写作:
\[
    f(x)=(1,x,x^2,\ldots,x^{n-1})(a_{n-1},a_{n-2},\ldots,a_0)^T
\]
于是 $(a_{n-1},a_{n-2},\ldots,a_0)^T$ 就是多项式 $f(x)$ 在基 $1,x,x^2,\ldots,x^{n-1}$ 下的坐标。
\end{example}

\begin{definition}[坐标表示与映射]
设 $V$ 为一个 $n$ 维线性空间,$X$ 是 $V$ 的一个基,那么向量在基 $X$ 下的坐标表示定义了一个从 $V$ 到 $\mathbb R^n$ 或 $\mathbb C^n$ 的一一映射:
\[
    \sigma: V\to\mathbb R^n (\mathbb C^n),\; x\mapsto (\xi_1,\xi_2,\ldots,\xi_n)^T\in\mathbb R^n (\mathbb C^n)
\]
\end{definition}

\begin{theorem}
任何 $n$ 维线性空间 $V$ 都与 $\mathbb R^n$ 或 $\mathbb C^n$ 代数同构,即存在一一映射 $\sigma:V\to\mathbb R^n(\mathbb C^n)$ 满足:
\begin{align*}
    &\sigma(x+y)=\sigma(x)+\sigma(y),\quad\forall x,y\in V\\
    &\sigma(kx)=k\sigma(x),\quad\forall x\in V,\,k\in \mathbb F
\end{align*}
\end{theorem}

\begin{proof}
充分性:验证 8 条性质即可;必要性:任给 $V$ 中的一个基,那么该基下的坐标表示就是一个符合条件的同构映射。
\end{proof}

上述定理说明,虽然 $n$ 维线性空间有无穷多,但是\textbf{所有线性空间都与 $\mathbb R^n$ 或 $\mathbb C^n$ 代数同构,因此只需研究 $\mathbb R^n$ 和 $\mathbb C^n$ 就足够了}。既然如此,为什么我们还要引入抽象的一般化的线性空间的定义呢?这是因为:1.可以把讨论的结论适用于更广的范围;2.由线性映射全体构成的线性空间,可以推广到无穷维的线性空间中的线性函数全体构成的线性空间(泛函);3.利于引进更多的代数运算,如向量的代数乘法,从而引出李代数、结合代数等。

% \begin{example}
% 思考:对于同一个集合 $V$,能否定义不同的加法和数乘使得线性空间 $(V,\mathbb R,\oplus,\odot)$ 和 $(V,\mathbb R,\oplus',\odot')$ 的维数不同?
% 以 $V=[0,1)$ 为例,我们可以定义适当的加法和数乘使得线性空间的维数为 $1,2,\ldots,\infty$,做法如下。考虑在 $[0,1)$ 与 $[0,1)\times[0,1)$ 之间建立一一映射:任给 $x\in[0,1)$,设其二进制小数表示为 $x=0.x_1x_2x_3x_4\ldots$,则定义映射 $\sigma$ 为:
% \[
%     \sigma:x\mapsto y=\begin{bmatrix}y_1\\y_2\end{bmatrix}=\begin{bmatrix}0.x_1x_3\ldots\\0.x_2x_4\ldots\end{bmatrix}
% \]
% 可以验证 $\sigma$ 的确是一一映射。于是对于 $x,y\in[0,1)$,定义加法和数乘为:
% \begin{align*}
%     &\oplus: (x,y)\mapsto x\oplusy=\sigma^{-1}(\sigma(x)+\sigma(y))\\
%     &\odot: (k,x)\mapsto k\odotx=\sigma^{-1}(k\cdot\sigma(k))
% \end{align*}
% 其中 $+$ 和 $\cdot$ 就是 $[0,1)\times[0,1)$ 上常规的向量加法和数乘,那么 $\sigma$ 就是一个同构映射。因此,$([0,1),\mathbb R,\oplus,\odot)$ 就是一个二维的线性空间。(FIXME:数域不太对)
% \end{example}

\begin{definition}[基变换与过渡矩阵]
设有两个基 $X=(x_1,\ldots,x_n)$,$Y=(y_1,\ldots,y_n)$,$Y$ 中每一个基向量由 $X$ 的基向量线性表示为:
\[
    \begin{cases}
    y_1=c_{11}x_1+\cdots+c_{n1}x_n\\
    \quad\vdots\\
    y_n=c_{1n}x_1+\cdots+c_{nn}x_n\\
    \end{cases}
\]
写作矩阵形式为:
\[Y=XC\]
称该式为基变换公式,称 $C$ 为过渡矩阵。
\end{definition}

\begin{remark}
这里 $X,Y$ 都是向量组而非矩阵,$XC$ 也并非矩阵乘法,只是可以按矩阵乘法来理解。
\end{remark}

\begin{corollary}
\label{cor:transmatrix}
过渡矩阵一定非奇异。
\end{corollary}
\begin{proof}
考虑反证法。假设 $C$ 奇异,则存在向量 $\xi\in\mathbb R^n(\mathbb C^n)$ 且 $\xi\neq 0$ 使得 $C\xi=0$. 于是 $Y\xi=XC\xi=0$,即 $\xi_1y_1+\xi_2y_2+\cdots+\xi_ny_n=0$,这与 $Y=(y_1,y_2,\ldots,y_n)$ 是一个基矛盾。
\end{proof}

\begin{com}
从推论 \ref{cor:transmatrix} 我们可以发现,任何一个非奇异矩阵都可以看成是线性空间的两个基之间的过渡矩阵,换句话说,是一个基在另一个基下的坐标表示。
\end{com}

\begin{theorem}[向量在不同基下的表示坐标的关系]
设有一向量 $x$,在两个基下的坐标表示分别为 $\xi=(\xi_1,\ldots,\xi_n)^T$ 和 $\eta=(\eta_1,\ldots,\eta_n)^T$,设 $C$ 为过渡矩阵,则有 $\xi=C\eta$ 或 $\eta=C^{-1}\xi$.
\end{theorem}
\begin{proof}
$x=X\xi=Y\eta\implies\xi=C\eta\iff\eta=C^{-1}\xi$
\end{proof}

\begin{remark}
式 $Y=XC$ 和 $\xi=C\eta$ 日后也将经常使用。
\end{remark}

\begin{com}
自然基下,向量 $x$ 和坐标表示是一致的,常常不加区别地用同一符号表示。
\end{com}

\begin{definition}[线性子空间]
设 $V_1$ 是数域 $\mathbb F$ 上线性空间 $V$ 的非空子集合,且满足:
\begin{enumerate}
    \item 对 $V$ 上的加法封闭:若 $x,y\in V_1$,则 $x+y\in V_1$;
    \item 对 $V$ 上的数乘封闭:若 $x\in V_1,\,k\in \mathbb F$,则 $kx\in V_1$.
\end{enumerate}
则称 $V_1$ 是 $V$ 的线性子空间或子空间。
\end{definition}

\begin{definition}[零子空间]
称仅由 $0$ 元素构成的子空间为零子空间,其维度为 0.
\end{definition}

\begin{definition}[子空间的和与直和]
设 $V_1,V_2$ 是线性空间 $V$ 的子空间,则它们的和定义为:
\[
    V_1+V_2=\{z\mid z=x+y,\,x\in V_1,\,y\in V_2\}
\]
当 $V_1\cap V_2=\{0\}$ 时,称它们的和为直和,记作 $V_1\oplus V_2$.
\end{definition}

\begin{property}[直和]
对 $z\in V_1\oplus V_2$,存在唯一的 $x\in V_1,\,y\in V_2$ 使得 $z=x+y$.
\end{property}
\begin{proof}
存在性显然,下证唯一性。假设 $z=x_1+y_1=x_2+y_2$,其中 $x_1,x_2\in V_1,\,y_1,y_2\in V_2$,那么 $x_1-x_2=y_2-y_1$. 由于 $x_1-x_2\in V_1,\,y_2-y_1\in V_2$ 且 $V_1\cap V_2=0$,因此只能是 $x_1=x_2,\,y_1=y_2$.
\end{proof}

% \begin{com}
% 这个性质实际构成了直和定义的背景,也就是说,如果知道 $V_1$ 和 $V_2$,我们可以得到 $V_1+V_2$;但是反过来,如果知道两个子空间 $V_3$ 和 $V_1$,且 $V_1\subset V_3$,我们是否可以知道唯一的子空间 $V_2$,使得 $V_1+V_2=V_3$?直和的定义告诉我们应该如何约束 $V_2$ 以得到唯一性。
% \end{com}

\begin{theorem}
子空间的交仍然是子空间,子空间的和仍然是子空间。
\end{theorem}
\begin{proof}
验证是否对加法和数乘封闭即可,略。
\end{proof}

\begin{theorem}[子空间和的维数公式]
设 $V_1,V_2$ 为线性空间 $V$ 的子空间,则:
\[
    \dim V_1+\dim V_2=\dim (V_1+V_2)+\dim (V_1\cap V_2)
\]    
\end{theorem}

\begin{remark}[基扩充]
\textbf{基扩充}是一种非常重要的证明思路:从最小的子空间 $V_1\cap V_2$ 出发,构造它的一个基 $X=(x_1,\ldots,x_r)$,然后分别扩充成 $V_1$ 的基 $(X,Y)=(x_1,\ldots,x_r,y_1,\ldots,y_s)$ 和 $V_2$ 的基 $(X,Z)=(x_1,\ldots,x_r,z_1,\ldots,z_t)$,最后证明 $(X,Y,Z)=(x_1,\ldots,x_r,y_1,\ldots,y_s,z_1,\ldots,z_t)$ 为 $V_1+V_2$ 的基。    
\end{remark}

\begin{proof}
延续基扩充的证明思路,要证明 $(X,Y,Z)=(x_1,\ldots,x_r,y_1,\ldots,y_s,z_1,\ldots,z_t)$ 为 $V_1+V_2$ 的基,只需证明 1) 线性无关;2) 可线性表示任一 $v\in V_1+V_2$.

1) 首先证明线性无关。设:
\[
    \sum_{i=1}^r a_ix_i+\sum_{j=1}^s b_jy_j+\sum_{k=1}^t c_kz_k=0
\]
则:
\[
    \sum_{i=1}^r a_ix_i+\sum_{j=1}^s b_jy_j=-\sum_{k=1}^t c_kz_k\in V_1
\]
由于 $z_k\notin V_1$,故 $c_k=0$,进而 $a_i=b_j=0$. 故线性无关。

2) 其次,任取 $v\in V_1+V_2$,则 $\exists v_1\in V_1,v_2\in V_2$ 使得 $v_1+v_2=v$. 设:
\[
    v_1=\sum_{i=1}^r a_ix_i+\sum_{j=1}^s c_jy_j,\quad v_2=\sum_{i=1}^r b_ix_i+\sum_{k=1}^t d_kz_k
\]
则:
\[
    v=\sum_{i=1}^r(a_i+b_i)x_i+\sum_{j=1}^s c_jy_j+\sum_{k=1}^t d_kz_k
\]
综上,$(X,Y,Z)$ 是 $V_1+V_2$ 的一个基。    
\end{proof}

\begin{corollary}
设 $V_1,V_2$ 为线性空间 $V$ 的子空间且 $V_1\cap V_2=\mathbf 0$,则:
\[
    \dim V_1+\dim V_2=\dim (V_1\oplus V_2)
\]
\end{corollary}

\begin{definition}
由向量组扩张为子空间:
\begin{itemize}
    \item 由单个向量 $x$ 对数乘运算封闭构成一维子空间:$L(x)=\{z\mid z=kx,k\in K\}$;
    \item 由向量组 $x_1,\ldots,x_m$ 扩张成的子空间:$L(x_1,\ldots,x_m)=L(x_1)+\cdots+L(x_m)$.
\end{itemize}
显然 $\dim L(x_1,\ldots,x_m)\leq m$.    
\end{definition}

通过前面的讨论我们知道了一般的线性空间都和 $\mathbb R^n$ 或 $\mathbb C^n$ 代数同构;另外,$\mathbb R^n$ 或 $\mathbb C^n$ 也是我们最为熟悉的线性空间。因此,我们将它们作为一般线性空间的代表进行讨论与研究,以此来研究抽象的线性空间的性质。为方便起见,下文我们用字母 $\mathbb F$ 表示 $\mathbb R$ 或 $\mathbb C$.

考虑 $\mathbb F^m$ 空间中的 $n$ 个向量 $a_1,\ldots,a_n\in\mathbb F^m$,将他们排列在一起即得到矩阵 $A=(a_1,\ldots,a_n)\in\mathbb F^{m\times n}$,此时称 $a_i$ 为矩阵 $A$ 的列向量。据此,我们下面介绍矩阵的几个基本属性——秩、值域/列空间、核空间/零空间。

\begin{definition}[列秩,行秩]
设 $A\in\mathbb F^{m\times n}$,称其列向量构成的极大线性无关组的大小为 $A$ 的列秩,称其行向量构成的极大线性无关组的大小为 $A$ 的行秩。
\end{definition}

\begin{theorem}[行秩等于列秩]
设 $A\in\mathbb F^{m\times n}$,则 $A$ 的行秩等于 $A$ 的列秩。
\end{theorem}
\begin{proof}
设 $A$ 的行秩为 $r$,其行向量组构成的一个极大线性无关组为 $(c_1,c_2,\ldots,c_r)$,其中 $c_i$ 为行向量,记矩阵:
\[
    C=\begin{bmatrix}c_1\\c_2\\\vdots\\c_r\end{bmatrix}\in\mathbb F^{r\times n}
\]
则 $A$ 的所有行向量都可以写作 $c_1,c_2,\ldots,c_r$ 的线性组合,用矩阵乘法来表示,即存在矩阵 $B=(b_1,b_2,\ldots,b_r)\in\mathbb F^{m\times r}$,使得
\[
    A_{m\times n}=B_{m\times r}C_{r\times n}
\]
其中 $b_1,b_2,\ldots,b_r$ 是 $B$ 的列向量。换个角度看待矩阵乘法,上式也意味着 $A$ 的所有列向量都可以写作 $b_1,b_2,\ldots,b_r$ 的线性组合,因此:$A \text{ 的列秩} \leq r=A \text{ 的行秩}$. 同理可得 $A \text{ 的行秩}\leq A \text{ 的列秩}$,故行秩等于列秩。
\end{proof}

\begin{definition}[秩]
由于矩阵 $A$ 的列秩与行秩相等,因此统一称作秩,记作 $\text{rank}(A)$.
\end{definition}

\begin{definition}[值域/列空间]
设 $A=(a_1,\ldots,a_n)\in\mathbb F^{m\times n}$,称其列向量张成的子空间 $L(a_1,\ldots,a_n)$ 为矩阵 $A$ 的值域或列空间,记作 $R(A)$,即:
\[
    R(A)=L(a_1,\ldots,a_n)\subset\mathbb F^m
\]
根据定义立刻可知,$\text{rank}(A)=\dim(R(A))$.
\end{definition}

\begin{definition}[核空间/零空间]
设 $A\in\mathbb F^{m\times n}$,称集合 $\{x\mid Ax=0\}$ 为矩阵 $A$ 的核空间或零空间,记作 $N(A)$,即:
\[
    N(A)=\{x\mid Ax=0\}\subset\mathbb F^n
\]
\end{definition}

\begin{theorem}
设 $A\in\mathbb F^{m\times n}$,则 $\dim(R(A))+\dim(N(A))=n$.
\end{theorem}
\begin{proof}
证明思路依旧是\textbf{基扩充}:设 $(x_1,\ldots,x_s)$ 为 $N(A)$ 的一个基,将其扩充为 $\mathbb R^n$ 的基 $(x_1,\ldots,x_s,y_1,\ldots,y_{n-s})$.  只需证明 $(Ay_1,\ldots, Ay_{n-s})$ 是 $R(A)$ 的基。

首先证明线性无关。假设:
\[
    \sum_{j=1}^{n-s}b_j(Ay_j)=0
\]
由于 $Ax_i=0\ (i=1,\ldots,s)$,所以:
\[
    \sum_{j=1}^{n-s}b_jAy_j=\sum_{i=1}^sa_i Ax_i+\sum_{j=1}^{n-s}b_jAy_j=A\left(\sum_{i=1}^sa_i x_i+\sum_{j=1}^{n-s}b_jy_j\right)=0
\]
也就是说:
\[
    \sum_{i=1}^sa_i x_i+\sum_{j=1}^{n-s}b_jy_j\in N(A)
\]
但是 $(x_1,\ldots,x_s)$ 与 $(y_1,\ldots,y_{n-s})$ 是线性无关的,所以只能是 $b_j=0, (j=1,\ldots,n-s)$. 因此线性无关。

其次证明可线性表示。任取 $z\in R(A)$,则存在 $w\in\mathbb R^n$ 使得 $z=Aw$.  设 $w$ 在 $(x_1,\ldots,x_s,y_1,\ldots,y_{n-s})$ 这个基下可以线性表示为:
\[
    w=\sum_{i=1}^s c_ix_i+\sum_{j=1}^{n-s}d_j y_j
\]
那么 $z$ 可以由 $(Ay_1,\ldots,Ay_{n-s})$ 线性表示为:
\[
    z=Aw=\sum_{i=1}^s c_i (Ax_i)+\sum_{j=1}^{n-s}d_j(Ay_j)=\sum_{j=1}^{n-s}d_j(Ay_j)
\]

\end{proof}

\begin{remark}
Gilbert Strang 的著名的四个基本子空间:
\begin{figure}[H]
    \centering
    \includegraphics[width=0.55\linewidth]{figs/space.png}
\end{figure}
\end{remark}


\subsection{线性映射,线性变换及其矩阵表示}

表示是什么?表示究其本质来说是一种映射,它把我们不熟悉或抽象的事物映射为我们熟知或具体的事物。例如:抽象的线性空间在一个基下可表示为实或复的列向量空间。同样地,线性空间之间的线性映射都可以表示为矩阵,这正是矩阵的代数本质所在。这就是本节所研究的内容。


\begin{definition}[线性映射]
设有数域相同的线性空间 $X$ 到线性空间 $Y$ 的映射 $T$,若满足:
\begin{itemize}
    \item $T(x+y)=T(x)+T(y)$
    \item $T(kx)=kT(x)$
\end{itemize}
则称 $T$ 为 $X$ 到 $Y$ 的线性映射。
\end{definition}

\begin{definition}[线性映射的矩阵表示]
\label{def:matrixrepr}
设有 $m$ 维线性空间 $W$ 和 $n$ 维线性空间 $V$,$X=(x_1,\ldots,x_m)$,$Y=(y_1,\ldots,y_n)$ 分别是 $W,V$ 的基。$X$ 被 $T$ 映射到 $V$ 中后可以由 $Y$ 线性表示,即:
\[
\begin{cases}
Tx_1=a_{11}y_1+\cdots+a_{n1}y_n\\
\quad\vdots\\
Tx_m=a_{1m}y_1+\cdots+a_{nm}y_n\\
\end{cases}
\]
写作矩阵形式为:
\[
TX=Y
\underbrace{\begin{bmatrix}
a_{11}&\cdots&a_{1m}\\
\vdots&\ddots&\vdots\\
a_{n1}&\cdots&a_{nm}\\
\end{bmatrix}}_A
\]
称 $A\in\mathbb R^{n\times m}$ 为 $T$ 在基 $X,Y$ 下的矩阵表示。
\end{definition}

\begin{remark}
式 $TX=YA$ 非常重要,日后将经常使用。
\end{remark}

\begin{theorem}[向量在不同基下的表示坐标的关系]
设向量 $x\in W$ 在 $X$ 下的坐标表示为 $\xi$,$Tx\in V$ 在 $Y$ 下的坐标表示为 $\eta$,$T$ 在 $X,Y$ 下的矩阵表示为 $A$,那么有 $\eta=A\xi$. 可视化如下:
\[
\begin{array}{ccccccc}
    T: & x   & \mapsto & y    & = & T & x   \\
    \downarrow & \downarrow & & \downarrow & & \downarrow & \downarrow \\
    A: & \xi & \mapsto & \eta & = & A & \xi
\end{array}
\]
\end{theorem}
\begin{proof}
$Tx=T(X\xi)=(TX)\xi=(YA)\xi=Y(A\xi)=Y\eta\implies \eta=A\xi$
\end{proof}

从定义 \ref{def:matrixrepr} 可以看见,\textbf{线性映射的矩阵表示依赖于基的选取},即 $A=\sigma(T;X,Y)$. 既然如此,一个自然的问题就是,同一个线性映射在不同基下的矩阵表示有什么关系呢?

\begin{theorem}[同一个线性映射在不同基下的矩阵表示的关系]
设 $W,V$ 空间中的另一组基为 $X',Y'$,且 $X'=XC,\,Y'=YD$,其中 $C,D$ 为过渡矩阵(因而可逆),那么有 $A'=D^{-1}AC$. 
注意其中 $D\in\mathbb R^{n\times n},\,A\in\mathbb R^{n\times m},\,C\in\mathbb R^{m\times m}$.
\end{theorem}
\begin{proof}
$TX'=T(XC)=(TX)C=(YA)C=Y'D^{-1}AC=Y'A'\implies A'=D^{-1}AC$
\end{proof}

\begin{definition}[线性映射的复合]
设 $S:W\to V,\,T: V\to U$,定义它们的复合为 $(T\circ S)(x)=T(S(x))$. 显然,线性映射的复合仍为线性映射。
\end{definition}

\begin{theorem}[复合线性映射的矩阵表示]
设 $W,V,U$ 下各有基 $X,Y,Z$,在这些基下 $S,T$ 的矩阵表示分别为:$A=\sigma(S;X,Y)$,$B=\sigma(T;Y,Z)$,则复合映射 $T\circ S$ 的矩阵表示为 $BA$.
\end{theorem}
\begin{proof}
$(T\circ S)(X)=T(S(X))=T(YA)=(TY)A=(ZB)A=Z(BA)$
\end{proof}

\begin{com}
可以看见 $BA$ 只与 $X,Z$ 有关,与 $Y$ 无关,即 $BA=\sigma(T\circ S;X,Z)$. 事实上,我们可以选取 $V$ 的另一组基证明这一点:设 $Y'$ 也是 $V$ 的基且 $Y=Y'C$,那么:
\begin{align*}
    &SX=YA=(Y'C)A=Y'(CA)=Y'A'\implies CA=A'\\
    &TY=T(Y'C)=(TY')C=(ZB')C=Z(B'C)=ZB\implies B'C=B
\end{align*}
因此 $BA=(B'C)A=B'(CA)=B'A'$.
\end{com}

\begin{theorem}
设$T$为线性空间$W$到线性空间$V$的线性映射,则 $W$ 内的线性子空间$W_1$在$V$中的象$V_1$为$V$的线性子空间。反之,$V$ 中的线性子空间 $V_1$ 的逆象 $T^{-1}(V_1)=\{x\mid \exists y\in V_1,\,y=Tx\}$ 也是 $W$ 中的子空间。
\end{theorem}
\begin{proof}
利用子空间的定义易证(验证是否满足两条要求即可)。
\end{proof}

\begin{theorem}
设$T$为线性空间$W$到线性空间$V$的线性映射,$W_1,W_2$为$W$内的子空间,则:
\begin{itemize}
    \item $T(W_1+W_2)=T(W_1)+T(W_2)$
    \item $T(W_1\cap W_2)\subset T(W_1)\cap T(W_2)$
\end{itemize}
\end{theorem}

\begin{definition}[线性映射的值域和核]
设有线性映射 $T:W\to V$,则有与矩阵类似的定义:
\begin{itemize}
    \item 值域:$R(T)=\{y\in V\mid y=Tx,\forall x\in W\}$
    \item 核:$N(T)=\{x\mid Tx=0,x\in W\}$
    \item 秩:$\dim(R(T))$
    \item 亏度:$\dim(N(T))$
\end{itemize}
\end{definition}

\begin{theorem}[线性映射的维数公式]
设有线性映射 $T:W\to V$,则:
\[\dim(R(T))+\dim(N(T))=\dim(W)\]
\end{theorem}
\begin{proof}
可以用基扩充的方式来证明,与矩阵的维数公式类似,此处略去。
\end{proof}

\begin{theorem}[线性映射构成的空间]
设有线性映射 $T_1,T_2: W\to V$,定义加法和数乘如下:
\begin{itemize}
    \item 加法:$(T_1+T_2)(x)=T_1(x)+T_2(x)$
    \item 数乘:$(kT_1)(x)=k(T_1(x))$
\end{itemize}
则 $W$ 到 $V$ 的线性映射全体在上述加法和数乘下构成一个线性空间。
\end{theorem}

\begin{remark}
根据上文的讨论,我们知道线性映射在给定基后可以用矩阵表示。因此\textbf{我们可以借助矩阵来研究线性映射的性质,或借助线性映射来研究矩阵的性质}。例如下面的定理。
\end{remark}

\begin{theorem}[线性映射复合的维数公式]
设 $A\in\mathbb C^{m\times n}$,$B\in\mathbb C^{n\times p}$,则:
\begin{align*}
    &\dim(N(AB))=\dim(N(B))+\dim(N(A)\cap R(B))\\
    &\dim(R(AB))=\dim(R(B))-\dim(N(A)\cap R(B))
\end{align*}
\end{theorem}
\begin{proof}
首先证明第一个式子,依旧采用基扩充的思路。存在一组线性无关的 $x_1,\ldots,x_r\in\mathbb C^p$ 使得 $(Bx_1,\ldots,Bx_r)$ 为 $N(A)\cap R(B)$ 的一个基,再取 $N(B)$ 的一个基 $(y_1,\ldots,y_s)$,则只需要证明 $(x_1,\ldots,x_r,y_1,\ldots,y_s)$ 构成 $N(AB)$ 的一个基即可。

首先证明线性无关。由于 $(x_1,\ldots,x_r)$ 线性无关,$(y_1,\ldots,y_s)$ 线性无关,因此只需要证明 $y_j\ (j=1,\ldots,s)$ 与 $(x_1,\ldots,x_r)$ 线性无关即可。这是容易的,因为 $y_j\in N(B),\,x_i\in R(B)$,而 $N(B)\cap R(B)=\{0\}$.

其次,任取 $z\in N(AB)$,那么 $ABz=0$.  当 $Bz=0$ 时,$z\in N(B)$,可以被 $(y_1,\ldots,y_s)$ 线性表示;当 $Bz\neq 0$ 时,$Bz\in N(A)\cap R(B)$,因此 $Bz$ 可以被 $(Bx_1,\ldots,Bx_r)$ 线性表示,即:
\[
    Bz=\sum_{i=1}^r a_i Bx_i\implies B\left(z-\sum_{i=1}^ra_ix_i\right)=0
\]
但由于 $z,x_i\notin N(B)$,所以只能是括号内为零,即 $z$ 可以被 $(x_1,\ldots,x_r)$ 线性表示。

对于第二个式子,可以类似地采用基扩充的思路证明,这里选择另一种方法。利用上一条定理的结论,结合:
\[
    \dim(R(AB))+\dim(N(AB))=\dim(R(B))+\dim(N(B))=p
\]
即可推出结论。
\end{proof}

\begin{remark}
将矩阵 $A,B$ 看作线性映射,那么这两条定理可以直观地按下图理解:
\begin{figure}[H]
    \centering
    \includegraphics[width=0.55\linewidth]{figs/mapping.png}
\end{figure}
\begin{itemize}
    \item $N(AB)$ 包含被 $B$ 映射到了 $0$ 的部分和没被 $B$ 映射到 $0$、但被 $A$ 映射到 $0$ 的部分。
    \item $R(AB)$ 是没有被 $B$ 映射到 $0$ 的部分中,也没有被 $A$ 映射到 $0$ 的部分。
\end{itemize}
\end{remark}

\begin{corollary}
设 $A\in\mathbb C^{m\times n}$,$B\in\mathbb C^{n\times p}$,$C\in\mathbb C^{p\times q}$,则:
\begin{align*}
    &\text{rank}(A)+\text{rank}(B)-n\leq \text{rank}(AB)\\
    &\dim(R(AB))+\dim(R(BC))-\dim(R(B))\leq \dim(R(ABC))
\end{align*}
\end{corollary}
\begin{proof}
1) 由于:
\[
    \dim(R(B))-\dim(R(AB))=\dim(N(A)\cap R(B))\leq\dim(N(A))=n-\dim(R(A))
\]
等号成立当且仅当 $N(A)\subset R(B)$. 所以:
\[
    \dim(R(A))+\dim(R(B))-n\leq\dim(R(AB))
\]
即 $\text{rank}(A)+\text{rank}(B)-n\leq \text{rank}(AB)$.

2) 类似地,由于:
\begin{align*}
    \dim(R(BC))-\dim(R(ABC))&=\dim(N(A)\cap R(BC))\\
    &\leq \dim(N(A)\cap R(B))=\dim(R(B))-\dim(R(AB))
\end{align*}
所以:
\[
    \dim(R(AB))+\dim(R(BC))-\dim(R(B))\leq \dim(R(ABC))
\]
等号成立的条件为 $N(A)\cap R(BC)=N(A)\cap R(B)$.
\end{proof}

\begin{definition}[线性变换]
线性变换是从一个线性空间映射到它本身的线性映射,即 $T:W\to W$.
\end{definition}

\begin{definition}[线性变换的矩阵表示]
由于线性变换只涉及一个空间,所以当我们讨论线性映射的矩阵表示时,只需选择一个基 $X$,即:
\[TX=XA\]
当然,我们也可以选择两个不同的基 $X,Y$,这时相当于把线性变换依旧视作线性映射。本课程以后提到线性变换时都只选择一个基。
\end{definition}

与线性映射在不同基下有不同的矩阵表示类似,线性变换在不同基下也有着不同的矩阵表示,这引出了相似矩阵的定义。

\begin{theorem}[线性变换在不同基下的矩阵表示]
\label{thm:sim-trans}
设线性变换 $T$ 在基 $X$ 下的矩阵表示为 $A$,在基 $X'$ 下的矩阵表示为 $A'$,且两个基之间的关系为:$X'=XC$,那么 $A'=C^{-1}AC$.
\end{theorem}
\begin{proof}
\[
TX'=T(XC)=(TX)C=(XA)C=X(AC)=X'C^{-1}AC=X'A'\implies A'=C^{-1}AC
\]
\end{proof}

\begin{definition}[相似矩阵]
设 $A,A'$ 为 $n$ 阶矩阵,若存在可逆矩阵 $C$ 使得:
\[A'=C^{-1}AC\]
则称 $A$ 与 $A'$ 是相似的。
\end{definition}

\begin{property}
矩阵相似关系为等价关系,即满足:
\begin{itemize}
    \item 自反性:$A$ 和 $A$ 相似;
    \item 对称性:若 $A$ 和 $B$ 相似,则 $B$ 和 $A$ 相似;
    \item 传递性:若 $A$ 和 $B$ 相似,$B$ 和 $C$ 相似,则 $A$ 和 $C$ 相似。
\end{itemize}
\end{property}

\begin{remark}
可以看见,\textbf{相似矩阵本质上是同一个线性变换在不同基下的表示}。因此,\textbf{相似等价意义下矩阵具有的性质本质上是对应线性变换的性质}。例如,我们即将看见相似矩阵的行列式相同,本质这是因为行列式对应着线性变换对原空间的单位超立方体变换后的体积。
\end{remark}

\begin{definition}[线性变换的多项式]
记 $T^2$ 表示复合变换 $T\circ T$;类似地,记 $T^k=T^{k-1}\circ T$. 进一步地,记线性变换的多项式为 $f(T)=a_0T^m+a_1T^{m-1}+\cdots+a_{m-1}T+a_mI$.
\end{definition}

\begin{theorem}[线性变换的多项式的矩阵表示]
若 $T$ 的矩阵表示为 $A$,那么 $T^k$ 的矩阵表示为 $A^k$;进一步地,多项式 $f(T)$ 的矩阵表示为 $f(A)$.
\end{theorem}

上面定义了线性变换的多项式,自然地,我们思考能否定义线性变换的一般函数,例如 $\exp(T)$ 或 $\sin(T)$?不过,这个问题需要 Hamilton-Cayley 定理 \ref{thm:Hamilton-Cayley} 和第三章的相关知识,我们将在 \ref{sec:3-matrix-function} 节中给出答案。


\subsection{线性变换的表示}

由于线性变换在不同基下的矩阵表示不相同,那么一个很自然的问题就是,怎样选择特殊的基使得给定线性变换在这个基下的表示矩阵尽可能简单?换个说法,由于同一线性变换的不同矩阵表示之间是相似关系,这个问题等价于,如何找到一个形式尽可能简单的矩阵使之相似于给定矩阵——这就是本节的终极目标 Jordan 标准形。不过在此之前,我们首先要介绍有关特征值与特征向量的基础知识。

\begin{definition}[矩阵的特征值与特征向量]
设 $A$ 为 $n$ 阶矩阵,若存在非零向量 $x$ 和数 $\lambda$ 满足 $Ax=\lambda x$,则称 $\lambda$ 为矩阵 $A$ 的特征值,$x$ 为属于 $\lambda$ 的特征向量。
\end{definition}

\begin{theorem}
$n$ 阶矩阵 $A$ 奇异当且仅当 $0$ 是 $A$ 的特征值。
\end{theorem}
\begin{proof}
$n$ 阶矩阵 $A$ 奇异 $\iff$ 存在 $x\neq0$ 使得 $Ax=0$ $\iff$ 存在 $x\neq0$ 使得 $Ax=0\cdot x$ $\iff$ $0$ 是 $A$ 的特征值。
\end{proof}

根据特征值的定义,若 $\lambda,\,x\neq0$ 是矩阵 $A$ 的特征值和对应特征向量,则有 $Ax=\lambda x$,即 $(\lambda I-A)x=0$. 由于 $x\neq 0$,因此该方程有解当且仅当 $\lambda I-A$ 奇异,即 $\det(\lambda I-A)=0$. 我们据此定义特征多项式与特征方程。

\begin{definition}[特征多项式,特征方程]
设 $A$ 为一个 $n$ 阶矩阵,称多项式 $\varphi(\lambda)=\det(\lambda I-A)$ 为矩阵 $A$ 的特征多项式,方程 $\det(\lambda I-A)=0$ 为矩阵 $A$ 的特征方程。
\end{definition}

设 $\lambda_0$ 是特征多项式 $\varphi(\lambda)$ 的零点,那么相应方程 $(\lambda_0I-A)x=0$ 就有非零解,不妨设为 $x_0\neq0$,于是有 $Ax_0=\lambda_0x_0$,即 $\lambda_0$ 是矩阵 $A$ 的特征值,$x_0$ 是相应的特征向量。因此我们有结论:特征多项式 $\varphi(\lambda)$ 的零点就是矩阵 $A$ 的特征值。反之,根据定义,矩阵 $A$ 的特征值显然是特征多项式的零点。又根据\textbf{代数基本定理},$n$ 阶多项式在复数域中必有 $n$ 个零点(可能重合),因此为了方便起见,我们给特征值的重数做出以下定义。

\begin{definition}[代数重数]
设 $A$ 为一个 $n$ 阶矩阵,$\lambda_0$ 为其一个特征值,称 $\lambda_0$ 作为特征多项式 $\varphi(\lambda)$ 的零点的重数为它的代数重数。
\end{definition}

有了代数重数的定义,我们就可以直接断言:\textbf{任意 $n$ 阶矩阵在复数域中恰有 $n$ 个特征值(可能重复)}。设这 $n$ 个特征值为 $\lambda_1,\lambda_2,\ldots,\lambda_n$(重复特征值重复写出),那么特征多项式也可以写作:
\[
    \varphi(\lambda)=(\lambda-\lambda_1)(\lambda-\lambda_2)\cdots(\lambda-\lambda_n)
\]
基于此,通过探究特征多项式根与系数的关系,我们可以得到下述常用结论。

\begin{theorem}[迹与特征值,行列式与特征值]
\ 

1) 矩阵 $A$ 的迹等于 $A$ 的所有特征值之和;

2) 矩阵 $A$ 的行列式等于 $A$ 的所有特征值之积。
\end{theorem}
\begin{proof}
设矩阵 $A$ 的特征值为 $\lambda_1,\lambda_2,\ldots,\lambda_n$,则特征多项式可写作:
\begin{align*}
    \varphi(\lambda)&=(\lambda-\lambda_1)(\lambda-\lambda_2)\cdots(\lambda-\lambda_n)\\
    &=\lambda^n-(\lambda_1+\lambda_2+\cdots+\lambda_n)\lambda^{n-1}+\cdots+(-1)^n\lambda_1\lambda_2\cdots\lambda_n
\end{align*}
另一方面,根据定义:
\[
    \varphi(\lambda)=\det(\lambda I-A)=
    \begin{vmatrix}
    \lambda-a_{11}&-a_{12}&\cdots&-a_{1n}\\
    -a_{21}&\lambda-a_{22}&\cdots&-a_{2n}\\
    \vdots&\vdots&\ddots&\vdots\\
    -a_{n1}&-a_{n2}&\cdots&\lambda-a_{nn}
    \end{vmatrix}
\]
将行列式按第一行展开,可以发现含有 $\lambda^n$ 和 $\lambda^{n-1}$ 的只有一项:
\[
    (\lambda-a_{11})(\lambda-a_{22})\cdots(\lambda-a_{nn})
\]
这是因为其他项最高只到 $\lambda^{n-2}$. 因此,对比二式中 $\lambda^{n-1}$ 的系数可知:
\[
    \lambda_1+\lambda_2+\cdots+\lambda_n=a_{11}+a_{22}+\cdots+a_{nn}=\text{tr}(A)
\]
即矩阵 $A$ 的迹等于 $A$ 的所有特征值之和。另外,在特征多项式中代入 $\lambda=0$,得:
\[
    \det(-A)=(-1)^n\det(A)=(-1)^n\lambda_1\lambda_2\cdots\lambda_n
\]
因此有:
\[
    \det(A)=\lambda_1\lambda_2\cdots\lambda_n
\]
即矩阵 $A$ 的行列式等于 $A$ 的所有特征值之积。
\end{proof}

\begin{theorem}
\label{thm:sim-eigen}
相似矩阵有相同的特征多项式。
\end{theorem}
\begin{proof}
设 $A$ 与 $B$ 相似,则存在可逆矩阵 $P$ 使得 $P^{-1}AP=B$,于是:
\begin{align*}
    \det(\lambda I-B)&=\det(P^{-1}(\lambda I)P-P^{-1}AP)=\det(P^{-1}(\lambda I-A)P)\\
    &=\det(P^{-1})\det(\lambda I-A)\det(P)=\det(\lambda I-A)
\end{align*}
即 $A$ 与 $B$ 的特征多项式相同。
\end{proof}

\begin{corollary}
相似矩阵的特征值相同、迹相同、行列式相同。
\end{corollary}

\begin{theorem}[$AB$ 和 $BA$ 有相同的非零特征值]
\label{thm:eigen-ab-ba}
设 $A\in\mathbb F^{m\times n},B\in\mathbb F^{n\times m}$,记 $AB$ 的特征多项式为 $\varphi_{AB}(\lambda)$,$BA$ 的特征多项式为 $\varphi_{BA}(\lambda)$,则:
\[
    \lambda^n\varphi_{AB}(\lambda)=\lambda^m\varphi_{BA}(\lambda)
\]
即 $AB$ 和 $BA$ 有相同的非零特征值。
\end{theorem}
\begin{proof}
由于:
\[
    \begin{bmatrix}I_m&0\\-B&I_n\end{bmatrix}\begin{bmatrix}I_m&A\\0&\lambda I_n\end{bmatrix}\begin{bmatrix}\lambda I_m-AB&0\\B&I_n\end{bmatrix}=\begin{bmatrix}I_m&A\\0&\lambda I_n-BA\end{bmatrix}
\]
等式两边取行列式即得证。
\end{proof}
\begin{proof}[更直接的证明方法]
设 $\lambda,x$ 为 $AB$ 的特征值和特征向量,即 $ABx=\lambda x$,那么左乘 $B$ 得到:\[(BA)(Bx)=\lambda (Bx)\]
也就是说 $\lambda$ 和 $Bx$ 是 $BA$ 的特征值和特征向量。
\end{proof}

\begin{corollary}
设 $A\in\mathbb F^{m\times n},B\in\mathbb F^{n\times m}$,则 $\text{tr}(AB)=\text{tr}(BA)$,$\det(AB)=\det(BA)$.
\end{corollary}

\begin{corollary}
\label{cor:trace}
设 $A\in\mathbb F^{m\times n},B\in\mathbb F^{n\times p},C\in\mathbb F^{p\times m}$,则 $\text{tr}(ABC)=\text{tr}(BCA)=\text{tr}(CAB)$. 以此类推,对于更多矩阵相乘的情形有类似的轮换性质。
\end{corollary}

\begin{theorem}[特征向量的线性无关性]
\label{thm:eigenind}
设 $A$ 为矩阵,$\lambda_1,\ldots,\lambda_s$ 为 $A$ 的\textbf{互不相同}的特征值,$x_1,\ldots,x_s$ 是分别属于这些特征值的特征向量,那么 $x_1,\ldots,x_s$ 线性无关。
\end{theorem}
\begin{proof}
考虑数学归纳法。首先,单个特征向量 $x_1$ 线性无关;其次,假设 $x_1,\ldots,x_{s-1}$ 线性无关,设 $\sum_{i=1}^s k_ix_i=0$,用 $A$ 左乘上式得:
\[
    \sum_{i=1}^sk_iAx_i=\sum_{i=1}^sk_i\lambda_ix_i=0
\]
根据上面两个式子消去 $x_s$,得:
\[
    \sum_{i=1}^{s-1}k_i(\lambda_i-\lambda_s)x_i=0
\]
根据归纳假设,有 $k_i(\lambda_i-\lambda_s)=0$;又特征值互不相同,故 $k_i=0\ (i=1,\ldots,s-1)$. 进而 $k_s=0$.
\end{proof}

\begin{theorem}[特征向量的线性无关性-续]
设 $A$ 为矩阵,$\lambda_1,\ldots,\lambda_k$ 为 $A$ 的\textbf{互不相同}的特征值,$x_{i1},\ldots,x_{ir_i}$ 是属于 $\lambda_i$ 的线性无关特征向量,那么 $x_{11},\ldots,x_{1r_1},\ldots,x_{k1},\ldots,x_{kr_k}$ 线性无关。
\end{theorem}
\begin{proof}
设 $\sum_{i=1}^k\sum_{j=1}^{r_i}k_{ij}x_{ij}=0$,记 $x_i=\sum_{j=1}^{r_i}k_{ij}x_{ij}$,左乘 $A$ 得:
\[
    Ax_i=\sum_{j=1}^{r_i}k_{ij}Ax_{ij}=\sum_{j=1}^{r_i}k_{ij}\lambda_ix_{ij}=\lambda_ix_i
\]
假设 $x_i\neq0$,则 $x_i$ 是 $A$ 属于 $\lambda_i$ 的特征向量,又由于 $\sum_{i=1}^kx_i=0$,根据定理 \ref{thm:eigenind} 知 $x_i=0$,矛盾,因此只能有 $x_i=0$. 由线性无关性可知,所有的 $k_{ij}=0$.
\end{proof}

\begin{theorem}
\label{thm:anysimilar}
任意 $n$ 阶矩阵都与一个上三角矩阵相似。
\end{theorem}
\begin{proof}
考虑数学归纳法。设 $A$ 是一个 $n$ 阶矩阵,$x_1$ 为 $A$ 的特征值 $\lambda$ 对应的特征向量。将 $x_1$ 扩充为 $\mathbb C^n$ 的一个基 $(x_1,\ldots,x_n)$,那么 $Ax_i$ 都可以被这个基线性表示:
\[
\begin{cases}
Ax_1=b_{11}x_1+b_{21}x_2+\cdots+b_{n1}x_n=\lambda x_1\\
Ax_2=b_{12}x_1+b_{22}x_2+\cdots+b_{n2}x_n\\
\quad\vdots\\
Ax_n=b_{1n}x_1+b_{2n}x_2+\cdots+b_{nn}x_n
\end{cases}
\]
写作矩阵形式:
\[
AX=XB=X\begin{bmatrix}\lambda&b_{12}&\cdots&b_{1n}\\0&b_{22}&\cdots&b_{2n}\\\vdots&\vdots&\ddots&\vdots\\0&b_{n2}&\cdots&b_{nn}\end{bmatrix}=X\begin{bmatrix}\lambda&\alpha^T\\0&B_1\end{bmatrix}
\]
根据归纳假设,设 $B_1=QUQ^{-1}$ 且 $U$ 是上三角矩阵,那么:
\[
AX=X\begin{bmatrix}\lambda&\alpha^T\\0&QUQ^{-1}\end{bmatrix}=X\begin{bmatrix}1&0\\0&Q\end{bmatrix}\begin{bmatrix}\lambda&\alpha^T\\0&U\end{bmatrix}\begin{bmatrix}1&0\\0&Q^{-1}\end{bmatrix}
\]
所以 $A$ 相似于上三角矩阵 $\begin{bmatrix}\lambda&\alpha^T\\0&U\end{bmatrix}$.
\end{proof}

\begin{definition}[零化多项式]
设 $A$ 是一个 $n$ 阶矩阵,若多项式 $f(x)$ 使得 $f(A)=0$,则称 $f(x)$ 为矩阵 $A$ 的一个零化多项式。
\end{definition}

\begin{theorem}[Hamilton-Cayley 定理]
\label{thm:Hamilton-Cayley}
设 $A$ 是一个 $n$ 阶矩阵,则 $A$ 的特征多项式是其零化多项式。形式化地说,设:
\[
\varphi(\lambda)=\det(\lambda I-A)=\lambda^n+a_1\lambda^{n-1}+\cdots+a_{n-1}\lambda+a_n
\]
则:
\[
\varphi(A)=A^n+a_1A^{n-1}+\cdots+a_{n-1}A+a_nI_n=0
\]
\end{theorem}

\begin{proof}
考虑数学归纳法。首先,当 $n=1$ 时,定理显然成立;其次,设定理对 $n-1$ 阶矩阵成立,下面证明对 $n$ 阶矩阵依然成立。设 $A$ 的 $n$ 个特征值为 $\lambda_1,\ldots,\lambda_n$,根据定理 \ref{thm:anysimilar} 的证明过程可知,$A$ 相似于:
\[R=\begin{bmatrix}\lambda_1&\alpha^T\\0&U\end{bmatrix}\]
即存在可逆矩阵 $P$ 使得 $A=P^{-1}RP$,其中 $U$ 为上三角矩阵。由于相似矩阵有相同的特征值,所以 $\lambda_1,\ldots,\lambda_n$ 也是 $R$ 的特征值。容易知道 $\lambda_2,\ldots,\lambda_n$ 是 $U$ 的特征值。又因为:
\[
\varphi(A)=\varphi(P^{-1}RP)=\sum_{i=0}^n a_{n-i}(P^{-1}RP)^i=\sum_{i=0}^n a_{n-i}P^{-1}R^iP=P^{-1}\varphi(R)P
\]
所以要证明 $\varphi(A)=0$,只需要证明 $\varphi(R)=0$.
\begin{align*}
\varphi(R)&=(R-\lambda_1I_n)\cdots(R-\lambda_n I_n)\\
&=\begin{bmatrix}\lambda_1-\lambda_1&\alpha^T\\0&U-\lambda_1I_{n-1}\end{bmatrix}\cdots\begin{bmatrix}\lambda_n-\lambda_1&\alpha^T\\0&U-\lambda_nI_{n-1}\end{bmatrix}\\
&=\begin{bmatrix}0&\alpha^T\\0&U-\lambda_1I_{n-1}\end{bmatrix}\begin{bmatrix}\prod_{j=2}^n(\lambda_j-\lambda_1)&\beta^T\\0&\prod_{j=2}^n(U-\lambda_jI_{n-1})\end{bmatrix}\\
&=\begin{bmatrix}0&0\\0&(U-\lambda_1 I_{n-1})\prod_{j=2}^n(U-\lambda_jI_{n-1})\end{bmatrix}
\end{align*}
根据归纳假设,$\prod_{j=2}^n(U-\lambda_jI_{n-1})=0$,因此上式为 $0$.
\end{proof}

\begin{corollary}
对于 $n$ 阶矩阵 $A$,$\{A^n,A^{n-1},\ldots,A,I_n\}$ 线性相关。
\end{corollary}

\begin{corollary}
\label{cor:invpoly}
任何一个 $n$ 阶可逆矩阵 $A$ 的逆可表示为 $A$ 的次数不超过 $n-1$ 的多项式,即:
\[A^{-1}=g(A)=a_1A^{n-1}+a_2A^{n-2}+\cdots+a_{n-1}A+a_nI_n\]
\end{corollary}

\begin{definition}[最小多项式]
零化多项式中,称次数最低的首项系数为 1 的零化多项式 $m(\lambda)$ 为最小多项式。(注意这里 $\lambda$ 只是一个变量符号,不是特征值的意思)
\end{definition}

\begin{theorem}
\label{thm:minpoly}
最小多项式可以整除任意其他首项系数为 1 的零化多项式 $\psi(\lambda)$,且是唯一的。
\end{theorem}
\begin{proof}
作多项式除法:$\psi(\lambda)=m(\lambda)p(\lambda)+r(\lambda)$,其中 $r(\lambda)$ 的次数小于 $m(\lambda)$ 的次数。由于 $\psi(A)=m(A)=0$,故 $r(A)=0$,但由于 $m(\lambda)$ 是次数最小的零化多项式,所以只能是 $r(\lambda)=0$.  因此 $m(\lambda)\mid\psi(\lambda)$.
\end{proof}

\begin{theorem}
矩阵 $A$ 的最小多项式 $m(\lambda)$ 和特征多项式 $\varphi(\lambda)$ 零点相同(重数可以不同)。换句话说,$m(\lambda)$ 的零点就是特征值,只是与 $\varphi(\lambda)$ 的次数不同。
\end{theorem}
\begin{proof}
根据定理 \ref{thm:minpoly},$\varphi(\lambda)=m(\lambda)p(\lambda)$,所以 $m(\lambda)=0\implies \varphi(\lambda)=0$. 因此现在只需证明 $\varphi(\lambda)=0\implies m(\lambda)=0$.
设 $\varphi(\lambda_0)=0$,$A x_0=\lambda_0 x_0$,则 $m(A) x_0=m(\lambda_0) x_0=0$,由于 $x_0\neq 0$,所以 $m(\lambda_0)=0$.
\end{proof}

\begin{com}
显然,如果矩阵 $A$ 的最小多项式次数为 $m$,那么 $\{A^{m-1},\ldots,A,I_n\}$ 线性无关,但再加入一个 $A^m$ 就线性相关了。
\end{com}

本节至此我们都是站在矩阵的角度分析其特征值和特征向量,而前文我们提到过,借助线性变换来研究矩阵是非常重要的手段。基于这种思想,下面我们引入线性变换的特征值和特征向量,进而引入特征子空间和不变子空间的概念,最终得到著名的 Jordan 标准形。

\begin{definition}[线性变换的特征值与特征向量]
设 $T$ 是一个线性变换,若存在非零向量 $x$ 和数 $\lambda$ 满足 $Tx=\lambda x$,则称 $\lambda$ 为 $T$ 的特征值,$x$ 为属于 $\lambda$ 的特征向量。
\end{definition}

\begin{theorem}[线性变换的特征值和特征向量与矩阵的特征值和特征向量]
设 $X$ 为线性空间中的一个基,线性变换 $T$ 在该基下的矩阵表示为 $A$. 若 $\lambda$ 为 $T$ 的一个特征值,对应特征向量 $x$ 在基 $X$ 下的坐标表示为 $\xi$,则 $\lambda$ 是 $A$ 的特征值,$\xi$ 是对应的特征向量。
\end{theorem}
\begin{proof}
由题意有 $x=X\xi,\,TX=XA,\,Tx=\lambda x$,因此:
\[
    \begin{cases}
    Tx=T(X\xi)=(TX)\xi=XA\xi\\
    Tx=\lambda x=\lambda X\xi=X(\lambda\xi)
    \end{cases}\implies A\xi=\lambda\xi
\]
即 $\lambda$ 是 $A$ 的特征值,$\xi$ 是对应的特征向量。
\end{proof}

\begin{remark}
线性变换的特征值和其矩阵表示的特征值是一样的,而特征向量的关系就是在选取的那个基下的坐标关系。
\end{remark}

基于上述结论,我们发现矩阵的特征值在本质上其实是背后的线性变换的性质,于是前文定理 \ref{thm:sim-eigen} 所述的“相似矩阵有相同的特征值”有了更直观简单的解释:相似矩阵是同一个线性变换的不同矩阵表示,其特征值就是线性变换的特征值,因此显然都是相同的。

\begin{definition}[特征子空间]
设 $T$ 为线性变换,$\lambda_0$ 为 $T$ 的一个特征值,则称 $\lambda_0I-T$ 的核空间($I$ 表示恒等变换)为 $T$ 属于 $\lambda_0$ 的特征子空间:
\[V_{\lambda_0}=N((\lambda_0I-T))=\{x\mid(\lambda_0 I-T)x=0\}\]
\end{definition}

\begin{com}
根据定义,若 $x\in V_{\lambda_0}$,则 $x$ 是属于 $\lambda_0$ 的特征向量。
\end{com}

\begin{definition}[几何重数]
设 $T$ 为线性变换,$\lambda_0$ 为 $T$ 的一个特征值,称 $\dim V_{\lambda_0}$ 为它的几何重数。
\end{definition}

\begin{theorem}[代数重数 $\geq$ 几何重数]
设 $T$ 为线性空间 $V$ 上的线性变换,$\lambda_0$ 为 $T$ 的一个特征值,则其代数重数大于等于几何重数。
\end{theorem}
\begin{proof}
设几何重数 $\dim V_{\lambda_0}=q$,取 $V_{\lambda_0}$ 的基 $x_1,\ldots,x_q$,将其扩充为 $V$ 的基 $x_1,\ldots,x_q,\ldots,x_n$. 由于对 $i=1,2,\ldots,q$,有 $Tx_i=\lambda_0x_i$,因此 $T$ 在基 $x_1,\ldots,x_q,\ldots,x_n$ 下的矩阵表示为:
\[
    A=\left[\begin{array}{ccc:ccc}
    \lambda_0 & \cdots & 0         & &        & \\
    \vdots    & \ddots & \vdots    & & A_{12} & \\
    0         & \cdots & \lambda_0 & &        & \\ \hdashline
              &        &           & &        & \\
              & O      &           & & A_{22} & \\
              &        &           & &        & \\
    \end{array}\right]
\]
于是特征多项式为:
\[
    \varphi(\lambda)=\det(\lambda I-A)=(\lambda-\lambda_0)^q\det(\lambda I_{n-q}-A_{22})
\]
故代数重数至少为 $q$.
\end{proof}

\begin{definition}[不变子空间]
若线性空间 $V$ 的线性子空间 $V_1$ 对线性变换 $T$ 保持不变,即:$\forall x\in V_1$,有 $Tx\in V_1$,则称 $V_1$ 是 $T$ 的不变子空间。这时 $T$ 可以看作 $V_1$ 上的线性变换,称为 $T$ 在 $V_1$ 上的限制 $T\vert V_1$.  但值得注意的是,$T$ 和 $T\vert V_1$ 是不同的线性变换(它们的输入维度都不同)。
\end{definition}
\begin{figure}[H]
    \centering
    \includegraphics[width=0.2\linewidth]{figs/TV1.png}
\end{figure}

\begin{property}
不变子空间的和与交也是不变子空间。
\end{property}
\begin{proof}
设 $V_1,V_2$ 都是 $T$ 的不变子空间,则对于 $z\in V_1+V_2$,存在 $x\in V_1,\,y\in V_2$ 使得 $z=x+y$,并且 $Tx\in V_1,\,Ty\in V_2$,于是 $Tz=T(x+y)=Tx+Ty\in V_1+V_2$. 另外,对于 $z\in V_1\cap V_2$,由于 $Tz\in V_1,\,Tz\in V_2$,故 $Tz\in V_1\cap V_2$.
\end{proof}

\begin{property}
线性变换 $T$ 的值域 $R(T)$ 和核 $N(T)$ 都是 $T$ 的不变子空间。
\end{property}
\begin{proof}
对 $\forall x\in R(T)$,$Tx\in R(T)$,故 $R(T)$ 是 $T$ 的不变子空间;对于 $\forall x\in N(T)$,$Tx=0\in N(T)$,故 $N(T)$ 是 $T$ 的不变子空间。
\end{proof}

\begin{property}
设 $f(t)$ 为一多项式,则 $T$ 的不变子空间也是 $f(T)$ 的不变子空间。
\end{property}
\begin{proof}
设 $V_1$ 是 $T$ 的不变子空间,即 $\forall x\in V_1$,有 $Tx\in V_1$.  那么 $T^2x=T(Tx)\in V_1$,以此类推有 $f(T)(x)\in V_1$,即 $V_1$ 也是 $f(T)$ 的不变子空间。
\end{proof}

\begin{corollary}
若 $T$ 为可逆变换,则 $T$ 的不变子空间也是 $T^{-1}$ 的不变子空间。
\end{corollary}
\begin{proof}
根据推论 \ref{cor:invpoly},$T^{-1}$ 可写作 $T$ 的多项式,故可得结论。
\end{proof}

\begin{corollary}
特征子空间为不变子空间。
\end{corollary}
\begin{proof}
注意到 $\lambda I-T$ 是 $T$ 的多项式,故特征子空间 $V_\lambda=N(\lambda I-T)$ 为 $T$ 的不变子空间。
\end{proof}

\begin{theorem}
设 $T$ 为一个线性变换,$x$ 为 $T$ 的特征向量,则 $L(x)=\{z\mid z=kx,\,k\in \mathbb C\}$ 为 $T$ 的一维不变子空间。
\end{theorem}
\begin{proof}
对于 $z\in L(x)$,由于 $z$ 是 $T$ 的特征向量,因此存在 $\lambda\in\mathbb C$ 使得 $Tz=\lambda z\in L(x)$,故 $L(x)$ 是 $T$ 的不变子空间。
\end{proof}

基于不变子空间的概念,下面引入分块对角化与对角化。

\begin{theorem}[分块对角化]
\label{thm:blockdiag}
设 $T$ 是线性空间 $V^n$ 上的线性变换,假若 $V^n$ 可以分解为 $s$ 个 $T$ 的不变子空间的直和:
\[
    V^n=V_1\oplus\cdots\oplus V_s
\]
则 $T$ 在 $V^n$ 的某个基下的矩阵表示为分块对角矩阵:
\[
    A=\begin{bmatrix}A_1&&\\&\ddots&\\&&A_s\end{bmatrix}
\]
反之,若 $T$ 在基 $X=(X_1,\ldots,X_s)$ 下的矩阵表示为分块对角矩阵,则 $V^n$ 可分解为 $s$ 个不变子空间的直和。
\begin{figure}[H]
    \centering
    \includegraphics[width=0.8\linewidth]{figs/subspaces.png}
\end{figure}
\end{theorem}
\begin{proof}
设 $V^n$ 可以分解为 $s$ 个不变子空间的直和,在每个不变子空间 $V_i$ 中选取一个基 $X_i=(x_{i1},\ldots,x_{in_i}),\,(i=1,\ldots,s)$,将它们合并构成 $V^n$ 的基 $X=(X_1,\ldots,X_s)$,则显然 $T$ 在这个基下的矩阵表示为一个分块对角矩阵。反之,若 $T$ 在基 $X=(X_1,\ldots,X_s)$ 下的矩阵表示为分块对角矩阵,那么根据定义容易知道 $X_i$ 张成的子空间 $V_i$ 是 $T$ 的不变子空间,且 $V^n$ 是它们的直和。
\end{proof}

\begin{definition}[可对角化]
若线性空间 $V^n$ 上的线性变换 $T$ 在 $V^n$ 的某个基下的表示矩阵为对角阵,则称 $T$ 是可对角化的。
\end{definition}

\begin{theorem}[可对角化的充要条件]
设 $T$ 为 $V^n$ 上的线性变换,则:
\begin{align*}
    T \text{可对角化}&\iff \text{存在一组特征向量构成的基}\\
    &\iff \text{有}\ n\ \text{个线性无关的特征向量}\\
    &\iff \text{各个特征值的代数重数和几何重数相等}
\end{align*}
\end{theorem}
\begin{proof}
根据定义易知,每个特征向量张成的一维子空间都是一个一维不变子空间。因此,当 $T$ 有 $n$ 个线性无关的特征向量时(这些特征向量构成了一组基),$V^n$ 就可以写作这些特征向量分别张成的一维不变子空间的直和。于是根据定理 \ref{thm:blockdiag},$T$ 在这些特征向量构成的基下的矩阵表示为对角阵。
\end{proof}

\begin{corollary}[可对角化的充分条件]
若 $T$ 有 $n$ 个不同特征值,则 $T$ 可对角化。
\end{corollary}
\begin{proof}
根据定理 \ref{thm:eigenind} 可知 $T$ 有 $n$ 个线性无关的特征向量,因此 $T$ 可对角化。
\end{proof}

至此我们初步回答了本节开头提出的问题。如果线性变换存在一组特征向量构成的基,那么在这个基下它的表示矩阵就是我们所期望的最简单的对角阵形式。然而,并不是所有矩阵都有 $n$ 个线性无关的特征向量,所以现在我们必须回到分块对角化的形式上继续研究。注意分块对角化的前提假设是“空间可以分解为若干不变子空间的直和”,那这个假设是否总是成立呢?下面的定理告诉我们,这个假设不仅成立,而且这些不变子空间与特征值息息相关。

\begin{theorem}[基于不变特征子空间的直和分解]
\label{thm:directsum}
设 $T$ 是线性空间 $V^n$ 上的线性变换,任取 $V^n$ 的一个基,$T$ 在该基下的矩阵为 $A$,$T$ 的特征多项式为:
\[
\varphi(\lambda)=\det(\lambda I-A)=(\lambda-\lambda_1)^{m_1}(\lambda-\lambda_2)^{m_2}\cdots(\lambda-\lambda_s)^{m_s}
\]
其中 $m_1+m_2+\cdots+m_s=n$,\textbf{注意 $\lambda_1,\lambda_2,\ldots,\lambda_s$ 可以重复},则 $V^n$ 可分解为不变子空间的直和:
\[
V^n=N_1\oplus N_2\oplus\cdots\oplus N_s
\]
其中 $N_i=\{x\mid (\lambda_i I-T)^{m_i}x=0\}$ 是线性变换 $(\lambda_iI-T)^{m_i}$ 的核空间。
\end{theorem}

基于该定理,若在每个 $N((\lambda_i I-T)^{m_i})$ 中取一个基,则 $T$ 在这些基下的矩阵表示是一个分块对角矩阵。这就引出了 Jordan 标准形的概念。

\begin{definition}[Jordan 块]
形如下式的矩阵称作 Jordan 块。
\[
    J(\lambda)=\begin{bmatrix}
    \lambda&1&&&\\
    &\lambda&1&&\\
    &&\ddots&\ddots&\\
    &&&\lambda&1\\
    &&&&\lambda
    \end{bmatrix}
\]
\end{definition}

\begin{definition}[Jordan 标准形]
由若干 Jordan 块构成的形如下式的分块对角矩阵称作 Jordan 标准形。
\[
    J=\begin{bmatrix}
    J_1(\lambda_1)&&&\\&J_2(\lambda_2)&&\\&&\ddots&\\&&&J_s(\lambda_s)
    \end{bmatrix}
\]
\end{definition}

\begin{theorem}
\label{thm:jordan-exists}
存在一种 $A$ 的特征多项式的分解:
\[
    \varphi(\lambda)=\det(\lambda I-A)=(\lambda-\lambda_1)^{m_1}(\lambda-\lambda_2)^{m_2}\cdots(\lambda-\lambda_s)^{m_s}
\]
其中 $m_1+m_2+\cdots+m_s=n$,\textbf{注意 $\lambda_1,\lambda_2,\ldots,\lambda_s$ 可以重复},使得 $A$ 相似于一个 Jordan 标准形:
\[P^{-1}AP=J\]
且除了 Jordan 块的排列顺序以外 Jordan 标准形唯一。
\end{theorem}

Jordan 标准形有什么优点呢?我们已经看到,任意 $n$ 阶矩阵都能相似于一个上三角矩阵,但是上三角矩阵太多太复杂了,不便于研究;另一方面,虽然对角矩阵足够简单,但不是所有矩阵都能相似于一个对角矩阵(需要有 $n$ 个线性无关的特征向量);而 Jordan 标准形形式上比上三角矩阵简单,同时所有矩阵都能相似于一个 Jordan 标准形,因此兼具了上三角矩阵与对角矩阵的优点。

不过为了计算 Jordan 标准形,我们还需要解决一个问题——特征多项式的分解不唯一,究竟怎么分解才对呢(注意定理 \ref{thm:jordan-exists} 只说明了存在,没有给出构造)?比如 $(\lambda-1)^4$ 既可以分解成 $(\lambda-1)(\lambda-1)^3$,也可以分解成 $(\lambda -1)^2(\lambda -1)^2$. 下面的基于多项式矩阵($\lambda$ 阵)的初等变换法给出了一种计算方法。

\vskip 6pt \noindent\textbf{Jordan 标准形的计算方法}:

\begin{enumerate}
    \item 写出 $A$ 的特征矩阵 $\lambda I-A$;
    \item 计算特征矩阵的\textbf{行列式因子}:$D_i(\lambda)$ 表示所有 $i$ 阶子式的最大公因式;
    \item 计算\textbf{不变因子}:$d_i(\lambda)=D_i(\lambda)/D_{i-1}(\lambda)$;其中 $D_0(\lambda)=1$;
    \item 计算\textbf{初等因子组}:将每个不变因子化为不可约因式,这些不可约因式称为初等因子,全体初等因子称为初等因子组;
    \item 写出 Jordan 标准形:一个初等因子对应一个 Jordan 块,初等因子次数就是 Jordan 块阶数。
\end{enumerate}

\begin{example}
设 $d_1(\lambda)=(\lambda-2)^2(\lambda-3),\,d_2(\lambda)=(\lambda-2)^2(\lambda-3)^5$,则初等因子组为 $\{(\lambda-2)^2,(\lambda-3),(\lambda-2)^2,(\lambda-3)^5\}$.  注意其中第一个 $(\lambda-2)^2$ 来自 $d_1(\lambda)$,第二个 $(\lambda-2)^2$ 来自 $d_2(\lambda)$.
\end{example}

\noindent\textbf{Jordan 标准形变换矩阵的计算方法}:上面求出了 Jordan 标准形 $J$,现在求解变换矩阵 $P$.
由于 $P^{-1}AP=J$,所以 $AP=PJ$.  鉴于 $J$ 是分块对角矩阵,所以只需要一块一块考虑即可:$AP_i=P_iJ_i(\lambda_i)$,其中 $P_i$ 是 $P$ 的对应列。显式地写出来:
\[
    A(p_1,p_2,\cdots,p_m)=(p_1,p_2,\cdots,p_m)
    \begin{bmatrix}
    \lambda_i&1&\cdots&0\\
    0&\lambda_i&\cdots&0\\
    \vdots&\vdots&\ddots&\vdots\\
    0&0&\cdots&\lambda_i
    \end{bmatrix}
\]
于是:
\[
    \begin{cases}
    Ap_1=\lambda_ip_1\\
    Ap_2=p_1+\lambda_ip_2\\
    Ap_3=p_2+\lambda_ip_3\\
    \quad\vdots\\
    Ap_m=p_{m-1}+\lambda_ip_m
    \end{cases}\implies
    \begin{cases}
    (\lambda_iI-A)p_1=0\\
    (\lambda_iI-A)p_2=-p_1\\
    (\lambda_iI-A)p_3=-p_2\\
    \quad\vdots\\
    (\lambda_iI-A)p_m=-p_{m-1}
    \end{cases}
\]
事实上这里 $p_1$ 是 $A$ 的特征向量,$p_2,\ldots,p_m$ 是 $A$ 的广义特征向量。也就是说,由于 $\lambda_i$ 的几何重数小于代数重数,所以找不到 $m$ 个特征向量,只能用广义特征向量填补。

\begin{theorem}[Jordan 标准形与最小多项式]
对于特征值 $\lambda_i$,其在最小多项式中的次数等于属于 $\lambda_i$ 的 Jordan 块的最高阶数。
\end{theorem}

\begin{theorem}[Jordan 标准形与几何重数]
对于特征值 $\lambda_i$,其几何重数等于属于 $\lambda_i$ 的 Jordan 块个数。
\end{theorem}

\begin{example}
考察矩阵:
\[
    A=\begin{bmatrix}
    1&1&&&&&&\\
    &1&1&&&&&\\
    &&1&&&&&\\
    &&&1&1&&&\\
    &&&&1&&&\\
    &&&&&-1&1&\\
    &&&&&&-1&\\
    &&&&&&&-1
    \end{bmatrix}
\]
可以看到 $\lambda=1$ 有一个 3 阶和一个 2 阶的 Jordan 块,所以最小多项式中 $(\lambda-1)$ 的次数为 3;同理,$(\lambda+1)$ 的次数为 2. 于是:
\[
    m(\lambda)=(\lambda-1)^3(\lambda+1)^2
\]
另外,$\lambda=1$ 和 $\lambda=-1$ 都有 2 个 Jordan 块,因此它们的几何重数都是 2.
\end{example}

\noindent\textbf{Jordan 标准形的多项式}:我们知道相似对角化的一个重要作用就是简化 $A^k$ 的计算:
\[A=P\Lambda P^{-1}\implies A^k=P\Lambda^kP^{-1}\]
而对于无法相似对角化的矩阵而言,Jordan 标准形也起到了类似的作用:
\[A=PJP^{-1}\implies A^k=PJ^kP^{-1}\]
因此我们现在需要关注 $J^k$ 的计算。由于 $J$ 是分块对角矩阵,所以我们只需要逐个考虑每一块即可。将 $J(\lambda)$ 写作:
\[
    J(\lambda)=\lambda I_{r\times r}+L
    ,\quad L=
    \begin{bmatrix}
    0&1&&&\\
    &0&1&&\\
    &&\ddots&\ddots&\\
    &&&0&1\\
    &&&&0
    \end{bmatrix}_{r\times r}
\]
其中 $L$ 是一个幂零矩阵,满足 $L^r=0$ 而 $L^{r-1}\neq 0$. 于是:
\begin{align*}
    J(\lambda)^k&=\sum_{i=0}^k\binom{k}{i}\lambda^{k-i}L^i=\sum_{i=0}^k\frac{k(k-1)\cdots(k-i+1)}{i!}\lambda^{k-i}L^i\\
    &=\sum_{i=0}^k\frac{1}{i!}(\lambda^k)^{(i)}L^i=\sum_{i=0}^{\min(k,r-1)}\frac{1}{i!}(\lambda^k)^{(i)}L^i
\end{align*}
更进一步,对于多项式 $f(x)=\sum_{k=0}^sa_kx^k$,有:
\begin{align*}
    f(J(\lambda))&=\sum_{k=0}^sa_kJ(\lambda)^k=\sum_{k=0}^sa_k\sum_{i=0}^{\min(k,r-1)}\frac{1}{i!}(\lambda^k)^{(i)}L^i\\
    &=\sum_{i=0}^{s}\frac{1}{i!}\left(\sum_{k=i}^sa_k\lambda^k\right)^{(i)}L^i=\sum_{i=0}^{s}\frac{1}{i!}\left(\sum_{k=0}^sa_k\lambda^k\right)^{(i)}L^i\\
    &=\sum_{i=0}^{s}\frac{1}{i!}\left(f(\lambda)\right)^{(i)}L^i\\
    &=\begin{bmatrix}
    f(\lambda)&f'(\lambda)&\frac{f''(\lambda)}{2!}&\frac{f'''(\lambda)}{3!}&\cdots&\frac{f^{(r-1)}(\lambda)}{(r-1)!}\\
    &f(\lambda)&f'(\lambda)&\frac{f''(\lambda)}{2!}&\cdots&\frac{f^{(r-2)}(\lambda)}{(r-2)!}\\
    &&f(\lambda)&f'(\lambda)&\cdots&\frac{f^{(r-3)}(\lambda)}{(r-3)!}\\
    &&&\ddots&\ddots&\vdots\\
    &&&&f(\lambda)&f'(\lambda)\\
    &&&&&f(\lambda)
    \end{bmatrix}_{r\times r}
\end{align*}


\subsection{欧式空间和酉空间}

\noindent 欧氏空间(内积空间)是定义了\textbf{内积}运算的\textbf{实数域} $\mathbb R$ 上线性空间。

\begin{definition}[内积,欧式空间]
设 $V$ 是实数域 $\mathbb R$ 上的线性空间,对 $V$ 中任意 $x$ 和 $y$, 按某种规则定义一个实数,用 $(x,y)$ 表示,且满足下列四个条件:
\begin{enumerate}
    \item 交换律:$(x,y)=(y,x)$
    \item 分配律:$(x,y+z)=(x,y)+(x,z)$
    \item 齐次性:$(kx,y)=k(x,y),\,\forall k\in \mathbb R$
    \item 非负性:$(x,x)\geq 0$,当且仅当 $x=0$ 时 $(x,x)=0$
\end{enumerate}
则称 $(x,y)$ 为 $x$ 与 $y$ 的内积,$V$ 为欧氏空间或实内积空间。
\end{definition}

\begin{remark}
任意线性空间上都可以定义内积,但是\textbf{不唯一}。一种较为简单的定义方式是根据坐标定义内积(见下文)。
\end{remark}

\begin{property}
$(x,ky)=k(x,y)$
\end{property}
\begin{property}
$(x,0)=(0,x)=0$
\end{property}
\begin{property}[线性性]
$\left(\sum_{i=1}^n\xi_ix_i,\sum_{j=1}^n\eta_jy_j\right)=\sum_{i=1}^n\sum_{j=1}^n\xi_i\eta_j(x_i,y_j)$
\end{property}

\begin{example}
考虑例 \ref{ex:linearspace} 中定义的线性空间 $(\mathbb R^n,\mathbb R,\oplus,\odot)$:
\begin{gather*}
    x\oplus y=((x_1^3+y_1^3)^{1/3},(x_2^3+y_2^3)^{1/3}\ldots,(x_n^3+y_n^3)^{1/3})^T\\
    k\odot x=k^{1/3}x
\end{gather*}
定义内积为:
\[
    (x,y)=(x_1\cdot y_1)^3+\cdots+(x_n+y_n)^3
\]
\end{example}

\begin{example}
\label{ex:innerproduct}
考虑例 \ref{ex:linearspace2} 中定义的线性空间 $(\mathbb R^+,\mathbb R,\oplus,\odot)$:
\begin{gather*}
    x\oplus y=x\cdot y\\
    k\odot x=x^k
\end{gather*}
定义内积为:
\[
    (x,y)=\log x\cdot \log y
\]
\end{example}

\begin{example}
对于例 \ref{ex:innerproduct} 中的一维空间,可以通过笛卡尔积将其扩充为多维空间 $V=\mathbb R^+\times \mathbb R^+\times\cdots\times \mathbb R^+$,定义加法和数乘为:
\begin{gather*}
    x\oplus y=(x_1\cdot y_1,x_2\cdot y_2,\ldots,x_n\cdot y_n)^T\\
    k\odot x=(x_1^k,x_2^k,\ldots,x_n^k)^T
\end{gather*}
定义内积为:
\[
    (x,y)=\log x_1\cdot\log y_1+\log x_2\cdot\log y_2+\cdots+\log x_n\cdot\log y_n
\]
\end{example}

\begin{example}[根据坐标定义内积]
设 $X$ 为 $V$ 上的一个基,向量 $x,y\in V$ 在该基下的坐标分别为 $\alpha=(\alpha_1,\ldots,\alpha_n)^T,\beta=(\beta_1,\ldots,\beta_n)^T$,则可以定义内积为:
\[
    (x,y)=\alpha_1\beta_1+\cdots+\alpha_n\beta_n=\alpha^T\beta
\]
容易验证这确实满足内积的 4 个条件。
注意这种定义方式与基的选取有关,可以推导不同基下这样定义的内积之间的关系。设 $X'=XC$,$x,y$ 在 $X'$ 下的坐标为 $\alpha',\beta'$,那么有:$\alpha'=C^{-1}\alpha,\,\beta'=C^{-1}\beta$,于是:
\[
    (x,y)'=(\alpha')^T\beta'=\alpha^T(C^{-1})^TC^{-1}\beta=\alpha^T A^{-1}\beta
\]
其中 $A=CC^T$ 为正定矩阵。
\end{example}

\begin{remark}
这门课上内积是一个抽象的概念,只有在上述坐标定义下可以写作 $\alpha^T\beta$ 的形式,否则只能写成 $(x,y)$ 的形式。
\end{remark}

\begin{definition}[长度/模/由内积诱导的范数]
\label{def:inner-product-norm}
称非负实数 $\Vert x\Vert=\sqrt{(x,x)}$ 为向量 $x$ 的长度或模,也称由内积诱导的范数。
\end{definition}

\begin{definition}[夹角]
定义非零向量 $x$ 和 $y$ 的夹角为:$\langle x,y\rangle=\arccos\dfrac{(x,y)}{\Vert x\Vert\Vert y\Vert}$
\end{definition}

\begin{definition}[Gram 矩阵]
设有向量组 $X=(x_1,\ldots,x_n)$,称矩阵:
\[
    \text{Gram}(x_1,\ldots,x_n)=[(x_i,x_j)]_{ij}
\]
为 $X$ 的 Gram 矩阵。
\end{definition}

\begin{definition}[基于 Gram 矩阵的线性无关判别定理]
$x_1,\ldots,x_n$ 线性无关的充要条件是它们组成的 Gram 矩阵非奇异。
\end{definition}
\begin{proof}
设 $a_1x_1+\cdots+a_nx_n=0$,与 $x_k$ 做内积得:
\[
    a_1(x_k,x_1)+\cdots+a_n(x_k,x_n)=0,\quad k=1,\ldots,n
\]
写作矩阵形式:
\[
    \text{Gram}(x_1,\ldots,x_n)\begin{bmatrix}a_1\\\vdots\\a_n\end{bmatrix}=0
\]
这是一个关于 $a_1,\ldots,a_n$ 的齐次线性方程,所以:
\[
\text{Gram 矩阵非奇异}\;\iff (a_1,\ldots,a_n)^T \;\text{只有零解}\; \iff x_1,\ldots,x_n\;\text{线性无关}
\]
\end{proof}

\begin{theorem}
设向量组 $X=(x_1,\ldots,x_n)$ 与向量组 $Y=(y_1,\ldots,y_n)$ 的 Gram 矩阵分别是 $A=\text{Gram}(X),\,B=\text{Gram}(Y)$,且 $Y=XC$(即 $C$ 是 $Y$ 在向量组 $X$ 下的表示矩阵),则:
\[
    B=C^TAC
\]
\end{theorem}
\begin{proof}
\[
    B_{ij}=(y_i,y_j)=\left(\sum_{k=1}^nc_{ki}x_k,\sum_{l=1}^nc_{lj}x_l\right)=\sum_{k=1}^n\sum_{l=1}^nc_{ki}c_{lj}(x_k,x_l)=\sum_{k=1}^n\sum_{l=1}^nc_{ki}A_{kl}c_{lj}
\]
写作矩阵形式就是 $B=C^TAC$.
\end{proof}

\begin{definition}[合同]
设 $A,B$ 为 $n$ 阶矩阵,若存在矩阵 $C$ 使得:
\[
B=C^TAC
\]
则称 $A$ 与 $B$ 合同。
\end{definition}

\begin{theorem}[Schwarz 不等式]
\[
    |(x,y)|\leq \Vert x\Vert\Vert y\Vert
\]
\end{theorem}
\begin{proof}
设有向量组 $X=(x_1,\ldots,x_m)$,设 $y$ 可由它们线性表示:$y=\sum_{i=1}^m\lambda_ix_i$,则:
\[
F(\lambda)=\Vert y\Vert^2=(y,y)=\left(\sum_{i=1}^m\lambda_ix_i,\sum_{j=1}^m\lambda_jx_j\right)=\sum_{i=1}^m\sum_{j=1}^m\lambda_i\lambda_j(x_i,x_j)=\lambda^T \text{Gram}(X)\lambda\geq0
\]
由于二次型 $F(\lambda)$ 非负,故 $\text{Gram}(X)$ 半正定,故 $\det(\text{Gram}(X))\geq 0$.
特别地,取 $m=2$,$X=(x,y)$,那么:
$$
\det\left(\begin{bmatrix}(x,x)&(x,y)\\(y,x)&(y,y)\end{bmatrix}\right)=(x,x)(y,y)-(x,y)(y,x)\geq 0
$$
化简即得 Schwarz 不等式。
\end{proof}
\begin{example}
设 $x=(x_1,\ldots,x_n)\in\mathbb R^n,\,y=(y_1,\ldots,y_n)\in\mathbb R^n$,则根据 Schwarz 不等式有:
\[
    |x_1y_1+\cdots+x_ny_n|^2\leq (x_1^2+\cdots+x_n^2)(y_1^2+\cdots+y_n^2)
\]
\end{example}
\begin{example}
设 $f,g$ 为 $[-1,1]$ 上的实值连续函数,则根据 Schwarz 不等式有:
\[
    \left|\int_{-1}^1f(x)g(x)\mathrm dx\right|^2\leq \left(\int_{-1}^1f^2(x)\mathrm dx\right)\left(\int_{-1}^1g^2(x)\mathrm dx\right)
\]
\end{example}

\begin{theorem}[三角不等式]
\[
    \Vert x+y\Vert\leq \Vert x\Vert+\Vert y\Vert
\]
\end{theorem}
\begin{proof}
\[
    \Vert x+y\Vert^2=(x+y,x+y)=(x,x)+2(x,y)+(y,x)\leq \Vert x\Vert^2+2\Vert x\Vert\Vert y\Vert+\Vert y\Vert^2=(\Vert x\Vert+\Vert y\Vert)^2
\]
\end{proof}

\begin{theorem}[平行四边形恒等式]
\label{thm:parallelogram}
\[
    \Vert x+y\Vert^2+\Vert x-y\Vert^2=2(\Vert x\Vert^2+\Vert y\Vert^2)
\]
\end{theorem}
\begin{proof}
\[
    \Vert x+y\Vert^2+\Vert x-y\Vert^2=\Vert x\Vert^2+\Vert y\Vert^2+2(x,y)+\Vert x\Vert^2+\Vert y\Vert^2-2(x,y)=2(\Vert x\Vert^2+\Vert y\Vert ^2)
\]
\end{proof}

\begin{theorem}[Reisz 表示定理]
欧氏空间 $V^n$ 中所有的线性函数都可以表示为内积的形式,即:设 $l(x)$ 为 $V^n$ 的一个线性函数,则存在一个向量 $u_l\in V^n$,使得对任一 $x\in V^n$ 都有 $l(x)=(u_l,x)$.
\end{theorem}
\begin{proof}
取 $V^n$ 中的一个基 $X=(x_1,\ldots,x_n)$,设 $x=\sum_{i=1}^n\alpha_ix_i$,则:
\[
    l(x)=l\left(\sum_{i=1}^n\alpha_ix_i\right)=\sum_{i=1}^n\alpha_i l(x_i)
\]
定义内积为基 $X$ 下坐标的内积,那么构造 $u_l$ 为对应坐标 $(l(x_1),\ldots,l(x_n))$ 的向量,即:
\[
    u_l=X\big(l(x_1),\ldots,l(x_n)\big)=\sum_{i=1}^nl(x_i)x_i
\]
那么就有 $l(x)=(u_l,x)$.
\end{proof}

\begin{definition}[正交]
若 $(x,y)=0$,则称 $x$ 与 $y$ 正交,记作 $x\perp y$.
\end{definition}
\begin{definition}[正交向量组]
若欧式空间中的一组非零向量两两正交,则称之为正交向量组。
\end{definition}

\begin{theorem}
\label{thm:perp-ind}
正交向量组一定线性无关。
\end{theorem}
\begin{proof}
设 $(x_1,x_2,\ldots,x_n)$ 是一个正交向量组,且 $\alpha_1x_1+\alpha2x_2+\cdots\alpha_nx_n=0$,则:
\[
    (\alpha_1x_1+\alpha2x_2+\cdots\alpha_nx_n,x_1)=\alpha_1(x_1,x_1)+\alpha_2(x_2,x_1)+\cdots+\alpha_n(x_n,x_1)=\alpha_1(x_1,x_1)=0
\]
由于 $x_1\neq 0$,故 $\alpha_1=0$;同理可得 $\alpha_1=\alpha_2=\cdots=\alpha_n=0$,故 $x_1,x_2,\ldots,x_n$ 线性无关。
\end{proof}

\begin{definition}[正交基,标准正交基]
在欧式空间 $V^n$ 中,$n$ 个正交向量组成的极大线性无关组构成 $V^n$ 的正交基;由单位向量组成的正交基称作标准正交基。
\end{definition}

\begin{theorem}[任意欧式空间中都存在一组正交基]
对 $V^n$ 中的任意一个基 $(x_1,x_2,\ldots,x_n)$,存在一组正交基 $(y_1,y_2,\ldots,y_n)$ 满足:
\[
L(x_1,x_2,\ldots,x_i)=L(y_1,y_2,\ldots,y_i),\quad\forall i=1,2,\ldots,n
\]
证明过程就是 Gram-Schmidt 正交化过程。
\end{theorem}

\vskip 6pt \noindent\textbf{Gram-Schmidt 正交化过程}:

1) 首先做正交化:
\begin{align*}
    &y_1=x_1\\
    &y_2=x_2-\frac{(x_2,y_1)}{(y_1,y_1)}y_1\\
    &y_3=x_3-\frac{(x_3,y_1)}{(y_1,y_1)}y_1-\frac{(x_3,y_2)}{(y_1,y_2)}y_2\\
    &\cdots\\
    &y_i=x_i-\sum_{k=1}^{i-1}\frac{(x_i,y_k)}{(y_k,y_k)}y_k\\
    &\cdots
\end{align*}

2) 然后做归一化:
\[
    z_i=\frac{y_i}{\Vert y_i\Vert},\quad i=1,\ldots,n
\]
\begin{proof}
正交性:数学归纳法,假设前 $y_1,\ldots,y_{i-1}$ 两两正交,那么对于 $j=1,\ldots,i-1$,有:
\begin{align*}
    (y_i,y_j)&=\left(x_i-\sum_{k=1}^{i-1}\frac{(x_i,y_k)}{(y_k,y_k)}y_k,y_j\right)\\
    &=(x_i,y_j)-\left(\sum_{k=1}^{i-1}\frac{(x_i,y_k)}{(y_k,y_k)}y_k,y_j\right)\\
    &=(x_i,y_j)-\sum_{k=1}^{i-1}\frac{(x_i,y_k)}{(y_k,y_k)}(y_k,y_j)\\
    &=(x_i,y_j)-\frac{(x_i,y_j)}{(y_j,y_j)}(y_j,y_j)\\
    &=0
\end{align*}
即 $y_i\perp y_j$. 根据归纳法,$y_1,\ldots,y_n$ 两两正交。
\end{proof}

\begin{definition}[子空间的正交性]
设 $V^n$ 的两个子空间 $V_1,V_2$ 满足:$\forall x\in V_1,\forall y\in V_2$,$(x,y)=0$,称 $V_1$ 与 $V_2$ 正交。
\end{definition}

\begin{definition}[正交补]
设 $V_1$ 为欧式空间 $V^n$ 的子空间,则定义其正交补为:
\[V_1^{\perp}=\{x\mid(x,y)=0,\forall y\in V_1,x\in V^n\}\]
\end{definition}

\begin{theorem}
\label{thm:v1+v1p}
设 $V_1$ 为欧式空间 $V^n$ 的子空间,则:
\[V_1\oplus V_1^{\perp}=V^n\]
\end{theorem}
\begin{proof}
显然有 $V_1\cap V_1^{\perp}=\{0\}$ 且 $V_1+V_1^{\perp}\subset V^n$,故只需证明 $V_1+V_1^{\perp}\supset V^n$.
设 $V_1$ 的一个正交基为 $(x_1,\ldots,x_r)$,任取 $z\in V^n$,设 $x=\sum_{i=1}^r(z,x_i)x_i$,则只需证 $y=z-x\in V_1^{\perp}$,即证 $(y,x_i)=0$.
\begin{align*}
    (y,x_i)&=(z-x,x_i)=\left(z-\sum_{j=1}^n(z,x_j)x_j,x_i\right)\\
    &=(z,x_i)-\sum_{i=1}^n(z,x_j)(x_j,x_i)=(z,x_i)-(z,x_i)=0
\end{align*}
\end{proof}

\begin{theorem}
对任意矩阵 $A\in\mathbb R^{m\times n}$,有:
\begin{align*}
    &R^{\perp}(A)=N(A^T),\quad R(A)\oplus N(A^T)=\mathbb R^m\\
    &R^{\perp}(A^T)=N(A),\quad R(A^T)\oplus N(A)=\mathbb R^n
\end{align*}
\end{theorem}
\begin{proof}
设 $A=(a_1,\ldots,a_n)$,其中列向量 $a_i\in R(A)$. 设 $x\in\mathbb R^m$,有:
\[
    x\in R^{\perp}(A) \iff (x,a_i)=0,\;i=1,\ldots,n \iff A^Tx=0 \iff x\in N(A^T)
\]
故 $R^{\perp}(A)=N(A^T)$. 同理可得 $R^{\perp}(A^T)=N(A)$. 再根据定理 \ref{thm:v1+v1p} 可得 $R(A)\oplus N(A^T)=\mathbb R^m,\,R(A^T)\oplus N(A)=\mathbb R^n$.
\end{proof}
\begin{remark}
该定理说明 $R(A)$ 与 $N(A^T)$ 互为正交补、$R(A^T)$ 与 $N(A)$ 互为正交补,即 Gilbert Strang 的四个基本子空间图中垂直符号的意义:
\begin{figure}[H]
    \centering
    \includegraphics[width=0.55\linewidth]{figs/spaces.png}
\end{figure}
\end{remark}

\begin{definition}[正交变换]
设 $V$ 是一个欧式空间,$T$ 为其上的线性变换。若 $T$ 保持 $V$ 中任意向量 $x$ 长度不变,即 $\Vert Tx\Vert=\Vert x\Vert$,则称 $T$ 为正交变换。
\end{definition}

\begin{theorem}[正交变换的等价定义]
欧式空间 $V$ 上的线性变换 $T$ 是正交变换的充要条件是保持内积不变,即对任意 $x,y\in V$,有 $(Tx,Ty)=(x,y)$.
\end{theorem}

\begin{definition}[正交矩阵]
若方阵 $Q$ 满足:$Q^TQ=I$ 或 $Q^{-1}=Q^T$.  即 $Q$ 各列向量标准正交,则称 $Q$ 为正交矩阵。
\end{definition}

\begin{theorem}[正交变换与正交矩阵]
欧式空间上的线性变换 $T$ 为正交变换的充要条件是其在\textbf{标准正交基}下的矩阵表示是正交矩阵。
\end{theorem}
\begin{proof}
设 $X$ 为一个标准正交基,$TX=XA$,任取 $x=X\alpha$,则 $Tx=TX\alpha=XA\alpha$,因此:
\begin{align*}
    &(Tx,Tx)=(A\alpha)^T(A\alpha)=\alpha^TA^TA\alpha=(x,x)=\alpha^T\alpha\\
    \iff\;&\alpha^T(I-A^TA)\alpha=0\\
    \iff\;& A^TA=I
\end{align*}
\end{proof}
\begin{note}
注意定理成立必须是在标准正交基下。
\end{note}

\begin{property}
正交矩阵非奇异。
\end{property}

\begin{property}
正交矩阵的逆仍为正交矩阵。
\end{property}

\begin{property}
正交矩阵的乘积仍为正交矩阵。
\end{property}

\begin{property}
正交基变换矩阵为正交矩阵。
\end{property}
\begin{proof}
设 $X,Y$ 为正交基,$Y=XC$,任取 $x=Y\alpha=XC\alpha,\,y=Y\beta=XC\beta$,则:
\[
    (x,y)=\alpha^T\beta=(C\alpha)^T(C\beta)=\alpha^T(C^TC)\beta\implies C^TC=I
\]
\end{proof}

\begin{property}
正交矩阵的特征值位于复平面的单位圆上。
\end{property}
\begin{proof}
设 $A$ 为正交矩阵,$Ax=\lambda x\ (x\neq 0)$,则两边取共轭转置得 $x^HA^T=\bar\lambda x^H$(注意 $A$ 是实矩阵,但其特征值和特征向量可能是复数)。于是:
\[
    x^Hx=x^HA^TAx=\lambda\bar\lambda x^Hx=|\lambda|^2x^Hx\implies |\lambda|^2=1
\]
\end{proof}

\begin{definition}[线性映射的共轭]
设 $P$ 是欧氏空间 $W$ 到欧氏空间 $V$ 的一个线性映射,$Q$ 是 $V$ 到 $W$ 的一个线性映射,若对 $\forall x\in W,y\in V$,有 $(Px,y)=(x,Qy)$,则称 $Q$ 为 $P$ 的共轭。
\end{definition}

\begin{figure}[H]
    \centering
    \includegraphics[width=0.5\linewidth]{figs/conj.png}
\end{figure}

\begin{theorem}
设 $X,Y$ 分别是欧式空间 $W,V$ 的标准正交基,$P$ 是 $W$ 到 $V$ 的一个线性映射,$Q$ 是 $P$ 的共轭。设 $P,Q$ 在 $X,Y$ 下的矩阵表示为 $A,B$,则 $B=A^T$.
\end{theorem}
\begin{proof}
由于 $PX=YA,\,QY=XB$,所以:
\begin{gather*}
    (Px_j,y_i)=\left(\sum_{k=1}^m a_{kj}y_k,y_i\right)=a_{ij}\\
    (x_j,Qy_i)=\left(x_j,\sum_{k=1}^nb_{ki}x_k\right)=b_{ji}
\end{gather*}
由于 $(Px_j,y_i)=(x_j,Qy_i)$,故 $a_{ij}=b_{ji}$,即 $B=A^T$.
\end{proof}

\begin{definition}[线性变换的共轭]
设 $T$ 是欧氏空间 $V$ 上的一个线性变换,若对 $\forall x,y\in V$,有 $(Tx,y)=(x,T^\ast y)$ 成立,则称 $T^\ast$ 为 $T$ 的共轭。
\end{definition}

\begin{theorem}
\label{thm:conjgram}
设 $T$ 在基 $X=(x_1,\ldots,x_n)$ 下的矩阵表示为 $A$,$X$ 的 Gram 矩阵为 $C$,那么 $T^\ast$ 在基 $X$ 下的矩阵表示为 $B=C^{-1}A^TC$.
\end{theorem}
\begin{proof}
由于 $TX=XA,\,T^\ast X=XB$,所以:
\begin{gather*}
    (Tx_i,x_j)=\left(\sum_{k=1}^na_{ki}x_k,x_j\right)=\sum_{k=1}^na_{ki}(x_k,x_j)=\sum_{k=1}^na_{ki}c_{kj}\\
    (x_i,T^\ast x_j)=\left(x_i,\sum_{k=1}^nb_{kj}x_k\right)=\sum_{k=1}^nb_{kj}(x_i,x_k)=\sum_{k=1}^nb_{kj}c_{ik}\\
\end{gather*}
得 $\sum_{k=1}^na_{ki}c_{kj}=\sum_{k=1}^n b_{kj}c_{ik}$,即 $A^TC=CB$.
\end{proof}

\begin{definition}[实对称变换]
设 $T$ 是欧氏空间 $V$ 的一个线性变换,且对 $V$ 中任意两个向量 $x,y$ 都有 $(Tx,y)=(x,Ty)$ 成立,则称 $T$ 为 $V$ 中一个实对称变换。
\end{definition}

\begin{theorem}[实对称变换与实对称矩阵]
欧氏空间中的线性变换是实对称变换的充要条件是它在\textbf{标准正交基}下的矩阵为实对称矩阵。
\end{theorem}
\begin{proof}
设 $X$ 为一个标准正交基,设 $T$ 在 $X$ 下的矩阵表示为 $A$,即 $TX=XA$.

必要性:由于 $X$ 为标准正交基,故其 Gram 矩阵为 $I$,由于 $T$ 本身就是自己的共轭,根据定理 \ref{thm:conjgram} 可知 $A=I^{-1}A^TI=A^T$.

充分性:
\begin{gather*}
(Tx_i,x_j)=\left(\sum_{k=1}^na_{ki}x_k,x_j\right)=a_{ji}\\
(x_i,Tx_j)=\left(x_i,\sum_{k=1}^na_{kj}x_k\right)=a_{ij}
\end{gather*}
得 $a_{ji}=a_{ij}$.
\end{proof}

\begin{theorem}
实对称矩阵特征值都为实数,属于不同特征值的特征向量相互正交。
\end{theorem}
\begin{proof}
设 $Ax=\lambda x\ (x\neq 0)$,则:
\[
x^HAx=\lambda x^Hx=(A^Hx)^Hx=(Ax)^Hx=(\lambda x)^Hx=\bar\lambda x^Hx\implies \lambda=\bar\lambda
\]
故 $\lambda\in\mathbb R$.
再设 $Ay=\mu y\ (y\neq 0)$ 且 $\lambda\neq \mu$,则:
\[
y^TAx=\lambda y^Tx=(A^Ty)^Tx=(Ay)^Tx=\mu y^Tx\implies y^Tx=0
\]
\end{proof}

下面我们将线性空间的数域从实数域扩展到复数域,那么相应的,欧式空间扩展为酉空间,内积扩展为复内积,正交变换和正交矩阵扩展为酉变换和酉矩阵,实对称变换和实对称矩阵扩展为 Hermite 矩阵和 Hermite 变换。它们都有着类似的性质。

\begin{definition}[酉空间,复内积]
设 $V$ 是复数域 $C$ 上的线性空间,对 $V$ 中任意 $x$ 和 $y$,按某种规则定义一个复数,用 $(x,y)$ 表示,且满足下列四个条件:
\begin{itemize}
    \item 交换律:$(x,y)=\overline{(y,x)}$
    \item 分配律:$(x,y+z)=(x,y)+(x,z)$
    \item 齐次性:$(kx,y)=k(x,y),\,\forall k\in \mathbb C$
    \item 非负性:$(x,x)\geq 0$,当且仅当 $x=0$ 时 $(x,x)=0$
\end{itemize}
则称 $(x,y)$ 为复内积,$V$ 为酉空间、复欧氏空间或复内积空间。
\end{definition}

\begin{example}[根据坐标定义复内积]
设 $X$ 为 $V$ 上的一个基,向量 $x,y\in V$ 在该基下的坐标分别为 $\alpha=(\alpha_1,\ldots,\alpha_n)^T,\beta=(\beta_1,\ldots,\beta_n)^T$,则可以定义内积为:
\[
    (x,y)=\alpha_1\bar\beta_1+\cdots+\alpha_n\bar\beta_n=\beta^H\alpha
\]
容易验证这确实满足内积的四个条件。
\end{example}
\begin{note}
注意共轭转置取在 $\beta$ 上,这是为了满足齐次性而导致的。在有些教材上,齐次性写作 $(x,ky)=k(x,y)$,则内积相应地会变成 $\alpha^H\beta$,共轭转置取在 $\alpha$ 上。
\end{note}

\begin{definition}[酉变换]
设 $T$ 是酉空间 $V$ 中的线性变换,若 $T$ 保持长度不变,即对 $V$ 中任意 $x$,有 $(Tx,Tx)=(x,x)$,则称 $T$ 为 $V$ 上的酉变换。
\end{definition}

\begin{theorem}[酉变换的等价定义]
酉空间 $V$ 上的线性变换 $T$ 是酉变换的充要条件是保持内积不变,即对任意 $x,y\in V$,有 $(Tx,Ty)=(x,y)$.
\end{theorem}

\begin{definition}[酉矩阵]
若 $n$ 阶矩阵 $A$ 满足 $A^HA=AA^H=I$,则称 $A$ 为酉矩阵。
\end{definition}

\begin{theorem}[酉变换与酉矩阵]
酉变换在酉空间的\textbf{标准正交基}下的矩阵是酉矩阵。
\end{theorem}

\begin{property}
酉矩阵的逆矩阵也是酉矩阵。
\end{property}

\begin{property}
两个酉矩阵的乘积也是酉矩阵。
\end{property}

\begin{definition}[复线性映射的共轭]
设 $P$ 是酉空间 $W$ 到酉空间 $V$ 的一个线性映射,$Q$ 是酉空间
$V$ 到酉空间 $W$ 的一个线性映射,若对 $\forall x\in W,y\in V$,有 $(Px,y)=(x,Qy)$,则称 $Q$ 为 $P$ 的共轭。
\end{definition}

\begin{theorem}
设 $X,Y$ 分别是酉空间 $W,V$ 的标准正交基,$P$ 是 $W$ 到 $V$ 的一个线性映射,$Q$ 是 $P$ 的共轭。设 $P,Q$ 在 $X,Y$ 下的矩阵表示为 $A,B$,则 $B=A^H$.
\end{theorem}

\begin{definition}[复线性变换的共轭]
设 $T$ 是酉空间 $V$ 上的一个线性变换,若对 $\forall x,y\in V$,有 $(Tx,y)=(x,T^\ast y)$ 成立,则称 $T^\ast$ 为 $T$ 的共轭。
\end{definition}

\begin{property}
线性变换 $T$ 的共轭仍是线性变换。
\end{property}
\begin{theorem}
设 $T$ 在基 $X=(x_1,\ldots,x_n)$ 下的矩阵表示为 $A$,$X$ 的 Gram 矩阵为 $C$,那么 $T^\ast$ 在基 $X$ 下的矩阵表示为 $B=C^{-1}A^HC$.
\end{theorem}

\begin{definition}[Hermite 变换/酉对称变换]
设 $T$ 为酉空间 $V$ 上的线性变换,若满足对任意 $x,y\in V$,都有 $(Tx,y)=(x,Ty)$,则称 $T$ 为 Hermite 变换或酉对称变换。
\end{definition}

\begin{definition}[Hermite 矩阵]
若 $n$ 阶矩阵 $A$ 满足 $A=A^H$,则称 $A$ 为 Hermite 矩阵。
\end{definition}

\begin{theorem}[Hermite 变换与 Hermite 矩阵]
Hermite 变换在\textbf{标准正交基}下的矩阵是 Hermite 矩阵。
\end{theorem}

\begin{theorem}
Hermite 矩阵的特征值都是实数,属于不同特征值的特征向量相互正交。
\end{theorem}

\begin{theorem}[Schur 定理]
\label{thm:schur}
\ 

1) 设 $A\in\mathbb C^{n\times n}$,则存在酉矩阵 $P$ 使得 $P^HAP=U$,其中 $U$ 为上三角矩阵。

2) 设 $A\in\mathbb R^{n\times n}$ 且所有特征值为实数,则存在正交矩阵 $Q$ 使得 $Q^TAQ=U$,其中 $U$ 为上三角矩阵。
\end{theorem}

\begin{remark}
在定理 \ref{thm:anysimilar} 中,我们证明了任意矩阵都相似于一个上三角矩阵。Schur 定理是该定理的加强版,它限制用来相似化的矩阵是一个酉矩阵。Schur 定理的证明过程与定理 \ref{thm:anysimilar} 是类似的,只不过基扩充时需要扩充为标准正交基,所有的可逆矩阵换成酉矩阵。
\end{remark}

\begin{definition}[正规矩阵]
设 $A\in\mathbb C^{n\times n}$ 且 $A^HA=AA^H$,称 $A$ 为正规矩阵。
\end{definition}
\begin{remark}
式 $A^HA=AA^H$ 意味着 $A$ 的 $i,j$ 行内积等于 $i,j$ 列内积。因此,前面提到的\textbf{正交矩阵、对称矩阵、酉矩阵、Hermite 矩阵都是正规矩阵}。
\end{remark}

\begin{theorem}[正规矩阵的充要条件]
\label{thm:unisim}
\ 

1) 设 $A\in\mathbb C^{n\times n}$,则 $A$ 为正规矩阵的充要条件为 $A$ 酉相似于对角矩阵,即存在酉矩阵 $P$ 使得 $P^HAP=D$,其中 $D$ 为对角矩阵。

2) 设 $A\in\mathbb R^{n\times n}$ 且所有特征值为实数,则 $A$ 为正规矩阵的充要条件为 $A$ 正交相似于对角矩阵,即存在正交矩阵 $Q$ 使得 $Q^TAQ=D$,其中 $D$ 为对角矩阵。
\end{theorem}
\begin{proof}
只证明 1),2) 类似可证。充分性易证;对于必要性,根据 Schur 定理 \ref{thm:schur},$A$ 酉相似于一个上三角矩阵:$P^HAP=U$. 容易证明,$A$ 正规 $\iff$ $U$ 正规,于是 $U^HU=UU^H$,故 $U$ 只能是对角矩阵。
\end{proof}

\begin{corollary}
实对称矩阵正交相似于对角矩阵。
\end{corollary}

% \begin{theorem}[Hermite 矩阵的谱分解]
% 设 $A$ 为 Hermite 矩阵,$\lambda_i,p_i$ 是 $A$ 的特征值和特征向量,则:
% \[
%     A=\lambda_1p_1p_1^H+\cdots+\lambda_np_np_n^H=P\Lambda P^H
% \]
% \end{theorem}
 \newpage
\section{随机变量}

\subsection{基本概念}

\begin{definition}[随机变量]
随机变量是试验结果的实值函数。
\begin{figure}[H]
    \centering
    \includegraphics[width=0.65\linewidth]{figs/随机变量示意图.png}
    \caption{随机变量示意图}
    \label{fig:random-variable}
\end{figure}
\end{definition}
\begin{com}
一般用大写字母表示随机变量,小写字母表示其取值。
\end{com}

\begin{definition}[离散随机变量]
若一个随机变量的值域为有限集或可数集,则称这个随机变量是离散的。
\end{definition}

\begin{definition}[概率质量函数]
定义随机变量 $X$ 取值为 $x$ 的概率为事件 $\{X=x\}$ 的概率,即所有与 $x$ 对应的试验结果组成的事件的概率,记作 $p_X(x)$,即:
\[p_X(x)=\Pb(\{X=x\})\]
称 $p_X$ 为 $X$ 的概率质量函数 (PMF).
\end{definition}

\begin{property}
PMF 满足非负性和归一化条件:
\[\sum_x p_X(x)=\sum_x\Pb(X=x)=1\]
\end{property}
\begin{property}
设 $S$ 为任一 $X$ 可能取值的集合,则:
\[\Pb(X\in S)=\sum_{x\in S}p_X(x)\]
\end{property}

\begin{example}
常见的离散随机变量包括伯努利、二项、几何和泊松随机变量等,详见附录 \ref{sec:random-variables}.
\end{example}

\begin{definition}[连续随机变量,概率密度函数]
对随机变量 $X$,若存在一个非负函数 $f_X$,使得:
\[
\Pb(X\in B)=\int_Bf_X(x)\mathrm dx
\]
对实数轴的集合 $B$ 都成立\footnote{本书只考虑黎曼积分,且 $f_X$ 为有有限/可数个间断点的分段连续函数。},则称 $X$ 为连续的随机变量,函数 $f_X$ 称为概率密度函数 (PDF). 特别地,当 $B$ 是一个区间时,有:
\[
\Pb(a\leq X\leq b)=\int_a^bf_X(x)\mathrm dx
\]
\end{definition}
\begin{property}
PDF 满足非负性和归一化条件:
\[\int_{-\infty}^{\infty}f_X(x)\mathrm dx=\Pb(-\infty<X<\infty)=1\]
\end{property}
\begin{property}
对于充分小的 $\delta$,有:
\[\Pb(x\leq X\leq x+\delta)=\int_x^{x+\delta}f_X(x)\mathrm dx\approx f(x)\cdot\delta\]
\end{property}

\begin{example}
常见的连续随机变量包括均匀、指数和正态随机变量等,详见附录 \ref{sec:random-variables}.
\end{example}


\subsection{分布函数}

\begin{definition}[分布函数]
设 $X$ 是一个随机变量(离散或连续),定义其分布函数(CDF)$F_X$ 为:
\[
F_X(x)=\Pb(X\leq x)=\begin{dcases}
    \sum_{k\leq x}p_X(k),&\text{$X$ 离散}\\
    \int_{-\infty}^xf_X(t)\mathrm dt,&\text{$X$ 连续}
\end{dcases}
\]
\end{definition}
\begin{com}
分布函数统一刻画了离散和连续情形。离散情形下的 PMF、连续情形下的 PDF 和一般情形下的 CDF 都是相应随机变量的概率律。
\end{com}

\begin{property}
设 $F_X$ 是随机变量 $X$ 的分布函数,则:
\begin{itemize}
    \item $F_X$ 单调非减。
    \item $F_X(x)\to 0\;(x\to-\infty),;F_X(x)\to1(x\to\infty)$.
    \item 当 $X$ 是离散随机变量时,$F_X$ 是阶梯函数。
    \item 当 $X$ 是连续随机变量时,$F_X$ 是连续函数。
\end{itemize}
\end{property}

\begin{theorem}[分布列与分布函数]
设 $X$ 是离散随机变量且取整数值,则:
\[
F_X(k)=\sum_{i=-\infty}^kp_X(i),\quad p_X(k)=F_X(k)-F_X(k-1)
\]
\end{theorem}
\begin{theorem}[概率密度函数与分布函数]
设 $X$ 是连续随机变量,则:
\[
F_X(x)=\int_{-\infty}^xf_X(t)\mathrm dt,\quad f_X(x)=\frac{\mathrm d}{\mathrm dx}F_X(x)
\]
第二个等式只在分布函数可微处成立。
\end{theorem}


\subsection{期望和方差}

\begin{definition}[期望/均值]
随机变量 $X$ 的期望定义为:
\[
\E X=\begin{dcases}
    \sum_x xp_X(x),&\text{$X$ 离散}\\
    \int_{-\infty}^{\infty}xf_X(x)\mathrm dx,&\text{$X$ 连续}
\end{dcases}
\]
特别地,对于连续情形,若 $f_X$ 不是绝对可积的,即 $\int_{-\infty}^\infty |x|f_X(x)\mathrm dx=\infty$,则称期望不存在。
\end{definition}

\begin{definition}[矩,中心矩]
定义随机变量 $X$ 的 $n$ 阶矩为 $\E[X^n]$,$n$ 阶中心矩为 $\E[(X-\E X)^n]$.
\end{definition}

\begin{definition}[方差,标准差]
定义随机变量 $X$ 的方差为其 2 阶中心矩,标准差为方差的平方根:
\[
\var(X)=\E\left[(X-\E X)^2\right],\quad \sigma(X)=\sqrt{\var(X)}
\]
\end{definition}

\begin{theorem}[随机变量函数的期望]
设 $X$ 是一随机变量,则 $Y=g(X)$ 的期望为:
\[\E Y=\E[g(X)]=\begin{dcases}
    \sum_xg(x)p_X(x),&\text{$X$ 离散}\\
    \int_{-\infty}^{+\infty}g(x)f_X(x)\mathrm dx,&\text{$X$ 连续}
\end{dcases}
\]
因此,我们不必先求出 $Y$ 的分布,只需知道 $X$ 的分布就能求出 $Y$ 的期望。
\end{theorem}

\begin{theorem}[随机变量的线性函数的期望和方差]
设 $X$ 是一个随机变量,$Y=aX+b$,其中 $a,b$ 为常数,则:
\[\E Y=a\E X+b,\quad \var(Y)=a^2\var(X)\]
\end{theorem}
\begin{proof}
仅对离散情形证明,连续情形类似。
\begin{gather*}
\E Y=\sum_x(ax+b)p_X(x)=a\sum_x xp_X(x)+b\sum_xp_X(x)=a\E X+b\\
\var(Y)=\E[(Y-\E Y)^2]=\E[((aX+b)-(a\E X+b))^2]=a^2\E[(X-\E X)^2]=a^2\var(X)
\end{gather*}
\end{proof}

\begin{theorem}[用矩表达方差]
设 $X$ 是一个随机变量,则:
\[
\var(X)=\E[X^2]-(\E X)^2
\]
\end{theorem}
\begin{proof}
\begin{align*}
\var(X)&=\E\left[(X-\E X)^2\right]\\
&=\E\left[X^2-2X\E X+(\E X)^2\right]\\
&=\E[X^2]-2 \E X\cdot \E X+(\E X)^2\\
&=\E[X^2]-(\E X)^2
\end{align*}
\end{proof}

\begin{example}
常见随机变量的期望和方差及其推导过程见附录 \ref{sec:random-variables}.
\end{example}


\subsection{联合分布与边缘分布}

\begin{figure}[H]
    \centering
    \includegraphics[width=0.8\linewidth]{figs/联合分布与边缘分布.png}
    \caption{联合分布与边缘分布的关系概览}
    \label{fig:union-marginal}
\end{figure}

\begin{definition}[联合概率质量函数]
设 $X,Y$ 是离散随机变量,定义 $(X,Y)$ 取值 $(x,y)$ 的概率为事件 $\{X=x,Y=y\}$ 的概率,记作 $p_{X,Y}(x,y)$,即:\[p_{X,Y}(x,y)=\Pb(X=x,Y=y)\]
称 $p_{X,Y}$ 为 $X,Y$ 的联合概率质量函数。
\end{definition}

\begin{definition}[联合概率密度函数]
设 $X,Y$ 是连续随机变量,若存在一个非负二元函数 $f_{X,Y}$ 使得:
\[
\Pb((X,Y)\in B)=\iint_{(x,y)\in B}f_{X,Y}(x,y)\mathrm dx\mathrm dy
\]
对平面上任意集合 $B$ 成立,则称 $f_{X,Y}$ 为联合概率密度函数。特别地,当 $B$ 是一个矩形区域时有:
\[
\Pb(a\leq X\leq b,c\leq Y\leq d)=\int_c^d\int_a^b f_{X,Y}(x,y)\mathrm dx\mathrm dy
\]
\end{definition}

\begin{property}
联合 PDF 满足归一化条件:
\[\int_{-\infty}^{\infty}\int_{-\infty}^{\infty}f_{X,Y}(x,y)\mathrm dx\mathrm dy=1\]
\end{property}
\begin{property}
对于充分小的 $\delta$,有:
\[\Pb(a\leq X\leq a+\delta,c\leq Y\leq c+\delta)=\int_c^{c+\delta}\int_a^{a+\delta}f_{X,Y}(x,y)\mathrm dx\mathrm dy\approx f_{X,Y}(a,c)\cdot\delta^2\]
\end{property}

\begin{theorem}[边缘概率质量函数]
设 $X,Y$ 是离散随机变量且联合 PMF 为 $p_{X,Y}$,则:
\[
p_X(x)=\sum_y p_{X,Y}(x,y),\quad p_Y(y)=\sum_xp_{X,Y}(x)
\]
称 $p_X$ 和 $p_Y$ 为边缘概率质量函数。
\end{theorem}

\begin{theorem}[边缘概率密度函数]
设 $X,Y$ 是连续随机变量且联合 PDF 为 $f_{X,Y}$,则:
\[
f_X(x)=\int_{-\infty}^{\infty}f_{X,Y}(x,y)\mathrm dy,\quad f_Y(y)=\int_{-\infty}^{\infty}p_{X,Y}(x)\mathrm dx
\]
称 $f_X$ 和 $f_Y$ 为边缘概率密度函数。
\end{theorem}

\begin{definition}[联合分布函数]
设 $X,Y$ 是两个随机变量(离散或连续),定义其联合分布函数为:
\[
F_{X,Y}(x,y)=\Pb(X\leq x,Y\leq y)
\]
\end{definition}

\begin{theorem}[联合概率密度函数与联合分布函数]
若随机变量 $X,Y$ 有联合概率密度函数 $f_{X,Y}$,则:
\[
F_{X,Y}(x,y)=\int_{-\infty}^x\int_{-\infty}^yf_{X,Y}(s,t)\mathrm ds\mathrm dt,\quad f_{X,Y}(x,y)=\frac{\partial^2}{\partial x\partial y}F_{X,Y}(x,y)
\]
\end{theorem}

\begin{theorem}[随机变量的二元函数的期望]
设 $X$ 和 $Y$ 是随机变量,则 $Z=g(X,Y)$ 的期望为:
\[
\E Z=\E[g(X,Y)]=\begin{dcases}
    \sum_{x}\sum_{y}g(x,y)p_{X,Y}(x,y),&\text{$X,Y$ 离散}\\
    \int_{-\infty}^{\infty}\int_{-\infty}^{\infty}g(x,y)f_{X,Y}(x,y)\mathrm dx\mathrm dy,&\text{$X,Y$ 连续}
\end{dcases}
\]
\end{theorem}

\begin{theorem}[随机变量的二元线性函数的期望]
设 $X$ 和 $Y$ 是随机变量,$a,b,c$ 为常数,则:
\[
\E[aX+bY+c]=a\E X+b\E Y+c
\]
\end{theorem}

\begin{theorem}[随机变量的多元线性函数的期望]
设 $X_1,X_2,\ldots,X_n$ 是 $n$ 个随机变量,$a_1,a_2,\ldots,a_n$ 是 $n$ 个常数,则:
\[
\E[a_1X_1+a_2X_2+\cdots+a_nX_n]=a_1\E X_2+a_2\E X_2+\cdots+a_n\E X_n
\]
\end{theorem}


\subsection{条件}

\begin{definition}[条件分布]
设 $X,Y$ 是离散随机变量,定义给定 $Y=y$ 下 $X$ 的条件概率质量函数为:
\[
p_{X\vert Y}(x\vert y)=\Pb(X=x\vert Y=y)=\frac{\Pb(X=x,Y=y)}{\Pb(Y=y)}=\frac{p_{X,Y}(x,y)}{p_Y(y)}
\]
类似地,设 $X,Y$ 是连续随机变量,定义给定 $Y=y$ 下 $X$ 的条件概率密度函数为:
\[
f_{X\vert Y}(x\vert y)=\frac{f_{X,Y}(x,y)}{f_Y(y)}
\]
\end{definition}

\begin{property}
条件 PMF/PDF 满足归一化条件:
\begin{gather*}
\sum_x p_{X\vert Y}(x\vert y)=1,\quad\text{$X$ 离散}\\
\int_{-\infty}^{\infty}f_{X\vert Y}(x\vert y)\mathrm dx=1,\quad\text{$X$ 连续}
\end{gather*}
\end{property}

\begin{theorem}[条件分布与联合分布]
设 $X,Y$ 是随机变量,由定义可知:
\begin{gather*}
p_{X,Y}(x,y)=p_Y(y)p_{X\vert Y}(x\vert y)=p_X(x)p_{Y\vert X}(y\vert x),\quad\text{$X$ 离散}\\
f_{X,Y}(x,y)=f_Y(y)f_{X\vert Y}(x\vert y)=f_X(x)f_{Y\vert X}(y\vert x),\quad\text{$X$ 连续}
\end{gather*}
\end{theorem}

\begin{theorem}[条件分布与边缘分布]
设 $X,Y$ 是随机变量,则根据全概率公式,有:
\begin{gather*}
p_X(x)=\sum_y p_{X\vert Y}(x\vert y)p_Y(y),\quad\text{$X$ 离散}\\
f_X(x)=\int_{-\infty}^{\infty} f_{X\vert Y}(x\vert y)f_Y(y)\mathrm dy,\quad\text{$X$ 连续}
\end{gather*}
\end{theorem}

\begin{theorem}[贝叶斯公式]
设 $X,Y$ 是随机变量,则有:
\begin{gather*}
p_{X\vert Y}(x\vert y)=\frac{p_{X,Y}(x,y)}{p_Y(y)}=\frac{p_{Y\vert X}(y\vert x)p_X(x)}{\sum_x p_{Y\vert X}(y\vert x)p_X(x)},\quad\text{$X,Y$ 离散}\\
f_{X\vert Y}(x\vert y)=\frac{f_{X,Y}(x,y)}{f_Y(y)}=\frac{f_{Y\vert X}(y\vert x)f_X(x)}{\int_{-\infty}^{\infty}f_{Y\vert X}(y\vert x)f_X(x)\mathrm dx},\quad\text{$X,Y$ 连续}
\end{gather*}
\end{theorem}

\begin{definition}[条件期望]
设 $X,Y$ 是随机变量,则给定 $Y=y$ 下 $X$ 的条件期望为:
\[
\E[X\vert Y=y]=\begin{dcases}
    \sum_xxp_{X\vert Y}(x\vert y),&\text{$X$ 离散}\\
    \int_{-\infty}^{\infty}xf_{X\vert Y}(x\vert y)\mathrm dx,&\text{$X$ 连续}
\end{dcases}
\]
对于随机变量的函数 $g(X)$,有:
\[
\E[g(X)\vert Y=y]=\begin{dcases}
    \sum_xg(x)p_{X\vert Y}(x\vert y),&\text{$X$ 离散}\\
    \int_{-\infty}^{\infty}g(x)f_{X\vert Y}(x\vert y)\mathrm dx,&\text{$X$ 连续}
\end{dcases}
\]
\end{definition}

\begin{note}
$\E[X|Y=y]$ 是一个数,其值依赖于 $y$,因此 $\E[X|Y]$ 是关于随机变量 $Y$ 的函数,从而也是一个随机变量。
\end{note}

\begin{theorem}[全期望公式]
设 $X,Y$ 是随机变量(离散或连续)且 $X$ 期望存在,则:
\[
\E X=\E[\E[X\vert Y]]=\begin{dcases}
    \sum_y \E[X\vert Y=y]p_Y(y),&\text{$Y$ 离散}\\
    \int_{-\infty}^{\infty} \E[X\vert Y=y]f_Y(y)\mathrm dy,&\text{$Y$ 连续}
\end{dcases}
\]
\end{theorem}
\begin{proof}
仅对离散情形证明,连续情形类似。
\begin{align*}
\E X&=\sum_x xp_X(x)\\
&=\sum_x x\sum_yp_{X|Y}(x\vert y)p_Y(y)\\
&=\sum_yp_Y(y)\sum_xxp_{X|Y}(x\vert y)\\
&=\sum_y \E[X\vert Y=y]p_Y(y)
\end{align*}
其中第二行应用了全概率公式。
\end{proof}

\begin{remark}
全期望公式常常是“反过来”使用的:当 $\mathbb EX$ 不好计算时,引入 $Y$ 转而计算 $\mathbb E[\mathbb E[X\vert Y]]$.
\end{remark}

\begin{definition}[条件方差]
设 $X,Y$ 是随机变量,则给定 $Y=y$ 下 $X$ 的条件方差为:
\[
\var(X\vert Y=y)=\E[(X-\E[X\vert Y=y])^2\vert Y=y]
\]
\end{definition}

\begin{note}
$\var(X\vert Y=y)$ 是一个数,其值依赖于 $y$,因此 $\var(X\vert Y)$ 是关于随机变量 $Y$ 的函数,从而也是一个随机变量,并且有:
\[
\var(X\vert Y)=\E[(X-\E[X\vert Y])^2\vert Y]=\E[X^2\vert Y]-(\E[X\vert Y])^2
\]
\end{note}

\begin{theorem}[全方差公式]
设 $X,Y$ 是随机变量且 $X$ 方差存在,则:
\[
\var(X)=\E[\var(X\vert Y)]+\var(\E[X\vert Y])
\]
\end{theorem}
\begin{proof}
\begin{align*}
\var(X)&=\E[X^2]-(\E X)^2\\
&=\E[\E[X^2\vert Y]]-(\E[\E[X\vert Y]])^2\\
&=\E[\E[X^2\vert Y]]-\E[(\E[X\vert Y])^2]+\E[(\E[X\vert Y])^2]-(\E[\E[X\vert Y]])^2\\
&=\E[\var(X\vert Y)]+\var(\E[X\vert Y])
\end{align*}
\end{proof}


\subsection{独立性}

\begin{definition}[独立]
设 $X,Y$ 是随机变量,称 $X$ 与 $Y$ 独立,若:
\begin{gather*}
p_{X,Y}(x,y)=p_X(x)p_Y(y),\quad\forall x,y,\quad\text{$X,Y$ 离散}\\
f_{X,Y}(x,y)=f_X(x)f_Y(y),\quad\forall x,y,\quad\text{$X,Y$ 连续}
\end{gather*}
\end{definition}
\begin{com}
离散情形下,若 $p_Y(y)>0$,则 $X$ 与 $Y$ 独立等价于:
\[p_{X\vert Y}(x\vert y)=p_X(x),\;\forall y\]
直观上,这说明 $Y$ 的取值不会给 $X$ 的取值带来信息。连续情形同理。
\end{com}

\begin{definition}[条件独立]
设 $X,Y,Z$ 是随机变量,给定 $Z=z$ 下(设 $p_Z(z)>0$ 或 $f_Z(z)>0$),称随机变量 $X$ 与 $Y$ 条件独立,若:
\begin{gather*}
p_{X,Y\vert Z}(x,y\vert z)=p_{X\vert Z}(x\vert z)p_{Y\vert Z}(y\vert z),\quad\forall x,y,\quad\text{$X,Y$ 离散}\\
f_{X,Y\vert Z}(x,y\vert z)=f_{X\vert Z}(x\vert z)f_{Y\vert Z}(y\vert z),\quad\forall x,y,\quad\text{$X,Y$ 连续}
\end{gather*}
\end{definition}

\begin{theorem}
若随机变量 $X,Y$ 相互独立,则:
\[
\E[XY]=\E X\E Y
\]
进一步地,对任意函数 $g,h$,有:
\[
\E[g(X)h(Y)]=\E[g(X)]\E[h(Y)]
\]
\end{theorem}
\begin{proof}
仅对离散情形证明,连续情形类似。
\begin{align*}
\E[XY]&=\sum_x\sum_y xyp_{X,Y}(x,y)\\
&=\sum_x\sum_y xyp_X(x)p_Y(y)\\
&=\sum_x xp_X(x)\sum_y yp_Y(y)\\
&=\E X\E Y
\end{align*}
第二个式子类似可证。
\end{proof}

\begin{corollary}
若随机变量 $X,Y$ 独立,则对任意函数 $g,h$,有 $g(X)$ 与 $h(Y)$ 独立。
\end{corollary}

\begin{theorem}
若随机变量 $X,Y$ 相互独立,则:
\[
\var(X+Y)=\var(X)+\var(Y)
\]
\end{theorem}
\begin{proof}
令 $\tilde X=X-\E X,\,\tilde Y=Y-\E Y$,由于方差在加减常数后保持不变,所以:
\begin{align*}
\var(X+Y)&=\var(\tilde X+\tilde Y)\\
&=\E[(\tilde X+\tilde Y)^2]\\
&=\E[{\tilde X}^2+2\tilde X\tilde Y+{\tilde Y}^2]\\
&=\E[{\tilde X}^2]+2\E[\tilde X\tilde Y]+\E[{\tilde Y}^2]\\
&=\var X+\var Y
\end{align*}
其中利用了 $\E[\tilde X\tilde Y]=\E[\tilde X]\E[\tilde Y]=0$.
\end{proof}

\begin{definition}[多个随机变量独立性]
称随机变量 $X,Y,Z$ 相互独立,若:
\begin{gather*}
p_{X,Y,Z}(x,y,z)=p_X(x)p_Y(y)p_Z(z),\quad\forall x,y,z,\quad\text{$X,Y,Z$ 离散}\\
f_{X,Y,Z}(x,y,z)=f_X(x)f_Y(y)f_Z(z),\quad\forall x,y,z,\quad\text{$X,Y,Z$ 连续}
\end{gather*}
\end{definition}

\begin{theorem}[多个独立随机变量和的方差]
设 $X_1,X_2,\ldots,X_n$ 为相互独立的随机变量,则:
\[
\var(X_1+X_2+\cdots+X_n)=\var(X_1)+\var(X_2)+\cdots+\var(X_n)
\]
\end{theorem}


% \subsection{正态随机变量}

% \begin{definition}[正态随机变量/高斯随机变量]
% 称连续随机变量 $X$ 为正态或高斯的,若其 PDF 为:
% \[
% f_X(x)=\frac{1}{\sqrt{2\pi}\sigma}e^{-\frac{(x-\mu)^2}{2\sigma^2}}
% \]
% 其中 $\mu,\sigma$ 是参数。
% \end{definition}

% \begin{theorem}[正态分布的均值和方差]
% 设 $X$ 为
% \end{theorem}
 \newpage
\section{随机变量的深入内容}

\subsection{随机变量的函数的分布}

\begin{theorem}[随机变量的函数]
设 $X$ 是离散随机变量,则 $Y=g(X)$ 的 PMF 为:
\[p_Y(y)=\sum_{\{x|y=g(x)\}}p_X(x)\]
设 $X$ 是连续随机变量,则 $Y=g(X)$ 的 CDF 为:
\[F_Y(y)=\Pb(Y\leqslant y)=\Pb(g(X)\leqslant y)=\int\limits_{\{x|g(x)\leqslant y\}}f_X(x)\mathrm dx\]
进而 PDF 为:
\[f_Y(y)=\frac{\mathrm d}{\mathrm dy}F_Y(y)\]
\end{theorem}

\begin{theorem}[线性函数情形]
\label{thm:rv-linear}
设 $X$ 是连续随机变量,$a,b\in\R$ 且 $a\neq0$,设 $Y=aX+b$,则:
\[f_Y(y)=\frac{1}{|a|}f_X\left(\frac{y-b}{a}\right)\]
\end{theorem}
\begin{proof}
先求 $Y$ 的 CDF:
\[F_Y(y)=\Pb(Y\leqslant y)=\Pb(aX+b\leqslant y)=\begin{cases}\Pb\left(X\leqslant \frac{y-b}{a}\right)=F_X\left(\frac{y-b}{a}\right),&a>0\\\Pb\left(X\geqslant \frac{y-b}{a}\right)=1-F_X\left(\frac{y-b}{a}\right),&a<0\end{cases}\]
然后求导得到 $Y$ 的 PDF:
\[f_Y(y)=\frac{\mathrm dF_Y(y)}{\mathrm dy}=\begin{cases}\frac{1}{a}f_X\left(\frac{y-b}{a}\right),&a>0\\-\frac{1}{a}f_X\left(\frac{y-b}{a}\right),&a<0\end{cases}\quad=\quad\frac{1}{|a|}f_X\left(\frac{y-b}{a}\right)\]
\end{proof}

\begin{example}[正态分布的线性变换仍然是正态分布]
设 $X\sim N(\mu,\sigma^2)$,$a,b\in\mathbb R$ 且 $a\neq0$,$Y=aX+b$,则根据定理 \ref{thm:rv-linear},有:
\begin{align*}
f_{Y}(y)&=\frac{1}{|a|}f_X\left(\frac{y-b}{a}\right)\\
&=\frac{1}{\sqrt{2\pi}\sigma|a|}\exp\left({-\frac{\left(\frac{y-b}{a}-\mu\right)^2}{2\sigma^2}}\right)\\
&=\frac{1}{\sqrt{2\pi}\sigma|a|}\exp\left({-\frac{(y-a\mu-b)^2}{2a^2\sigma^2}}\right)
\end{align*}
所以,$Y=aX+b\sim N(a\sigma+b,a^2\sigma^2)$.
\end{example}

\begin{theorem}[单调函数情形]
设 $X$ 是连续随机变量,$g$ 是严格单调的可逆函数,且其反函数 $h$ 可微,则 $Y=g(X)$ 在其支撑集 $\{y\mid f_Y(y)>0\}$ 上的 PDF 是:
\[f_Y(y)=f_X(h(y))\left|\frac{\mathrm d}{\mathrm dy}h(y)\right|\]
\end{theorem}
\begin{proof}
先求 $Y$ 的 CDF:
\[F_Y(y)=\Pb(Y\leqslant y)=\begin{cases}\Pb(X\leqslant h(y))=F_X(h(y)),&g\;\text{单调递增}\\\Pb(X\geqslant h(y))=1-F_X(h(y)),&g\;\text{单调递减}\end{cases}\]
于是求导得:
\[f_Y(y)=\begin{cases}f_X(h(y))h'(y),&g\;\text{单调递增}\\-f_X(h(y))h'(y),&g\;\text{单调递减}\end{cases}\quad =\quad f_X(h(y))\left|\frac{\mathrm d}{\mathrm dy}h(y)\right|\]
\end{proof}

\begin{theorem}[随机变量的二元函数]
设 $X,Y$ 是离散随机变量,则 $Z=g(X,Y)$ 的 PMF 为:
\[p_Z(z)=\sum_{\{(x,y)|z=g(x,y)\}}p_{X,Y}(x,y)\]
设 $X,Y$ 是连续随机变量,则 $Z=g(X,Y)$ 的 CDF 为:
\[F_Z(z)=\Pb(Z\leqslant z)=\Pb(g(X,Y)\leqslant z)=\iint\limits_{\{(x,y)|g(x,y)\leqslant z\}}f_{X,Y}(x,y)\mathrm dx\mathrm dy\]
进而 PDF 为:
\[f_Z(z)=\frac{\mathrm d}{\mathrm dz}F_Z(z)\]
\end{theorem}

\begin{example}[瑞利分布]
设 $X,Y\sim N(0,\sigma^2)$ 且相互独立,$R=\sqrt{X^2+Y^2}$,称 $R$ 服从瑞利分布。
首先计算 CDF:
\begin{align*}
F_R(r)&=\Pb(R\leqslant r)=\Pb(X^2+Y^2\leqslant r^2)\\
&=\iint\limits_{\{(x,y)|x^2+y^2\leqslant r^2\}}p_{X,Y}(x,y)\mathrm dx\mathrm dy\\
&=\iint\limits_{x^2+y^2\leqslant r^2}\frac{1}{2\pi\sigma^2}e^{-\frac{x^2+y^2}{2\sigma^2}}\mathrm dx\mathrm dy\\
&=\frac{1}{2\pi\sigma^2}\int_0^{2\pi}\mathrm d\theta\int_0^{r}e^{-\frac{\rho^2}{2\sigma^2}}\rho\mathrm d\rho\\
&=\left.e^{-\frac{\rho^2}{2\sigma^2}}\right|_{r}^0=1-e^{-\frac{r^2}{2\sigma^2}}
\end{align*}
故瑞利分布的 PDF 为:
\[f_R(r)=\frac{\mathrm dF_R(r)}{\mathrm dr}=\frac{r}{\sigma^2}e^{-\frac{r^2}{2\sigma^2}}\]
\end{example}

\begin{theorem}[极值分布]
设 $X,Y$ 是独立的随机变量,$M=\max(X,Y),N=\min(X,Y)$,则:
\[F_M(z)=F_X(z)F_Y(z),\qquad F_N(z)=1-(1-F_X(z))(1-F_Y(z))\]
\end{theorem}
\begin{proof}
\begin{align*}
F_M(z)&=\Pb(M\leqslant z)=\Pb(X\leqslant z, Y\leqslant z)=\Pb(X\leqslant z)\Pb(Y\leqslant z)=F_X(z)F_Y(z)\\
F_N(z)&=\Pb(N\leqslant z)=1-\Pb(N>z)=1-\Pb(X>z, Y>z)\\
&=1-\Pb(X>z)\Pb(Y>z)=1-(1-F_X(z))(1-F_Y(z))
\end{align*}
\end{proof}

\begin{corollary}
设 $X_1,X_2,\cdots,X_n$ 是 $n$ 个相互独立的随机变量,$M=\max\limits_{1\leqslant i\leqslant n}X_i$,$N=\min\limits_{1\leqslant i\leqslant n}X_i$,则:
\[F_M(z)=\prod_{i=1}^nF_{X_i}(z),\qquad F_N(z)=1-\prod_{i=1}^n(1-F_{X_i}(z))\]
\end{corollary}

\begin{theorem}[独立随机变量之和——卷积]
\label{thm:sum-ind-rv}
设 $X,Y$ 是独立的离散随机变量,$Z=X+Y$,则:
\[p_Z(z)=\sum_x p_X(x)p_Y(z-x)\]
设 $X,Y$ 是独立的连续随机变量,$Z=X+Y$,则:
\[f_Z(z)=\int_{-\infty}^{+\infty} f_X(x)f_Y(z-x)\mathrm dx\]
称上述两个式子为卷积(convolution)。
\end{theorem}

\begin{example}[独立正态随机变量之和仍服从正态分布]
设 $X\sim N(\mu_x,\sigma_x^2)$,$Y\sim N(\mu_y,\sigma_y^2)$ 且相互独立,$Z=X+Y$,则根据定理 \ref{thm:sum-ind-rv},有:
\[
f_Z(z)=\int_{-\infty}^{+\infty}\frac{1}{2\pi\sigma_x\sigma_y}e^{-\frac{(x-\mu_x)^2}{2\sigma_x^2}-\frac{(z-x-\mu_y)^2}{2\sigma_y^2}}\mathrm dx=\int_{-\infty}^{+\infty}\frac{1}{2\pi\sigma_x\sigma_y}e^{-\frac{1}{2}\left[\frac{u^2}{\sigma_x^2}+\frac{(v-u)^2}{\sigma_y^2}\right]}\mathrm du
\]
其中做了代换 $u=x-\mu_x,\,v=z-\mu_x-\mu_y$. 由于:
\[
\frac{u^2}{\sigma_x^2}+\frac{(v-u)^2}{\sigma_y^2}=\frac{u^2}{\sigma_x^2}+\frac{u^2}{\sigma_y^2}+\frac{v^2}{\sigma_y^2}-\frac{2uv}{\sigma_y^2}=\left(\frac{\sqrt{\sigma_x^2+\sigma_y^2}}{\sigma_x\sigma_y}u-\frac{\sigma_x v}{\sigma_y\sqrt{\sigma_x^2+\sigma_y^2}}\right)^2+\frac{v^2}{\sigma_x^2+\sigma_y^2}
\]
令 $t=\frac{\sqrt{\sigma_x^2+\sigma_y^2}}{\sigma_x\sigma_y}u-\frac{\sigma_x v}{\sigma_y\sqrt{\sigma_x^2+\sigma_y^2}}$,则:
\[
f_Z(z)=\frac{1}{2\pi\sigma_x\sigma_y}\frac{\sigma_x\sigma_y}{\sqrt{\sigma_x^2+\sigma_y^2}}e^{\frac{v^2}{\sigma_x^2+\sigma_y^2}}\int_{-\infty}^{+\infty}e^{-\frac{1}{2}t^2}\mathrm dt=\frac{1}{\sqrt{2\pi}\sqrt{\sigma_x^2+\sigma_y^2}}e^{\frac{(z-\mu_x-\mu_y)^2}{\sigma_x^2+\sigma_y^2}}
\]
所以 $Z\sim N(\mu_x+\mu_y,\sigma_x^2+\sigma_y^2)$.
\end{example}

\begin{example}[$n$ 个独立正态随机变量之和]
\label{ex:ind-normal-sum}
设 $X_i\sim N(\mu_i,\sigma_i^2),\,i=1,2,\ldots,n$ 且相互独立,则:
\[\sum\limits_{i=1}^nX_i\sim N\left(\sum\limits_{i=1}^n\mu_i,\sum\limits_{i=1}^n\sigma_i^2\right)\]
\end{example}

\begin{theorem}[随机变量的多元函数(可逆函数情形)]
设 $\mathbf{X}=(X_1,X_2,\cdots,X_n)$ 是一个随机向量,$T$ 是 $\R ^n$ 上的一可逆映射,$\mathbf Y=T(\mathbf X)$,则 $\mathbf Y$ 的 PDF 为:
\[f_{\mathbf Y}(\mathbf y)=f_{\mathbf X}(T^{-1}(\mathbf y))|J|\]
其中,$J$ 表示 $T^{-1}:\mathbf y\mapsto \mathbf x$,即 $\mathbf x=T^{-1}(\mathbf y)$ 的雅各比行列式:
\[
J=\begin{vmatrix}
\frac{\partial x_1}{\partial y_1}&\frac{\partial x_1}{\partial y_2}&\cdots&\frac{\partial x_1}{\partial y_n}\\
\frac{\partial x_2}{\partial y_1}&\frac{\partial x_2}{\partial y_2}&\cdots&\frac{\partial x_2}{\partial y_n}\\
\vdots&\vdots&\vdots&\vdots\\
\frac{\partial x_n}{\partial y_1}&\frac{\partial x_n}{\partial y_2}&\cdots&\frac{\partial x_n}{\partial y_n}\end{vmatrix}
\]
\end{theorem}
\begin{proof}
设 $D\subset \R^n$ 是一个性质好的集合,则:
\begin{align*}
\Pb(\mathbf Y\in D)&=\Pb(T(\mathbf X)\in D)\\
&=\Pb(\mathbf X\in T^{-1}(D))&\text{两边同时施以}\;T^{-1}\\
&=\int_{T^{-1}(D)}f_{\mathbf X}(\mathbf x)\mathrm d\mathbf x\\
&=\int_{D}f_{\mathbf X}(T^{-1}(y))|J|\mathrm d\mathbf y&\text{变量代换}\;\mathbf y=T(\mathbf x)
\end{align*}
又:
\[\Pb(\mathbf Y\in D)=\int_Df_\mathbf Y(\mathbf y)\mathrm d\mathbf y\]
根据 $D$ 一定的任意性,可知:
\[f_{\mathbf Y}(\mathbf y)=f_{\mathbf X}(T^{-1}(\mathbf y))|J|\]
\end{proof}


\subsection{协方差与相关系数}

\begin{figure}[H]
    \centering
    \includegraphics[width=0.9\linewidth]{figs/随机变量的独立性与相关性.png}
    \caption{随机变量的独立性与相关性示意图}
    \label{fig:independence-correlation}
\end{figure}

\begin{definition}[协方差与相关系数]
设 $X,Y$ 是两个随机变量,定义协方差与相关系数分别为:
\[
\cov(X,Y)=\E [(X-\E  X)(Y-\E  Y)],\quad\rho(X,Y)=\frac{\cov(X,Y)}{\sqrt{\var X\var Y}}
\]
当 $\cov(X,Y)=\rho(X,Y)=0$ 时,称 $X$ 和 $Y$ 不相关。
\end{definition}

\begin{property}
从定义出发容易证明以下性质:
\begin{itemize}
    \item $\cov(X,X)=\var(X)$
    \item $\cov(X,aY+b)=a\cdot\cov(X,Y)$ 
    \item $\cov(X,Y+Z)=\cov(X,Y)+\cov(X,Z)$
\end{itemize}
\end{property}

\begin{property}
$|\rho(X,Y)|\leq 1$.
\end{property}
\begin{proof}
对 $\forall t\in \R$,由于 $\var(Y-tX)=t^2\var X-2t\cov(X,Y)+\var Y\geq 0$,所以:
\[\Delta=4\cov^2(X,Y)-4\var X\var Y\leq 0\]
即有 $|\rho(X,Y)|\leq 1$.
\end{proof}

\begin{theorem}
设 $X,Y$ 是两个随机变量,则:
\[\cov(X,Y)=\E[XY]-\E X\E Y\]
\end{theorem}
\begin{proof}
\begin{align*}
\cov(X,Y)&=\E[(X-\E X)(Y-\E Y)]\\
&=\E[XY-\E X\cdot Y-X\cdot \E Y+\E X\E Y]\\
&=\E[XY]-\E X\cdot\E Y-\E X\cdot\E Y+\E X\cdot\E Y\\
&=\E[XY]-\E X\E Y
\end{align*}
\end{proof}

\begin{corollary}[独立与相关]
设 $X,Y$ 是两个随机变量,若 $X,Y$ 独立,则 $X,Y$ 不相关。
\end{corollary}
\begin{note}
反之不成立,不相关不能推出独立。
\end{note}

\begin{example}[不相关且不独立]
设随机变量 $X,Y$ 以 $1/4$ 的概率取 $(1,0),(0,1),(-1,0),(0,-1)$,则 $\E X=\E Y=\E[XY]=0$,于是 $\cov(X,Y)=\E[XY]-\E X\E Y=0$,故 $X,Y$ 不相关。但是 $p_{X,Y}(1,0)=1/4\neq p_X(1)p_Y(0)=1/4\cdot 1/2=1/8$,故二者不独立。直观上,$X$ 取非零值就要求 $Y$ 取零值,因此不独立。
\end{example}

\begin{theorem}[随机变量和的方差]
设随机变量 $X,Y$ 有有限的方差,则:
\[\var(X+Y)=\var X+2\cov(X,Y)+\var Y\]
更一般的,设随机变量 $X_1,X_2,\ldots,X_n$ 有有限的方差,则:
\[
\var\left(\sum_{i=1}^n X_i\right)=\sum_{i=1}^n\var(X_i)+\sum_{\{(i,j)\mid i\neq j\}}\cov(X_i,X_j)
\]
\end{theorem}
\begin{proof}
设 $\tilde X_i=X_i-\E[X_i]$,则:
\begin{align*}
\var\left(\sum_{i=1}^n X_i\right)&=\E\left[\left(\sum_{i=1}^n{\tilde X_i}^2\right)^2\right]\\
&=\E\left[\sum_{i=1}^n\sum_{j=1}^n\tilde X_i\tilde X_j\right]\\
&=\sum_{i=1}^n\sum_{j=1}^n\E\left[\tilde X_i\tilde X_j\right]\\
&=\sum_{i=1}^n\E[{\tilde X_i}^2]+\sum_{\{(i,j)\mid i\neq j\}}^n\E\left[\tilde X_i\tilde X_j\right]\\
&=\sum_{i=1}^n\var(X_i)+\sum_{\{(i,j)\mid i\neq j\}}\cov(X_i,X_j)
\end{align*}
\end{proof}

\begin{corollary}
设随机变量 $X,Y$ 有有限的方差,$a,b$ 为常数,则:
\[
\var(aX+bY)=a^2\var X+2ab\cov(X,Y)+b^2\var Y
\]
或写作矩阵形式:
\[
\var(aX+bY)=\begin{bmatrix}a&b\end{bmatrix}\begin{bmatrix}\var X&\cov(X,Y)\\\cov(Y,X)&\var Y\end{bmatrix}\begin{bmatrix}a\\b\end{bmatrix}
\]
\end{corollary}

% \begin{property}
% $\var(X)=0\implies \mathbb P(X=\E  X)=1$
% \end{property}
% \begin{proof}
% 根据切比雪夫不等式(见后文),$\mathbb P\left(|X-\E  X|\geq \frac{1}{n}\right)\leq n^2\var(X)=0$ 对 $\forall n$ 都成立,故 $\mathbb P(X=\E  X)=1$.
% \end{proof}

\subsection{再论条件期望与条件方差}
\label{sec:cond-e-cond-var}

\begin{definition}[条件期望作为估计量]
设 $X,Y$ 是随机变量,若将 $Y$ 视作能提供关于 $X$ 的信息的观测值,则可将条件期望视为给定 $Y$ 下对 $X$ 的估计,记作:
\[
\hat X=\E[X\vert Y]
\]
注意 $\hat X$ 是随机变量 $Y$ 的函数。进一步地,记估计误差为:
\[
\tilde X=\hat X-X
\]
则 $\tilde X$ 是随机变量 $X,Y$ 的二元函数。
\end{definition}

\begin{property}
根据全期望公式易知,条件期望是无偏估计,即 $\E\hat X=\E X$,或 $\E\tilde X=0$.
\end{property}

\begin{property}
$\E[\tilde X\vert Y]=0$,即对任意 $y$,都有 $\E[\tilde X\vert Y=y]=0$.
\end{property}
\begin{proof}
\[
\E[\tilde X\vert Y]=\E[(\hat X-X)\vert Y]=\E[\hat X\vert Y]-\E[X\vert Y]=\hat X-\hat X=0
\]
\end{proof}

\begin{property}
估计量 $\hat X$ 与估计误差 $\tilde X$ 不相关。
\end{property}
\begin{proof}
\[
\E[\hat X\tilde X]=\E[\E[\hat X\tilde X\vert Y]]=\E[\hat X\E[\tilde X\vert Y]]=0=\E\hat X\E\tilde X
\]
\end{proof}

\begin{property}
$\var(X)=\var(\hat X)+\var(\tilde X)$.
\end{property}
\begin{proof}
由于 $\hat X$ 与 $\tilde X$ 不相关,故 $\cov(\hat X,\tilde X)=0$,又 $X=\hat X+\tilde X$,故:
\[
\var(X)=\var(\hat X)+2\cov(\hat X,\tilde X)+\var(\tilde X)=\var(\hat X)+\var(\tilde X)
\]
\end{proof}


\subsection{矩母函数}

\begin{definition}[矩母函数]
设 $X$ 是一个随机变量,定义其矩母函数为:
\[
M_X(s)=\E[e^{sX}]=\begin{dcases}
    \sum_x e^{sx}p_X(x),&\text{$X$ 离散}\\
    \int_{-\infty}^{\infty} e^{sx}f_X(x)\mathrm dx,&\text{$X$ 连续}
\end{dcases}
\]
其定义域为使得 $\E[e^{sX}]$ 存在的 $s$. 在上下文清晰时可简记作 $M(s)$.
\end{definition}

\begin{theorem}[矩母函数计算矩]
设随机变量 $X$ 的矩母函数为 $M(s)$,则:
\[
\left.\frac{\mathrm d^n}{\mathrm ds^n}M(s)\right|_{s=0}=\E[X^n]
\]
\end{theorem}
\begin{proof}
假设积分(期望)与微分可交换,则:
\[
\frac{\mathrm d^n}{\mathrm ds^n}M(s)=\frac{\mathrm d^n}{\mathrm ds^n}\E[e^{sX}]=\E\left[\frac{\mathrm d^n}{\mathrm ds^n}e^{sX}\right]=\E[X^ne^{sX}]
\]
代入 $s=0$ 得:
\[
\left.\frac{\mathrm d^n}{\mathrm ds^n}M(s)\right|_{s=0}=\E[X^n]
\]
\end{proof}

\begin{theorem}[矩母函数与分布]
若随机变量 $X$ 的矩母函数 $M_X(s)$ 满足:存在一个正数 $a$,使得对 $\forall s\in[-a,a]$,$M_X(s)$ 都是有限的,则矩母函数 $M_X(s)$ 唯一决定 $X$ 的分布函数。
\end{theorem}

\begin{theorem}[独立随机变量之和]
设 $X,Y$ 是独立随机变量,$Z=X+Y$,则:
\[M_Z(s)=M_X(s)M_Y(s)\]
\end{theorem}
\begin{proof}
\[
M_Z(s)=\E[e^{sZ}]=\E[e^{s(X+Y)}]=\E[e^{sX}e^{sY}]=\E[e^{sX}]\E[e^{sY}]=M_X(s)M_Y(s)
\]
\end{proof}
\begin{corollary}
设 $X_1,X_2,\ldots,X_n$ 是独立随机变量,$Z=X_1+X_2+\cdots+X_n$,则:
\[
M_Z(s)=M_{X_1}(s)M_{X_2}(s)\cdots M_{X_n}(s)
\]
\end{corollary}

\begin{definition}[多元矩母函数]
设 $X_1,X_2,\ldots,X_n$ 是随机变量,定义它们的多元矩母函数为:
\[
M_{X_1,X_2,\ldots,X_n}(s_1,s_2,\ldots,s_n)=\E[e^{s_1X_1+s_2X_2+\cdots+s_nX_n}]
\]
\end{definition}


\subsection{随机个随机变量之和}

\begin{theorem}[随机个随机变量之和的期望、方差与矩母函数]
设 $N$ 是取正整数值的随机变量,$X_1,X_2,\ldots$ 是独立同分布的随机变量,且 $N,X_1,X_2,\ldots$ 相互独立(即这些随机变量的任意有限子集都是独立的)。设 $Y=X_1+X_2+\cdots+X_N$,则:
\begin{gather*}
    \E Y=\E N\E X\\
    \var(Y)=\E N\var(X)+(\E X)^2\var(N)\\
    M_Y(s)=\sum_{n=1}^\infty (M_X(s))^n p_N(n)
\end{gather*}
其中 $\E X,\var(X),M_X(s)$ 表示各 $X_i$ 的期望、方差和矩母函数。
\end{theorem}
\begin{proof}
给定正整数 $n$,随机变量 $X_1+X_2+\cdots+X_n$ 与 $N$ 独立,故与事件 $\{N=n\}$ 独立,故:
\begin{align*}
\E[Y\vert N=n]&=\E[X_1+X_2+\cdots+X_N\vert N=n]\\
&=\E[X_1+X_2+\cdots+X_n\vert N=n]\\
&=\E[X_1+X_2+\cdots+X_n]\\
&=n\E X
\end{align*}
这对于任意非负整数 $n$ 都成立,因此:
\[
\E[Y\vert N]=N\E X
\]
于是根据全期望公式,有:
\[
\E Y=\E[\E[Y\vert N]]=\E[N\E X]=\E N\E X
\]
类似地,给定正整数 $n$,有:
\begin{align*}
\var(Y\vert N=n)&=\var(X_1+X_2+\cdots+X_N\vert N=n)\\
&=\var(X_1+X_2+\cdots+X_n\vert N=n)\\
&=\var(X_1+X_2+\cdots+X_n)\\
&=n\var(X)
\end{align*}
这对任意正整数 $n$ 都成立,因此:
\[
\var(Y\vert N)=N\var(X)
\]
于是根据全方差公式,有:
\begin{align*}
\var(Y)&=\E[\var(Y\vert N)]+\var(\E[Y\vert N])\\
&=\E[N\var(X)]+\var(N\E X)\\
&=\E N\var(X)+(\E X)^2\var(N)
\end{align*}
类似地,给定正整数 $n$,有:
\begin{align*}
\E[e^{sY}\vert N=n]&=\E[e^{s(X_1+X_2+\cdots+X_N}\vert N=n]\\
&=\E[e^{s(X_1+X_2+\cdots+X_n}\vert N=n]\\
&=\E[e^{s(X_1+X_2+\cdots+X_n}]\\
&=\E[e^{sX_1}]\E[e^{sX_2}]\cdots\E[e^{sX_n}\\
&=(M_{X}(s))^n
\end{align*}
上式对任意正整数 $n$ 都成立,因此:
\[
\E[e^{sY}\vert N]=(M_X(s))^N
\]
于是根据全期望公式,有:
\[
M_Y(s)=\E[e^{sY}]=\E[\E[e^{sY}\vert N]]=\E[(M_X(s))^N]=\sum_{n=1}^\infty (M_X(s))^n p_N(n)
\]
\end{proof}

\begin{com}
对比 $M_Y(s)$ 与 $M_N(s)$:
\begin{gather*}
    M_Y(s)=\sum_{n=1}^\infty (M_X(s))^n p_N(n)\\
    M_N(s)=\sum_{n=1}^\infty (e^s)^np_N(n)
\end{gather*}
可以看见 $M_Y(s)$ 就是将 $M_N(s)$ 中的函数 $e^s$ 替换为 $X_i$ 的矩母函数 $M_X(s)$.
\end{com}
 \newpage
\section{极限理论}

给定一列独立同分布随机变量 $X_1,X_2,\ldots$,定义前 $n$ 项和 $S_n=X_1+X_2+\cdots+X_n$,本章的极限理论研究 $S_n$ 及其相关变量在 $n\to\infty$ 时的极限性质。


\subsection{马尔可夫不等式与切比雪夫不等式}

\begin{theorem}[马尔可夫不等式]
设随机变量 $X$ \textbf{只取非负值},则对任意 $a>0$,有:
\[
\mathbb P(X\geq a)\leq \frac{\mathbb EX}{a}
\]
\end{theorem}
\begin{com}
粗略来讲,马尔可夫不等式指出,一个非负随机变量如果均值很小,那么该随机变量取大值的概率也很小。
\end{com}
\begin{proof}
这里假设 $X$ 是连续随机变量,离散类似。
\[
\mathbb EX=\int_0^{+\infty}xf_X(x)\mathrm dx\geq \int_a^{+\infty}xf_X(x)\mathrm dx\geq a\int_a^{+\infty}f_X(x)\mathrm dx=a\cdot\mathbb P(X\geq a)
\]
\end{proof}

\begin{example}
\label{ex:markov}
设 $X\sim U(0,4)$,则 $\E X=2$,由马尔可夫不等式可得:
\[
\Pb(X\geq2)\leq\frac{2}{2}=1,\quad\Pb(X\geq 3)\leq\frac{2}{3},\quad\Pb(X\geq 4)\leq\frac{2}{4}=\frac{1}{2}
\]
而真实概率是:
\[
\Pb(X\geq2)=\frac{1}{2},\quad\Pb(X\geq 3)=\frac{1}{4},\quad\Pb(X\geq 4)=0
\]
可见由马尔可夫不等式给出的上界非常的粗糙。
\end{example}

\begin{theorem}[切比雪夫不等式]
设随机变量 $X$ 的均值为 $\mu$,方差为 $\sigma^2$,则对任意 $c>0$,有:
\[
\mathbb P(|X-\mu|\geq c)\leq \frac{\sigma^2}{c^2}
\]
\end{theorem}
\begin{com}
粗略来讲,切比雪夫不等式指出,如果一个随机变量的方差非常小,那么该随机变量取远离均值的概率也非常小。
\end{com}
\begin{proof}
利用马尔可夫不等式,
\[
\mathbb P(|X-\mu|\geq c)=\mathbb P((X-\mu)^2\geq c^2)\leq \frac{\mathbb E[(X-\mu)^2]}{c^2}=\frac{\sigma^2}{c^2}
\]
\end{proof}

相比马尔可夫不等式,切比雪夫不等式利用了随机变量的方差的信息,因此会更准确。不过均值和方差也仅仅是粗略描述了随机变量的性质,因此它给出的上界可能依旧距离精确概率较远。

\begin{example}
仍然考虑例 \ref{ex:markov},即 $X\sim U(0,4),\,\E X=2,\,\var X=\frac{4}{3}$,则根据切比雪夫不等式有:
\[
\Pb(|X-2|\geq1)\leq\frac{4}{3}
\]
由于概率一定小于等于 1,所以这个不等式并没有带来任何信息。
\end{example}

\begin{example}
设 $X\sim E(1)$,则 $\E X=\var X=1$,对任意 $c>2$,应用切比雪夫不等式可得:
\[
\Pb(X\geq c)=\Pb(X-1\geq c-1)=\Pb(|X-1|\geq c-1)\leq\frac{1}{(c-1)^2}
\]
而真实概率是 $\Pb(X\geq c)=e^{-c}$,可见切比雪夫不等式给出的上界比较保守。
\end{example}


\subsection{弱大数定律}

\begin{definition}[依概率收敛]
设 $X_1,X_2,\ldots$ 是随机变量序列,$a\in\mathbb R$,若对任意 $\epsilon>0$,都有:
\[
\lim_{n\to\infty}\Pb(|X_n-a|\geq\epsilon)=0
\]
则称 $\{X_n\}$ 依概率收敛到 $a$.
\end{definition}

\begin{com}
用 $\epsilon$-$N$ 语言描述上述定义中的数列极限,则依概率收敛可以等价叙述为:对任意 $\epsilon>0$ 和 $\delta>0$,存在 $N\in\mathbb N$,当 $n\geq N$ 时,都有:
\[
\Pb(|X_n-a|\geq\epsilon)\leq\delta
\]
其中 $\epsilon$ 称作精度,$\delta$ 称作置信水平。
\end{com}

\begin{theorem}[弱大数定律]
设 $X_1,X_2,\ldots$ 是独立同分布的随机变量序列,其公共分布均值为 $\mu$,则样本均值 $M_n=\frac{1}{n}\sum_{i=1}^nX_i$ 依概率收敛到 $\mu$,即对任意 $\epsilon>0$,有:
\[
\lim_{n\to\infty}\mathbb P(|M_n-\mu|\geq\epsilon)=0
\]
\end{theorem}
\begin{proof}
这里仅对方差有界的情形进行证明,方差无界时弱大数定律依然成立,但是证明较为精巧。
考虑 $M_n$ 的均值和方差:
\begin{align*}
&\mathbb EM_n=\frac{1}{n}\sum_{i=1}^n\mathbb EX_i=\mu\\
&\text{var}(M_n)=\frac{1}{n^2}\sum_{i=1}^n\text{var}(X_i)=\frac{\sigma^2}{n}
\end{align*}
于是根据切比雪夫不等式,对任意 $\epsilon>0$,有:
\[
\mathbb P(|M_n-\mu|\geq\epsilon)\leq \frac{\sigma^2}{n\epsilon^2}\to0\quad(n\to\infty)
\]
\end{proof}
\begin{com}
一般情形的弱大数定理称为辛钦大数定律,而方差有界的情形称之为切比雪夫大数定律。
更特殊的,对于 $X_1,X_2,\ldots\overset{\text{i.i.d.}}{\sim}B(1,p)$ 的情形而言,我们称之为伯努利大数定律:
\[
\lim_{n\to\infty}\mathbb P(|M_n-p|\geq \epsilon)=0
\]
\end{com}


\subsection{中心极限定理}

\begin{theorem}[中心极限定理]
设 $X_1,X_2,\ldots$ 是独立同分布的随机变量序列,序列的每一项的均值为 $\mu$,方差为 $\sigma^2$,则对 $\forall x\in\R$,有:
\[
\lim_{n\to\infty}\mathbb P\left(\frac{\sum\limits_{i=1}^nX_i-n\mu}{\sqrt n\sigma}\leq  x\right)=\Phi(x)
\]
其中 $\Phi(x)$ 表示标准正态分布的 CDF.
\end{theorem}
\begin{com}
粗略来讲,中心极限定理表明,在大样本的情况下,独立同分布的随机变量序列的样本均值的标准化结果服从标准正态分布。
\end{com}

\begin{com}
这个一般情形的定理被称作林德伯格-莱维中心极限定理。
\end{com}

\begin{com}
对于 $X_1,X_2,\ldots\overset{\text{i.i.d.}}{\sim} B(1,p)$ 的特殊情形而言,我们称之为棣莫弗-拉普拉斯中心极限定理:
\[
\lim_{n\to\infty}\mathbb P\left(\frac{\sum\limits_{i=1}^nX_i-np}{\sqrt{np(1-p)}}\leq x\right)=\Phi(x)
\]
\end{com}


\subsection{强大数定律}

\begin{definition}[以概率 1 收敛/几乎必然收敛]
设 $X_1,X_2,\ldots$ 是随机变量序列,$a\in\mathbb R$,若:
\[
\Pb\left(\lim_{n\to\infty}X_n=a\right)=1
\]
则称 $\{X_n\}$ 以概率 1 收敛到 $a$ 或几乎必然收敛到 $a$.
\end{definition}

\begin{theorem}[强大数定律]
设 $X_1,X_2,\ldots$ 是均值为 $\mu$ 的独立同分布的随机变量序列,则样本均值 $M_n=\frac{1}{n}\sum_{i=1}^nX_i$ 以概率 1 收敛到 $\mu$,即:
\[
\Pb\left(\lim_{n\to\infty}M_n=\mu\right)=1
\]
\end{theorem}

\begin{com}
弱大数定律是指 $M_n$ 显著性偏离 $\mu$ 的事件的概率在 $n\to\infty$ 时趋于 0,但是对任意有限的 $n$,这个概率可以是正的。所以可以想象的是,在 $M_n$ 这个无穷的序列中,常常有 $M_n$ 显著偏离 $\mu$. 而强大数定律则进一步告诉我们,这样的显著偏离事件只能发生有限次。
\end{com}
 \newpage
\section{贝叶斯统计推断}


\subsection{贝叶斯推断与后验分布}

\begin{figure}[H]
    \centering
    \includegraphics[width=0.8\linewidth]{figs/贝叶斯推断模型.png}
    \caption{贝叶斯推断模型}
    \label{fig:bayesian-inference}
\end{figure}

记感兴趣的未知量为 $\Theta$,并视其为一个随机变量或随机变量的有限集合。我们的目标是基于观测到相关随机变量的值 $X=(X_1,\ldots,X_n)$ 来提取 $\Theta$ 的信息。假定我们已知先验分布 $p_\Theta$ 或 $f_\Theta$,以及条件分布 $p_{X\vert\Theta}$ 或 $f_{X\vert\Theta}$,则贝叶斯推断问题由 $\Theta$ 的后验分布 $p_{\Theta\vert X}$ 或 $f_{\Theta\vert X}$ 完全决定。后验分布可根据贝叶斯公式计算。
% 根据 $\Theta$ 和 $X$ 的离散性和连续性,贝叶斯公式由四种组合方式:
% \begin{gather*}
% p_{\Theta\vert X}(\theta\vert x)=\frac{p_\Theta(\theta)p_{X\vert\Theta}(x\vert\theta)}{\sum_{\theta'}p_\Theta(\theta')p_{X\vert\Theta}(x\vert\theta')},\qquad\text{$\Theta$ 离散,$X$ 离散}\\
% p_{\Theta\vert X}(\theta\vert x)=\frac{p_\Theta(\theta)f_{X\vert\Theta}(x\vert\theta)}{\sum_{\theta'}p_\Theta(\theta')f_{X\vert\Theta}(x\vert\theta')},\qquad\text{$\Theta$ 离散,$X$ 连续}\\
% f_{\Theta\vert X}(\theta\vert x)=\frac{f_\Theta(\theta)p_{X\vert\Theta}(x\vert\theta)}{\int f_\Theta(\theta')p_{X\vert\Theta}(x\vert\theta')\mathrm d\theta'},\qquad\text{$\Theta$ 连续,$X$ 离散}\\
% f_{\Theta\vert X}(\theta\vert x)=\frac{f_\Theta(\theta)f_{X\vert\Theta}(x\vert\theta)}{\int f_\Theta(\theta')f_{X\vert\Theta}(x\vert\theta')\mathrm d\theta'},\qquad\text{$\Theta$ 连续,$X$ 连续}
% \end{gather*}

\begin{example}[正态随机变量公共均值的推断]
\label{ex:normal-mean-infer}
设随机变量观测值 $X=(X_1,\ldots,X_n)$ 具有相同的未知均值。假设在给定均值的条件下,$X_i$ 是正态的,且相互独立,方差分别为 $\sigma_1^2,\ldots,\sigma_n^2$. 又设 $X_i$ 的公共均值为随机变量 $\Theta$,且先验分布为正态分布 $N(x_0,\sigma_0^2)$. 那么:
\begin{gather*}
f_\Theta(\theta)=c_1\cdot\exp\left(-\frac{(\theta-x_0)^2}{2\sigma_0^2}\right)\\
f_{X\vert\Theta}(x\vert\theta)=c_2\cdot\exp\left(-\frac{(x_1-\theta)^2}{2\sigma_1^2}\right)\cdots\exp\left(-\frac{(x_n-\theta)^2}{2\sigma_n^2}\right)
\end{gather*}
其中 $c_1,c_2$ 为常数。根据贝叶斯公式,有:
\begin{align*}
f_{\Theta\vert X}(\theta\vert x)\propto f_\Theta(\theta)f_{X\vert\Theta}(x\vert\theta)\propto\exp\left(-\sum_{i=0}^n\frac{(\theta-x_i)^2}{2\sigma_i^2}\right)\propto\exp\left(-\frac{(\theta-m)^2}{2v}\right)
\end{align*}
其中根据配方可得:
\[
m=\frac{\sum_{i=0}^nx_i/\sigma_i^2}{\sum_{i=0}^n1/\sigma_i^2},\quad v=\frac{1}{\sum_{i=0}^n1/\sigma_i^2}
\]
于是后验概率就是以 $m$ 为均值、$v$ 为方差的正态分布。
\end{example}

\begin{definition}[共轭分布]
在贝叶斯推断中,若先验分布与后验分布是同一个分布族,则先验分布与后验分布被称为共轭分布。
\end{definition}
\begin{com}
例 \ref{ex:normal-mean-infer} 显示正态分布与其自身是共轭分布。这并不是一个普遍的情形,除了正态分布以外,其他常见的与自身共轭的分布有伯努利分布和二项分布。
\end{com}


\subsection{点估计,假设检验,最大后验概率准则}

\paragraph{点估计}
给定 $X$ 的观测值 $x$,贝叶斯推断给出了后验分布 $p_{\Theta\vert X}(\theta\vert x)$ 或 $f_{\Theta\vert X}(\theta\vert x)$. 后验分布包含了 $x$ 提供的所有信息,但有时我们希望得到一个数而不是一个概率分布,这就是点估计。具体而言,点估计指从 $X$ 的观测值中提取一个数 $\hat\theta=g(x)$ 作为对 $\Theta$ 的估计的方法。

\begin{definition}[估计量,估计值]
设 $g$ 是一个关于 $X$ 的函数,则称随机变量 $\hat\Theta=g(X)$ 为估计量。当观测到 $X$ 的值 $x$ 后,则称 $\hat\Theta$ 的取值 $\hat\theta=g(x)$ 为估计值。不同的函数 $g$ 引出不同的估计量。
\end{definition}

\begin{definition}[最大后验概率估计量]
给定 $X$ 的观测值 $x$,选择使得后验概率最大的 $\theta$ 作为估计值,即:
\[
\hat\theta=\max_\theta p_{\Theta\vert X}(\theta\vert x)\;\text{或}\;\max_\theta f_{\Theta\vert X}(\theta\vert x)
\]
\end{definition}

\begin{definition}[条件期望估计量]
给定 $X$ 的观测值 $x$,选择条件期望作为估计值,即:
\[
\hat\theta=\E[\Theta\vert X=x]
\]
\end{definition}
\begin{com}
在下一节中将看到,条件期望估计量其实就是最小均方估计量。
\end{com}

\begin{example}[正态随机变量公共均值的估计量]
\label{ex:normal-mean-estimate}
在例 \ref{ex:normal-mean-infer} 中,我们得到了正态随机变量公共均值 $\Theta$ 的后验分布是以 $m$ 为均值、$v$ 为方差的正态分布,其中:
\[
m=\frac{\sum_{i=0}^nx_i/\sigma_i^2}{\sum_{i=0}^n1/\sigma_i^2},\quad v=\frac{1}{\sum_{i=0}^n1/\sigma_i^2}
\]
由于正态分布的概率密度函数在均值处取最大值,所以最大后验概率估计为:
\[
\hat\theta=m
\]
而正态分布的均值参数正是其期望,因此条件期望估计也为:
\[
\hat\theta=\E[\Theta\vert X=x]=m
\]
因此在该模型中,最大后验概率估计和条件期望估计恰好相同。
\end{example}

\paragraph{假设检验}
在一个假设检验问题中,$\Theta$ 取 $\theta_1,\ldots,\theta_m$ 中的一个值,其中 $m$ 是一个较小的整数(常常是 $m=2$)。称事件 $H_i=\{\Theta=\theta_i\}$ 为第 $i$ 个假设。观测到 $X$ 的取值 $x$ 后,我们希望依据某种准则选出一个最“合理”的假设。

\begin{definition}[假设检验的最大后验概率准则]
给定观测值 $x$,最大后验概率准则选择使得后验概率 $\Pb(\Theta=\theta_i\vert X=x)$ 最大的假设 $H_i$:
\[
\arg\max_i\;\Pb(\Theta=\theta_i\vert X=x)
\]
根据贝叶斯公式,这等价于:
\begin{gather*}
    \arg\max_i\;p_\Theta(\theta_i)p_{X\vert\Theta}(x\vert\theta_i),\quad\text{$X$ 离散}\\
    \arg\max_i\;p_\Theta(\theta_i)f_{X\vert\Theta}(x\vert\theta_i),\quad\text{$X$ 连续}
\end{gather*}
\end{definition}
\begin{com}
对任意观测值 $x$,最大后验概率准则是错误率最小的决策准则。
\end{com}


\subsection{贝叶斯最小均方估计}

\begin{theorem}[最小均方估计——无观测值情形]
\label{thm:lms-not-given-x}
考虑在没有观测值的情况下,用常数 $\hat\theta$ 去估计 $\Theta$,则使得均方误差 $\E[(\Theta-\hat\theta)^2]$ 最小的估计为:
\[\hat\theta=\E\Theta\]
换句话说,有:
\[
\E[(\Theta-\E\Theta)^2]\leq\E[(\Theta-\hat\theta)^2],\quad\forall\,\hat\theta
\]
\end{theorem}
\begin{proof}
对任何估计 $\hat\theta$,有均方误差:
\[
\E[(\Theta-\hat\theta)^2]=\var(\Theta-\hat\theta)+(\E[\Theta-\hat\theta])^2=\var(\Theta)+(\E\Theta-\hat\theta)^2
\]
由于 $\var(\Theta)$ 与 $\hat\theta$ 无关,故当 $(\E\Theta-\hat\theta)^2$ 最小时均方误差最小,也即 $\hat\theta=\E\Theta$.
\end{proof}

\begin{theorem}[最小均方估计——给定观测值情形]
\label{thm:lms-given-x}
设给定观测值 $x$,则使得均方误差 $\E[(\Theta-\hat\theta)^2\vert X=x]$ 最小的估计为条件期望估计:
\[\hat\theta=\E[\Theta\vert X=x]\]
换句话说,有:
\[
\E[(\Theta-\E[\Theta\vert X=x])^2\vert X=x]\leq\E[(\Theta-\hat\theta)^2\vert X=x],\quad\forall\,\hat\theta
\]
\end{theorem}
\begin{proof}
在定理 \ref{thm:lms-not-given-x} 的证明中加入 $X=x$ 的条件即可。
\end{proof}

\begin{theorem}[最小均方估计——总体情形]
\label{thm:lms-overall}
总体上,设估计量为 $\hat\Theta$,则使得均方误差 $\E[(\Theta-\hat\Theta)^2]$ 最小的估计量为:
\[
\hat\Theta=\E[\Theta\vert X]
\]
换句话说,有:
\[
\E[(\Theta-\E[\Theta\vert X])^2]\leq \E[(\Theta-\hat\Theta)^2],\quad\forall\,\hat\Theta=g(X)
\]
\end{theorem}
\begin{proof}
对于任意给定 $X$ 的取值 $x$,$\hat\theta=g(x)$ 是一个数,因此:
\[
\E[(\Theta-\E[\Theta\vert X=x])^2\vert X=x]\leq\E[(\Theta-g(x))^2\vert X=x]
\]
根据 $x$ 的任意性,有:
\[
\E[(\Theta-\E[\Theta\vert X])^2\vert X]\leq\E[(\Theta-g(X))^2\vert X]
\]
根据全期望公式,两边取期望得:
\[
\E[(\Theta-\E[\Theta\vert X])^2]\leq\E[(\Theta-g(X))^2]
\]
\end{proof}

\begin{property}
将最小均方估计和估计误差分别记为:
\[
\hat\Theta=\E[\Theta\vert X],\quad\tilde\Theta=\hat\Theta-\Theta
\]
在 \ref{sec:cond-e-cond-var} 节中已经推导了一些性质,这里列举如下:
\begin{itemize}
    \item $\tilde\Theta$ 是无偏的,它的条件期望和非条件期望都是 0:
    $\E[\tilde\Theta]=0,\;\E[\tilde\Theta\vert X=x]=0,\,\forall\,x$
    \item 估计误差 $\tilde\Theta$ 与估计量 $\hat\Theta$ 不相关:
    $\var(\hat\Theta,\tilde\Theta)=0$
    \item $\Theta$ 的方差可以分解为:
    $\var(\Theta)=\var(\hat\Theta)+\var(\tilde\Theta)$
\end{itemize}
\end{property}

\begin{theorem}[推广到多观测情形]
设有多个观测量 $X_1,\ldots,X_n$,则最小均方估计为:
\[
\hat\Theta=\E[\Theta\vert X_1,\ldots,X_n]
\]
换句话说,有:
\[
\E[(\Theta-\E[\Theta\vert X_1,\ldots,X_n])^2]\leq\E[(\Theta-\hat\Theta)^2],\quad\forall\,\hat\Theta=g(X_1,\ldots,X_n)
\]
\end{theorem}

\subsection{贝叶斯线性最小均方估计}
\label{sec:bayesian-lms}

\begin{definition}[线性估计量]
限定 $g$ 是关于 $X$ 的线性函数 $g(X)=aX+b$,称随机变量:
\[\hat\Theta=g(X)=aX+b\]
为线性估计量。进一步地,若有多个观测量 $X_1,\ldots,X_n$,则线性估计量的形式为:
\[\hat\Theta=a_1X_1+\cdots+a_nX_n+b\]
\end{definition}

\begin{theorem}[线性最小均方估计]
\label{thm:linear-lms-estimate}
基于 $X$ 的 $\Theta$ 的线性最小均方估计是:
\[
\hat\Theta=\E\Theta+\frac{\cov(\Theta,X)}{\var(X)}(X-\E X)=\E\Theta+\rho\frac{\sigma_\Theta}{\sigma_X}(X-\E X)
\]
其中 $\rho=\dfrac{\cov(\Theta,X)}{\sigma_\Theta\sigma_X}$ 是相关系数,此时所得均方误差为:
\[
\E[(\Theta-aX-b)^2]=(1-\rho^2)\sigma_\Theta^2
\]
\end{theorem}
\begin{proof}
假设固定 $a$,则问题等价于选择常数 $b$ 来估计随机变量 $\Theta-aX$. 根据之前的讨论,最优解为:
\[
b=\E[\Theta-aX]=\E\Theta-a\E X
\]
因此问题转化为:
\[
\min_a\;\E[(\Theta-aX-\E[\Theta-aX])^2]=\var(\Theta-aX)
\]
打开方差:
\[
\var(\Theta-aX)=\var(\Theta)+\var(aX)-2\cov(\Theta,aX)=\sigma_\Theta^2+a^2\sigma_X^2-2a\rho\sigma_\Theta\sigma_X
\]
这是关于 $a$ 的二次函数,最小值在顶点处取得:
\[
a=\frac{\rho\sigma_\Theta\sigma_X}{\sigma_X^2}=\rho\frac{\sigma_\Theta}{\sigma_X}
\]
因此线性最小均方估计为:
\[
\hat\Theta=aX+b=aX+\E\Theta-a\E X=\E\Theta+\rho\frac{\sigma_\Theta}{\sigma_X}(X-\E X)
\]
且估计误差为:
\[
\var(\Theta-aX)=\sigma_\Theta^2+a^2\sigma_X^2-2a\rho\sigma_\Theta\sigma_X=\sigma_\Theta^2+\rho^2\sigma_\Theta^2-2\rho^2\sigma_\Theta^2=(1-\rho^2)\sigma_\Theta^2
\]
\end{proof}
\begin{remark}
直观上,估计量以 $\E\Theta$ 为基础,通过 $X-\E X$ 的取值来调整。例如,不妨假设 $\rho>0$,则当观测到比 $\E X$ 更大的 $X$ 取值后,我们对 $\Theta$ 的估计也就相应地提高。另外,当 $|\rho|$ 接近 1 时,$X$ 和 $\Theta$ 高度相关,了解 $X$ 将帮助我们准确地估计 $\Theta$,因此均方误差较小。
\end{remark}

\begin{example}[正态随机变量公共均值的估计量-续]
在例 \ref{ex:normal-mean-estimate} 中,我们得到正态随机变量公共均值的最小均方估计(条件期望估计)为:
\[
\hat\theta=\E[\Theta\vert X=x]=m=\frac{\sum_{i=0}^nx_i/\sigma_i^2}{\sum_{i=0}^n1/\sigma_i^2}
\]
进一步,注意到上式是观测值 $x_1,\ldots,x_n$ 的线性组合,因此它其实也是线性最小均方估计。也即,在该模型中,最大后验概率估计、最小均方估计和线性最小均方估计恰好都是相同的。
\end{example}
 \newpage
\section{广义逆矩阵}

\subsection{投影矩阵及其应用}

\begin{definition}[投影算子]
设 $\mathbb C^n=L\oplus M$,则对于任意 $x\in\mathbb C^n$ 有唯一分解 $x=y+z,\,y\in L,\,z\in M$.  称将 $x$ 变为 $y$ 的变换为沿着 $M$ 到 $L$ 的投影算子,记作 $P_{L,M}$,即:
\[P_{L,M}x=y\]
\begin{figure}[H]
    \centering
    \includegraphics[width=0.2\linewidth]{figs/proj.png}
\end{figure}
\end{definition}

\begin{property}
$R(P_{L,M})=L,\,N(P_{L,M})=M$. 注意 $x-y\in M$.
\end{property}

\begin{definition}[投影矩阵]
投影算子 $P_{L,M}$ 在 $\mathbb C^n$ 的基 $(e_1,\ldots,e_n)$ 下的矩阵称为投影矩阵。
\end{definition}

\begin{lemma}
设 $A\in\mathbb C^{n\times n}$ 是幂等矩阵,即 $A^2=A$,则:
\[N(A)=R(I-A)\]
\end{lemma}
\begin{proof}
任取 $x\in N(A)$,则 $Ax=0$,则 $x=Ax+(I-A)x=(I-A)x\in R(I-A)$,因此 $N(A)\subset R(I-A)$;
任取 $y\in\mathbb C^n$,$A(I-A)y=(A-A^2)y=0$,故 $(I-A)y\in N(A)$,故 $R(I-A)\subset N(A)$;
综上,$N(A)=R(I-A)$.
\end{proof}

\begin{theorem}[投影与幂等]
矩阵 $P$ 为投影矩阵的充要条件是 $P$ 为幂等矩阵,即:
\[
    P_{n\times n}=P_{L,M}\iff P^2=P
\]
\end{theorem}
\begin{proof}
必要性:设 $C^n=L\oplus M$,则对于任意 $x\in\mathbb C^n$,存在唯一的分解 $x=y+z,\,y\in L,\,z\in M$.  于是 $P_{L,M}x=y$. 因此 $P_{L,M}^2x=P_{L,M}y=y=P_{L,M}x$,即 $P_{L,M}$ 是幂等的。

充分性:任意 $x\in\mathbb C^n$ 可分解为 $x=Px+(I-P)x$,根据引理知 $N(P)=R(I-P)$,又 $\mathbb C^n=R(P)\oplus N(P)$,所以这样的分解是唯一的,于是 $P=P_{R(P),N(P)}$.
\end{proof}

\noindent\textbf{计算方法}:取 $L$ 的一组基 $(q_1,\ldots,q_r)$ 和 $M$ 的一组基 $(q_{r+1},\ldots,q_n)$,则任意向量 $x\in\mathbb C^n$ 可表示为:
\[
    x=(q_1,\ldots,q_r,q_{r+1},\ldots,q_n)y=Qy
\]
于是:
\[
    P_{L,M}x=QI_ry=QI_rQ^{-1}x\implies P_{L,M}=QI_rQ^{-1}
\]
其中 $I_r$ 表示前 $r$ 个对角元为 1、其余为 0 的对角矩阵。

\begin{com}
可以看见上面的计算方法涉及到基的选取,但可以证明选取不同的基算出来的 $P_{L,M}$ 都是一样的。
假设另选一组基 $\bar Q_L=(\bar q_1,\ldots,\bar q_r)$ 和 $\bar Q_M=(\bar q_{r+1},\ldots,\bar q_n)$,设 $\bar Q_L=Q_LR_1,\,\bar Q_M=Q_MR_2$,则 $\bar Q=Q\text{diag}(R_1,R_2)$,于是:
\[
    \bar P_{L,M}=\bar Q I_r\bar Q^{-1}=Q\text{diag}(R_1,R_2)I_r\text{diag}(R_1^{-1},R_2^{-1})Q^{-1}=QI_rQ^{-1}=P_{L,M}
\]
可见 $P_{L,M}$ 与基的选取无关。
\end{com}

\begin{definition}[正交投影算子]
设 $L$ 是 $\mathbb C^n$ 的子空间,则沿着 $L^{\perp}$ 到 $L$ 的投影算子 $P_{L,L^{\perp}}$ 为正交投影算子,简记为 $P_L$.
\begin{figure}[H]
    \centering
    \includegraphics[width=0.2\linewidth]{figs/proj-o.png}
\end{figure}
\end{definition}

\begin{definition}[正交投影矩阵]
正交投影算子 $P_{L}$ 在 $\mathbb C^n$ 的基 $e_1,\ldots,e_n$ 下的矩阵称为正交投影矩阵。
\end{definition}


\begin{theorem}[正交投影与幂等 Hermite]
矩阵 $P$ 为正交投影矩阵的充要条件是 $P$ 为幂等 Hermite 矩阵。
\end{theorem}
\begin{proof}
必要性:若 $P$ 为正交投影矩阵,则根据上一节定理知它是幂等矩阵,于是 $R(I-P)=N(P)$。又 $R(P)\perp N(P)$,所以 $R(P)\perp R(I-P)$,因此对于任意 $x,y\in\mathbb C^n$,有:
\begin{align*}
    x^HP^H(I-P)y=0&\implies P^H(I-P)=0\implies P^H=P^HP\\&\implies P=(P^HP)^H=P^HP=P^H
\end{align*}
即 $P$ 是 Hermite 矩阵。

充分性:若 $P$ 是幂等 Hermite 矩阵,则根据上一节定理知它是投影矩阵 $P_{R(P),N(P)}$.  又由于 $P^H=P$,所以:
\[
    P_{R(P),N(P)}=P_{R(P),N(P^H)}=P_{R(P),R^\perp(P)}
\]
即 $P$ 是正交投影矩阵。
\end{proof}

\noindent\textbf{计算方法}:取 $L$ 的一组基 $X=(x_1,\ldots,x_r)$,$L^{\perp}$ 的一组基 $y=(y_1,\ldots,y_{n-r})$,则 $X^HY=Y^HX=O$.  根据上一节投影矩阵的计算方法知:
\[
    P_L=P_{L,L^\perp}
    =\left[\begin{array}{c:c}X&Y\end{array}\right]\;I_r\;\left[\begin{array}{c:c}X&Y\end{array}\right]^{-1}
    =\left[\begin{array}{c:c}X&O\end{array}\right]\left[\begin{array}{c:c}X&Y\end{array}\right]^{-1}
\]
由于:
\[
    \left[\begin{array}{c:c}X&Y\end{array}\right]^{H}\left[\begin{array}{c:c}X&Y\end{array}\right]=\left[\begin{array}{c}X^H\\\hdashline Y^H\end{array}\right]\left[\begin{array}{c:c}X&Y\end{array}\right]=\left[\begin{array}{c:c}X^HX&O\\\hdashline O&Y^HY\end{array}\right]
\]
于是:
\[
    \left[\begin{array}{c:c}X&Y\end{array}\right]^{-1}=\left[\begin{array}{c:c}(X^HX)^{-1}&O\\\hdashline O&(Y^HY)^{-1}\end{array}\right]\left[\begin{array}{c}X^H\\\hdashline Y^H\end{array}\right]=\left[\begin{array}{c}(X^HX)^{-1}X^H\\\hdashline(Y^HY)^{-1}Y^H\end{array}\right]
\]
因此:
\[
    P_L=\left[\begin{array}{c:c}X&O\end{array}\right]\left[\begin{array}{c:c}X&Y\end{array}\right]^{-1}=\left[\begin{array}{c:c}X&O\end{array}\right]\left[\begin{array}{c}(X^HX)^{-1}X^H\\\hdashline(Y^HY)^{-1}Y^H\end{array}\right]=X(X^HX)^{-1}X^H
\]

\begin{com}
同样的,正交投影矩阵的计算也与选取的基无关。假设有另一组基 $\bar X=(\bar x_1,\ldots,\bar x_r)$,设 $\bar X=XR$,则:
\begin{align*}
    \bar P_L&=\bar X({\bar X}^H\bar X)^{-1}{\bar X}^H=XR(R^HX^HXR)^{-1}R^HX^H\\
    &=XRR^{-1}(X^HX)^{-1}(R^H)^{-1}R^HX^H=X(X^HX)^{-1}X^H=P_L
\end{align*}
可见 $P_L$ 与基的选取无关。
\end{com}

\begin{remark}
由于 $X$ 是列满秩矩阵,根据下一节的内容可知 $X^+=(X^HX)^{-1}X$,所以 $P_L=XX^+$.
\end{remark}


\subsection{广义逆矩阵的存在、性质及构造方法}

\begin{definition}
\label{def:moore-penrose}
设矩阵 $A\in\mathbb C^{m\times n}$,若矩阵 $X\in\mathbb C^{n\times m}$ 满足如下四个 Penrose 方程:
\begin{align*}
    &AXA=A\tag{1}\label{1}\\
    &XAX=X\tag{2}\label{2}\\
    &(AX)^H=AX\tag{3}\label{3}\\
    &(XA)^H=XA\tag{4}\label{4}
\end{align*}
则称 $X$ 为 $A$ 的 Moore-Penrose 逆,记作 $A^+$.
若 $X$ 值满足上述四个方程中的第 $(i),(j),\ldots,(l)$ 个方程,则称 $X$ 为 $A$ 的 $\{i,j,\ldots,l\}$-逆,记作 $A^{(i,j,\ldots,l)}$,其全体记为 $A\{i,j,\ldots,l\}$.

如下为 1-逆的示意图:
\begin{figure}[H]
    \centering
    \includegraphics[width=0.5\linewidth]{figs/1inv.png}
\end{figure}
\end{definition}

\begin{theorem}
对任意 $A\in\mathbb C^{m\times n}$,$A^+$ 存在且唯一。
\end{theorem}
\begin{proof}
存在性。对 $A$ 做奇异值分解 $A=U\begin{bmatrix}\Sigma&0\\0&0\end{bmatrix}V^H$,取 $X=V\begin{bmatrix}\Sigma^{-1}&0\\0&0\end{bmatrix}U^H$,可以验证 $X$ 满足 $A^+$ 的四个条件。

唯一性。设 $X,Y$ 均是 $A^+$,则:
\begin{align*}
    Y&=YAY=Y(AY)^H=YY^HA^H=YY^H(AXA)^H\\&=YY^HA^H(AX)^H=Y(AY)^HAX=YAYAX=YAX\\
    X&=XAX=(XA)^HX=A^HX^HX=(AYA)^HX^HX\\&=(YA)^HA^HX^HX=(YA)^H(XA)^HX=(YA)^HXAX=(YA)^HX=YAX
\end{align*}
故 $X=Y$.
\end{proof}

\begin{remark}
上述定理的证明过程也给出了 $A^+$ 的一种基于奇异值分解的计算方法:
\[
    A=U\begin{bmatrix}\Sigma&0\\0&0\end{bmatrix}V^H\implies     A^+=V\begin{bmatrix}\Sigma^{-1}&0\\0&0\end{bmatrix}U^H
\]
\end{remark}

\begin{theorem}
\[
    \lim_{\delta\to0}(\delta^2I+A^HA)^{-1}A^H=A^+
\]
\end{theorem}
\begin{proof}
设 $A=U\begin{bmatrix}\Sigma&0\\0&0\end{bmatrix}V^H$,则 $A^HA=V\begin{bmatrix}\Sigma^2&0\\0&0\end{bmatrix}V^H$,于是:
\[\delta^2I+A^HA=V\begin{bmatrix}\Sigma^2+\delta^2I&0\\0&\delta^2I\end{bmatrix}V^H\]
因此:
\begin{align*}
    (\delta^2I+A^HA)^{-1}A^H&=V\begin{bmatrix}\left[\sigma_i^2+\delta^2\right]_{r\times r}&0\\0&\delta^2I\end{bmatrix}^{-1}V^HA^H\\
    &=V\begin{bmatrix}\left[\dfrac{1}{\sigma_i^2+\delta^2}\right]_{r\times r}&0\\0&\delta^{-2}I\end{bmatrix}(AV)^H\\
    &=V\begin{bmatrix}\left[\dfrac{1}{\sigma_i^2+\delta^2}\right]_{r\times r}&0\\0&\delta^{-2}I\end{bmatrix}\begin{bmatrix}\Sigma&0\\0&0\end{bmatrix}U^H\\
    &=V\begin{bmatrix}\left[\dfrac{\sigma_i}{\sigma_i^2+\delta^2}\right]_{r\times r}&0\\0&0\end{bmatrix}U^H\\
\end{align*}
于是当 $\delta\to0$ 时,
\[
\lim_{\delta\to0}(\delta^2I+A^HA)^{-1}A^H=\lim_{\delta\to0}V\begin{bmatrix}\left[\dfrac{\sigma_i}{\sigma_i^2+\delta^2}\right]_{r\times r}&0\\0&0\end{bmatrix}U^H=V\begin{bmatrix}\Sigma^{-1}&0\\0&0\end{bmatrix}U^H=A^+
\]
\end{proof}

\begin{lemma}
\begin{align*}
    N(A)\supset N(B)&\iff \exists X,A=XB\\
    R(A)\subset R(B)&\iff \exists X,A=BX
\end{align*}
\end{lemma}

\begin{corollary}
\label{cor:rankAB}
\begin{align*}
    &\text{rank}(AB)=\text{rank}(A)\implies \exists X,A=ABX\\
    &\text{rank}(BA)=\text{rank}(A)\implies \exists X,A=XBA
\end{align*}
\end{corollary}
\begin{proof}
由于 $R(AB)\subset R(A)$,且 $\dim R(AB)=\text{rank}(AB)=\text{rank}(A)=\dim R(A)$,故 $R(AB)=R(A)$.  于是 $R(AB)\supset R(A)$,根据引理知 $\exists X,A=ABX$.

类似地,由于 $N(BA)\supset N(A)$,且 $\dim N(BA)=n-\text{rank}(BA)=n-\text{rank}(A)=\dim N(A)$,故 $N(BA)=N(A)$.  于是 $N(BA)\subset N(A)$,根据引理知 $\exists X,A=XBA$.
\end{proof}

\begin{remark}
推论的这两个式子在证明中\textbf{非常常用},即用更复杂的式子表示简单的矩阵,反而有助于证明。
\end{remark}

\begin{theorem}
矩阵 $A\in\mathbb C^{m\times n}$ 有唯一 1-逆的充要条件为 $A$ 是非奇异矩阵,且该 1-逆就是 $A^{-1}$.
\end{theorem}
\begin{proof}
充分性显然,必要性证明如下。设 $Au=0$,$AXA=A$,那么容易验证 $X'=X+u\cdot[1,0,\ldots,0]$ 也满足 $AX'A=A$,由于 1-逆唯一,故 $u=0$,即 $N(A)=\{0\}$.  类似可以证明 $N(A^H)=\{0\}$,于是 $A$ 列满秩且行满秩,故 $A$ 为可逆方阵。
\end{proof}

\begin{property}[1]
$(A^{(1)})^H\in A^{H}\{1\}$.
\end{property}

\begin{property}[2]
$\lambda^+ A^{(1)}\in(\lambda A)\{1\}$. 其中 $\lambda\in\mathbb C,\,\lambda^{+}=\begin{cases}\lambda^{-1},&\lambda\neq 0\\0,&\lambda=0\end{cases}$.
\end{property}

\begin{property}[3]
若 $S$ 和 $T$ 非奇异,则 $T^{-1}A^{(1)}S^{-1}\in(SAT)\{1\}$.
\end{property}

\begin{property}[4]
$\text{rank}(A^{(1)})\geq \text{rank}(A)$.
\end{property}
\begin{proof}
$\text{rank}(A)=\text{rank}(AA^{(1)}A)\leq\text{rank}(A^{(1)})$.
\end{proof}

\begin{property}[5]
$AA^{(1)}$ 和 $A^{(1)}A$ 均为幂等矩阵且与 $A$ 同秩。
\end{property}
\begin{proof}
$\text{rank}(AA^{(1)})\leq\text{rank}(A)=\text{rank}(AA^{(1)}A)\leq\text{rank}(AA^{(1)})$,故 $\text{rank}(AA^{(1)})=\text{rank}(A)$.
\end{proof}

\begin{property}[6]
$R(AA^{(1)})=R(A),\,N(A^{(1)}A)=N(A),\,R((A^{(1)}A)^H)=R(A^H)$.
\end{property}
\begin{proof}
$R(AA^{(1)})\subset R(A)=R(AA^{(1)}A)\subset R(AA^{(1)})$,故 $R(AA^{(1)})=R(A)$.

类似地,$N(A)\subset N(A^{(1)}A)\subset N(AA^{(1)}A)=N(A)$,故 $N(A^{(1)}A)=N(A)$.
\end{proof}

\begin{property}[7]
$A^{(1)}A=I_n\iff \text{rank}(A)=n$,$AA^{(1)}=I_m\iff \text{rank}(A)=m$.
\end{property}
\begin{proof}
根据性质 5,$\text{rank}(A^{(1)}A)=\text{rank}(A)$,因此必要性:$A^{(1)}A=I_n\implies\text{rank}(A^{(1)}A)=n\implies\text{rank}(A)=n$;充分性:$\text{rank}(A)=n\implies \text{rank}(A^{(1)}A)=n$,即 $A^{(1)}A$ 可逆,又 $A^{(1)}A$ 幂等,故为单位阵。另一个类似。证毕。
\end{proof}

\begin{property}[8]
推论 \ref{cor:rankAB} 的进一步阐述,给出了存在的 $X$ 的具体形式:
\begin{gather*}
    AB(AB)^{(1)}A=A\iff\text{rank}(AB)=\text{rank}(A)\\
    B(AB)^{(1)}AB=B\iff\text{rank}(AB)=\text{rank}(B)
\end{gather*}
\end{property}
\begin{proof}
这里只证明第一行,第二行类似可证。

充分性:根据推论 \ref{cor:rankAB},存在 $X$ 使得 $A=ABX$,于是 $AB(AB)^{(1)}A=AB(AB)^{(1)}ABX=ABX=A$.

必要性:$\text{rank}(A)\geq\text{rank}(AB)\geq\text{rank}(AB(AB)^{(1)}A)=\text{rank}(A)$,故 $\text{rank}(AB)=\text{rank}(A)$.
\end{proof}

\begin{theorem}
设 $Y,Z\in A\{1\}$,则 $X=YAZ\in A\{1,2\}$.
\end{theorem}
\begin{proof}
$XAX=YAZAYAZ=YAYAZ=YAZ=X$,故 $X\in A\{2\}$. 又 $AXA=AYAZA=AZA=A$,故 $X\in A\{1\}$. 综上,$X\in A\{1,2\}$.
\end{proof}
\begin{proof}[证明 2(利用通解形式,见下文)]
奇异值分解 $A=U\begin{bmatrix}\Sigma&0\\0&0\end{bmatrix}V^H$,则 $Y,Z$ 可分别写作:
\[
    Y=V\begin{bmatrix}\Sigma&C_1\\D_1&E_1\end{bmatrix}U^H,\quad Z=V\begin{bmatrix}\Sigma&C_2\\D_2&E_2\end{bmatrix}U^H
\]
于是:
\[
    X=YAZ=V\begin{bmatrix}B^{-1}&C_1\\D_1&E_1\end{bmatrix}U^HU\begin{bmatrix}B&0\\0&0\end{bmatrix}V^HV\begin{bmatrix}B^{-1}&C_2\\D_2&E_2\end{bmatrix}U^H=V\begin{bmatrix}B^{-1}&C_2\\D_1&D_1BC_2\end{bmatrix}U^H
\]
这正是 1,2-逆的通解形式,故 $X\in A\{1,2\}$.
\end{proof}

\begin{theorem}
给定矩阵 $A$ 和 $X\in A\{1\}$,则 $X\in A\{1,2\}$ 的充要条件是 $\text{rank}(X)=\text{rank}(A)$.
\end{theorem}
\begin{proof}
充分性。由于 $X\in A\{1\}$,故 $A=AXA$,于是 $\text{rank}(A)=\text{rank}(AXA)\leq\text{rank}(XA)\leq\text{rank}(X)=\text{rank}(A)$,故 $\text{rank}(X)=\text{rank}(XA)$. 根据推论,存在 $Y$ 使得 $X=XAY$,于是 $XAX=XAXAY=XAY=X$,故 $X\in A\{2\}$.

必要性。由于 $A=AXA,\,XAX=X$,于是 $\text{rank}(X)=\text{rank}(XAX)\leq\text{rank}(A)=\text{rank}(AXA)\leq\text{rank}(X)$,故 $\text{rank}(X)=\text{rank}(A)$.
\end{proof}

\begin{lemma}
对任意矩阵 $A$ 均有:
\[
    \text{rank}(A^HA)=\text{rank}(A)=\text{rank}(AA^H)
\]
\end{lemma}
\begin{proof}
由于 $A^HAx=0\implies x^HA^HAx=0\implies Ax=0$,所以 $N(A^HA)\subset N(A)$.  又 $N(A^HA)\supset N(A)$,于是 $N(A^HA)=N(A)$,于是 $\text{rank}(A^HA)=\text{rank}(A)$. 另一个类似。
\end{proof}

\begin{theorem}
设有矩阵 $A$,则:
\begin{align*}
    &Y=(A^HA)^{(1)}A^H\in A\{1,2,3\}\\
    &Z=A^H(AA^H)^{(1)}\in A\{1,2,4\}
\end{align*}
\end{theorem}
\begin{proof}
由于 $\text{rank}(A^HA)=\text{rank}(A)=\text{rank}(AA^H)$,根据 1-逆的性质 8 有:
\[
    A=A(A^HA)^{(1)}A^HA,\quad A^H=A^HA(A^HA)^{(1)}A^H
\]
因此:
\begin{align*}
    &AYA=A(A^HA)^{(1)}A^HA=A&\implies Y\in A\{1\}\\
    &YAY=(A^HA)^{(1)}A^HA(A^HA)^{(1)}A^H=(A^HA)^{(1)}A^H=Y&\implies Y\in A\{2\}
\end{align*}
又存在 $X$ 使得 $A=XA^HA$,故:
\begin{align*}
    AY&=A(A^HA)^{(1)}A^H=(XA^HA)(A^HA)^{(1)}(XA^HA)^H\\
    &=XA^HA(A^HA)^{(1)}A^HAX^H=XA^HAX^H
\end{align*}
是 Hermite 矩阵,于是 $Y\in A\{3\}$.
$Z$ 可类似证明。
\end{proof}

\begin{theorem}
\[
    A^+=A^{(1,4)}AA^{(1,3)}
\]
\end{theorem}
\begin{proof}
设 $X=A^{(1,4)}AA^{(1,3)}$,根据关于 1,2-逆的定理知 $X\in A\{1,2\}$. 另外,
\[
    AX=AA^{(1,4)}AA^{(1,3)}=AA^{(1,3)},\quad XA=A^{(1,4)}AA^{(1,3)}A=A^{(1,4)}A
\]
均是 Hermite 矩阵,从而得到结论。
\end{proof}
\begin{proof}[证明 2(利用通解形式,见下文)]
{\color{red}{TODO}}
\end{proof}

\begin{theorem}
给定矩阵 $A\in\mathbb C^{m\times n}$,有:
\begin{enumerate}
    \item $\text{rank}(A^+)=\text{rank}(A)$.
    \item $(A^+)^+=A$.
    \item $(A^H)^+=(A^+)^H,\,(A^T)^+=(A^+)^T$.
    \item $(A^HA)^+=A^+(A^H)^+,\,(AA^H)^+=(A^H)^+A^+$.
    \item $A^+=(A^HA)^+A^H=A^H(AA^H)^+$.
    \item $R(A^+)=R(A^H),\,N(A^+)=N(A^H)$.
\end{enumerate}
\end{theorem}
\begin{proof}
前 5 条都可以通过定义证明。对于第 6 条,根据 1 可知 $\text{rank}(A^+)=\text{rank}(A)=\text{rank}(A^H)$,根据 5 可知 $R(A^+)\subset R(A^H),\,N(A^+)\supset N(A^H)$,于是 $R(A^+)=R(A^H),\,N(A^+)=N(A^H)$.
\end{proof}

\begin{corollary}
若 $A\in\mathbb C_n^{m\times n}$,即列满秩,则 $A^+=(A^HA)^{-1}A^H$;若 $A\in\mathbb C_m^{m\times n}$,即行满秩,则 $A^+=A^H(AA^H)^{-1}$.
\end{corollary}

\begin{corollary}
若 $\alpha\in\mathbb C^n$,且 $\alpha\neq 0$,则 $\alpha^+=(\alpha^H\alpha)^{-1}\alpha^H$,而 $(\alpha^H)^+=(\alpha^+)^H=\alpha(\alpha^H\alpha)^{-1}$.
\end{corollary}

\vskip 6pt \noindent\textbf{广义逆的通解形式}:
设 $A\in\mathbb C^{m\times n}_r$,则存在 $m$ 阶可逆矩阵(或酉矩阵)$P$ 和 $n$ 阶可逆矩阵(或酉矩阵)$Q$ 使得 $A=P\begin{bmatrix}B&0\\0&0\end{bmatrix}Q$,其中 $B$ 为 $r$ 阶可逆矩阵。那么,各广义逆的通解形式如下表所示:

\begin{table}[H]
    \centering
    \begin{tabular}{ll}
    \toprule
    \textbf{广义逆} & \textbf{通解} \\ \midrule
    $X \in A\{1\}$ & $\exists\ C,D,E,\quad\;X=Q^{-1}\begin{bmatrix}B^{-1}&C\\D&E\end{bmatrix}P^{-1}$ \\
    $X \in A\{1,2\}$ & $\exists\ C,D,\quad X=Q^{-1}\begin{bmatrix}B^{-1}&C\\D&DBC\end{bmatrix}P^{-1}$ \\
    $X \in A\{1,3\}$ & $\exists\ D,E,\quad\;X=Q^{-1}\begin{bmatrix}B^{-1}&0\\D&E\end{bmatrix}P^{-1}$ \\
    $X \in A\{1,4\}$ & $\exists\ C,E,\quad\;X=Q^{-1}\begin{bmatrix}B^{-1}&C\\0&E\end{bmatrix}P^{-1}$ \\
    $X \in A\{1,2,3\}$ & $\exists\ D,\quad\;X=Q^{-1}\begin{bmatrix}B^{-1}&0\\D&0\end{bmatrix}P^{-1}$ \\
    $X \in A\{1,2,4\}$ & $\exists\ C,\quad\;X=Q^{-1}\begin{bmatrix}B^{-1}&C\\0&0\end{bmatrix}P^{-1}$ \\
    $X \in A\{1,3,4\}$ & $\exists\ E,\quad\;X=Q^{-1}\begin{bmatrix}B^{-1}&0\\0&E\end{bmatrix}P^{-1}$ \\
    $X \in A\{1,2,3,4\}$ & $X=Q^{-1}\begin{bmatrix}B^{-1}&0\\0&0\end{bmatrix}P^{-1}$ \\ \bottomrule
    \end{tabular}
\end{table}

% |       广义逆        |                             通解                             |
% | :-----------------: | :----------------------------------------------------------: |
% |    $X\in A\{1\}$    | $\exists\ C,D,E,\quad\;X=Q^{-1}\begin{bmatrix}B^{-1}&C\\D&E\end{bmatrix}P^{-1}$ |
% |   $X\in A\{1,2\}$   | $\exists\ C,D,\quad X=Q^{-1}\begin{bmatrix}B^{-1}&C\\D&DBC\end{bmatrix}P^{-1}$ |
% |   $X\in A\{1,3\}$   | $\exists\ D,E,\quad\;X=Q^{-1}\begin{bmatrix}B^{-1}&0\\D&E\end{bmatrix}P^{-1}$ |
% |   $X\in A\{1,4\}$   | $\exists\ C,E,\quad\;X=Q^{-1}\begin{bmatrix}B^{-1}&C\\0&E\end{bmatrix}P^{-1}$ |
% |  $X\in A\{1,2,3\}$  | $\exists\ D,\quad\;X=Q^{-1}\begin{bmatrix}B^{-1}&0\\D&0\end{bmatrix}P^{-1}$ |
% |  $X\in A\{1,2,4\}$  | $\exists\ C,\quad\;X=Q^{-1}\begin{bmatrix}B^{-1}&C\\0&0\end{bmatrix}P^{-1}$ |
% |  $X\in A\{1,3,4\}$  | $\exists\ E,\quad\;X=Q^{-1}\begin{bmatrix}B^{-1}&0\\0&E\end{bmatrix}P^{-1}$ |
% | $X\in A\{1,2,3,4\}$ |  $X=Q^{-1}\begin{bmatrix}B^{-1}&0\\0&0\end{bmatrix}P^{-1}$   |

% 如果 $P,Q$ 是酉矩阵,则 $Q^{-1},P^{-1}$ 写为 $Q^H,P^H$ 即可。
\begin{remark}
与通解形式相对应,称定义 \ref{def:moore-penrose} 中给出的广义逆的定义为方程形式。
做证明时,有时使用方程形式不容易想到思路,而使用通解只需要无脑计算即可。
\end{remark}
\begin{remark}
应用奇异值分解可以使得 $A=U\begin{bmatrix}\Sigma&0\\0&0\end{bmatrix}V^H$,这是上面的特殊情形,因此\textbf{做证明题时直接奇异值分解}就行了。
但是做计算题时,奇异值分解比较麻烦,所以我们不必追求让 $B$ 成为对角矩阵,只需要使得 $B$ 可逆即可。可以通过如下方式计算 $P,Q,B$:
\[
    \begin{bmatrix}A&I_m\\I_n&0\end{bmatrix}\xrightarrow[\text{列变换}]{\text{行变换}}\begin{bmatrix}\begin{bmatrix}B&0\\0&0\end{bmatrix}&P\\Q&0\end{bmatrix}
\]
也可以通过 QR 分解做:首先使用列置换矩阵 $P$ 使得 $AP$ 前 $r$ 列线性无关,则对 $AP$ 做 QR 分解得 $AP=Q_1\begin{bmatrix}R_1&G\\0&0\end{bmatrix}$,其中 $R_1$ 为上三角矩阵。再对 $\begin{bmatrix}R_1^H&0\\G^H&0\end{bmatrix}$ 做 QR 分解得 $\begin{bmatrix}R_1^H&0\\G^H&0\end{bmatrix}=Q_2\begin{bmatrix}R_2&0\\0&0\end{bmatrix}$,其中 $R_2$ 为上三角矩阵。于是:
\[
    AP=Q_1\begin{bmatrix}R_2^H&0\\0&0\end{bmatrix}Q_2^H\implies A=Q_1\begin{bmatrix}R_2^H&0\\0&0\end{bmatrix}Q_2^HP^T
\]
这样得到的 $B$ 是一个下三角矩阵。
\end{remark}

\begin{definition}[广义逆的等价定义]
设 $A\in\mathbb C^{m\times n}$,若矩阵 $X\in\mathbb C^{n\times m}$ 满足 $AX=P_{R(A)},\,XA=P_{R(X)}$,其中 $P_L$ 是空间 $L$ 上的正交投影矩阵,则称 $X$ 为 $A$ 的 Moore 广义逆矩阵。
\end{definition}

\begin{theorem}
Moore 广义逆矩阵和 Penrose 广义逆矩阵是等价的。
\end{theorem}


\subsection{广义逆矩阵的计算方法}

\begin{theorem}[利用 Hermite 标准形计算 1-逆和 1,2-逆]
设 $A\in\mathbb C_r^{m\times n}$,又设 $Q\in\mathbb C_m^{m\times m}$ 和 $P\in\mathbb C_n^{n\times n}$ 使得
\[
    QAP=\begin{bmatrix}I_r&K\\0&0\end{bmatrix}
\]
成立($P$ 可以只是一个列置换矩阵),则对任意 $L\in\mathbb C^{(n-r)\times (m-r)}$,$n\times m$ 矩阵
\[
    X=P\begin{bmatrix}I_r&0\\0&L\end{bmatrix}Q
\]
是 $A$ 的 1-逆,若令 $L=0$ 则 $X$ 是 $A$ 的 1,2-逆。
\end{theorem}

\begin{remark}
理论基础显然是上一节的 1-逆和 1,2-逆的通解形式,不过这里不要求 $K=0$,相应代价就是通解中的 $C,D$ 这里必须是零,也就是说得到的是一种特解。
\end{remark}

\begin{theorem}[满秩分解求广义逆矩阵]
设 $A\in\mathbb C_r^{m\times n}$ 的满秩分解为 $A=FG$,则:
\begin{enumerate}
    \item $G^{(i)}F^{(1)}\in A\{i\},\,i=1,2,4$.
    \item $G^{(1)}F^{(i)}\in A\{i\},\,i=1,2,3$.
    \item $G^{(1)}F^{+}\in A\{1,2,3\}$,$G^{+}F^{(1)}\in A\{1,2,4\}$.
    \item $A^+=G^+F^{(1,3)}=G^{(1,4)}F^+$.
    \item $A^+=G^+F^+=G^H(GG^H)^{-1}(F^HF)^{-1}F^H$.
\end{enumerate}
\end{theorem}

\begin{remark}
由于 $F$ 列满秩、$G$ 行满秩,根据上一节 1-逆的性质 7,有 $F^{(1)}F=GG^{(1)}=I_r$.  利用这一点,由定义即可验证 1 与 2。
3 和 4 可由 1 和 2 得到。5 利用了上一节关于行满秩与列满秩的矩阵的 $A^+$ 公式。
\end{remark}

\begin{theorem}[Zlobec 公式计算 $A^+$]
\[
    A^+=A^H(A^HAA^H)^{(1)}A^H
\]
\end{theorem}
\begin{proof}[证明(利用通解形式)]
设 $A=U\begin{bmatrix}\Sigma&0\\0&0\end{bmatrix}V^H$,则:
\[
    A^HAA^H=V\begin{bmatrix}\Sigma&0\\0&0\end{bmatrix}U^HU\begin{bmatrix}\Sigma&0\\0&0\end{bmatrix}V^HV\begin{bmatrix}\Sigma&0\\0&0\end{bmatrix}U^H=V\begin{bmatrix}\Sigma^3&0\\0&0\end{bmatrix}U^H
\]
于是:
\[
    (A^HAA^H)^{(1)}=U\begin{bmatrix}\Sigma^{-3}&C\\D&E\end{bmatrix}V^H
\]
因此:
\[
    A^H(A^HAA^H)^{(1)}A^H=V\begin{bmatrix}\Sigma&0\\0&0\end{bmatrix}U^HU\begin{bmatrix}\Sigma^{-3}&C\\D&E\end{bmatrix}V^HV\begin{bmatrix}\Sigma&0\\0&0\end{bmatrix}U^H=V\begin{bmatrix}\Sigma^{-1}&0\\0&0\end{bmatrix}U^H
\]
这就是 $A^+$ 的通解形式。
\end{proof}

如果用方程形式去证明,需要一些引理的帮助,显得非常麻烦,这里不做叙述。不过这些引理中有一些值得注意,写在下面。

\begin{theorem}
设 $A\in\mathbb C_r^{m\times n},\,U\in\mathbb C^{n\times p},\,V\in\mathbb C^{q\times m}$,则
\[
    U(VAU)^{(1)}V\in A\{1\}\iff \text{rank}(VAU)=\text{rank}(A)
\]
\end{theorem}
\begin{remark}
这个定理是 1-逆的性质 8 的扩展。回顾性质 8(做了变量替换):
\begin{align*}
    &AU(AU)^{(1)}A=A\iff\text{rank}(AU)=\text{rank}(A)\\
    &A(VA)^{(1)}VA=A\iff\text{rank}(VA)=\text{rank}(A)
\end{align*}
第一条是在 $A$ 的右边乘上 $U$,第二条是在 $A$ 的左边乘上 $V$,而这个定理左右同时乘了 $V$ 和 $U$.
\end{remark}
\begin{proof}
充分性。由 $\text{rank}(VAU)=\text{rank}(A)$ 知 $R(VAU)=R(AU)=R(A),\,N(VAU)=N(VA)=N(A)$. 故存在 $X,Y$ 使得 $A=AUX=YVA$,于是 $AU(VAU)^{(1)}VA=YVAU(VAU)^{(1)}VAUX=YVAUX=YVA=A$.

必要性:{\color{red}{???TODO}}
\end{proof}

\begin{theorem}
对任意矩阵 $A$,满足 $X\in A\{1,2\}$ 和 $R(X)=R(A^H),\,N(X)=N(A^H)$ 的唯一矩阵为 $A^+$.
\end{theorem}

\begin{theorem}[Greville 公式计算 $A^+$]
Greville 公式是计算 $A^+$ 的**增量**公式。
设 $A\in\mathbb C^{m\times n}$,记 $a_k$ 为 $A$ 的第 $k$ 列,$A_k$ 为 $A$ 的前 $k$ 列构成的子矩阵;又记:
\[
    d_k=A^+_{k-1}a_k,\quad c_k=a_k-A_{k-1}d_k
\]
则:
\[
    A^+_k=\begin{bmatrix}A^+_{k-1}-d_kb_k^H\\b_k^H\end{bmatrix},\quad\text{where}\quad b_k^H=\begin{cases}c_k^+,&c_k\neq 0\\(1+d_k^Hd_k)^{-1}d_k^HA^+_{k-1},&c_k=0\end{cases}
\]
\end{theorem}


\subsection{广义逆矩阵与线性方程组求解}

对于方程组 $Ax=b$,如果 $A$ 非奇异,则 $x=A^{-1}b$ 是唯一解。而在其他情况下,我们希望得到类似的结果。
\begin{itemize}
    \item 如果方程组相容,且其解有无数多个,我们希望求\textbf{极小范数解},即 $\min_{Ax=b}\Vert x\Vert$;
    \item 如果方程组不相容,即无解,那么我们希望求矛盾方程组的\textbf{最小二乘解},即 $\min \Vert Ax-b\Vert$;
    \item 一般而言,最小二乘解也不唯一,因此我们希望求\textbf{极小范数最小二乘解},即 $\min_{\min\Vert Ax-b\Vert}\Vert x\Vert$.
\end{itemize}
\begin{com}
本节所用范数均为 2 范数。
\end{com}

\begin{theorem}[线性方程组的相容性、通解与 1-逆]
\label{thm:compatible}
设 $A\in\mathbb C^{m\times n},\,B\in\mathbb C^{p\times q},\,D\in\mathbb C^{m\times q}$,则矩阵方程 $AXB=D$ 相容的充要条件是:
\[
    AA^{(1)}DB^{(1)}B=D
\]
当方程相容时,通解为:
\[
    X=A^{(1)}DB^{(1)}+Y-A^{(1)}AYBB^{(1)}
\]
其中 $Y\in\mathbb C^{n\times p}$ 为任意矩阵。
\end{theorem}
\begin{proof}
充分性,取 $X=A^{(1)}DB^{(1)}$ 即可;必要性,若 $AXB=D$ 有解,则 $D=AXB=AA^{(1)}AXBB^{(1)}B=AA^{(1)}DB^{(1)}B$.

对于通解,首先显然 $X=A^{(1)}DB^{(1)}+Y-A^{(1)}AYBB^{(1)}$ 是方程的解;其次,若 $X$ 是方程的解,则取 $Y=X$ 即可写作通解形式。
\end{proof}

\begin{corollary}
设 $A\in\mathbb C^{m\times n}$,取 $A^{(1)}\in A\{1\}$,则:
\[
    A\{1\}=\{A^{(1)}+Z-A^{(1)}AZAA^{(1)}\mid Z\in\mathbb C^{n\times m}\}
\]
\end{corollary}
\begin{proof}
任意 $X\in A\{1\}$ 满足矩阵方程 $AXA=A$,代入上述定理的通解形式得:
\begin{align*}
    X&=A^{(1)}AA^{(1)}+Y-A^{(1)}AYAA^{(1)}\\
    &=A^{(1)}AA^{(1)}+A^{(1)}+Z-A^{(1)}A(A^{(1)}+Z)AA^{(1)}&Y=A^{(1)}+Z\\
    &=A^{(1)}+Z+A^{(1)}AA^{(1)}-A^{(1)}AA^{(1)}AA^{(1)}-A^{(1)}AZAA^{(1)}\\
    &=A^{(1)}+Z-A^{(1)}AZAA^{(1)}
\end{align*}
\end{proof}

\begin{theorem}
\label{thm:compatible2}
线性方程组 $Ax=b$ 相容的充要条件是:
\[
    AA^{(1)}b=b
\]
通解为:
\[
    x=A^{(1)}b+(I-A^{(1)}A)y
\]
其中 $y\in\mathbb C^{n}$ 为任意向量。
\end{theorem}
\begin{proof}
在定理 \ref{thm:compatible} 中取 $X=x,\,B=I,\,D=b$ 即可。
\end{proof}

定理 \ref{thm:compatible2} 是给定 $A^{(1)}$ 后求解方程的解,反过来,利用方程的解也可以给出 $A^{(1)}$.

\begin{theorem}
若对于任意满足 $Ax=b$ 相容的 $b$,$x=Xb$ 都是解,则 $X\in A\{1\}$.
\end{theorem}
\begin{proof}
考虑 $Ax=a_i$,其中 $a_i$ 为 $A$ 的列,由于 $x=Xa_i$ 是方程的解,所以 $AXa_i=a_i$,于是 $AXA=A$,故 $X\in A\{1\}$.
\end{proof}

\begin{lemma}[极小范数解]
相容方程组 $Ax=b$ 的极小范数解唯一,且这个唯一解在 $R(A^H)$ 中。
\end{lemma}
\begin{proof}
由于 $R(A^H)=N(A)^\perp$,所以设 $x=y+z$,其中 $y=P_{R(A^H)}x\in R(A^H),\,z=P_{N(A)}x\in N(A)$,于是:
\[
    \Vert x\Vert^2=\Vert y+z\Vert^2=\Vert y\Vert^2+\Vert z\Vert^2\geq \Vert y\Vert^2
\]
由于 $Az=0\implies Ay=b$,即 $y$ 也是方程的解,所以为了让 $x$ 是极小范数解,只能是 $z=0$,因此 $x=y\in R(A^H)$.

唯一性。设 $x'\in R(A^H)$ 且 $Ax'=b$,则 $A(x-x')=0$,即 $x-x'\in N(A)=R^{\perp}(A^H)$.  又 $x-x'\in R(A^H)$,故 $x-x'=0$.
\end{proof}

\begin{lemma}
集合 $A\{1,4\}$ 由矩阵方程 $XA=A^{(1,4)}A$ 的所有解组成,其中 $A^{(1,4)}\in A\{1,4\}$.
\end{lemma}
\begin{proof}
$AXA=AA^{(1,4)}A=A$,所以 $X\in A\{1\}$;$(XA)^H=(A^{(1,4)}A)^H=A^{(1,4)}A=XA$,所以 $X\in A\{4\}$.  综上 $X\in A\{1,4\}$.

另一方面,若 $X\in A\{1,4\}$,则
\begin{align*}
    A^{(1,4)}A&=A^{(1,4)}AXA=(A^{(1,4)}A)^H(XA)^H=A^H(A^{(1,4)})^HA^HX^H\\
    &=(AA^{(1,4)}A)^HX^H=A^HX^H=XA
\end{align*}
即 $X$ 是方程的解。
\end{proof}

\begin{remark}
该定理说明尽管 $A^{(1,4)}$ 不唯一,但是 $A^{(1,4)}A$ 唯一。
\end{remark}

\begin{corollary}
$A^{(1,4)}A=P_{R(A^H)}$.
\end{corollary}

\begin{theorem}
设 $A\in\mathbb C^{m\times n},\,A^{(1,4)}\in A\{1,4\}$,则:
\[
    A\{1,4\}=\{A^{(1,4)}+Z(I-AA^{(1,4)})\mid Z\in\mathbb C^{n\times m}\}
\]
\end{theorem}
\begin{proof}
根据引理,任意 $X\in A\{1,4\}$ 满足方程 $XA=A^{(1,4)}A$,代入通解形式得:
\begin{align*}
    X&=A^{(1,4)}AA^{(1,4)}+Y-YAA^{(1,4)}\\
    &=A^{(1,4)}AA^{(1,4)}+A^{(1,4)}+Z-(A^{(1,4)}+Z)AA^{(1,4)}&Y=A^{(1,4)}+Z\\
    &=A^{(1,4)}+Z+A^{(1,4)}AA^{(1,4)}-(A^{(1,4)}+Z)AA^{(1,4)}\\
    &=A^{(1,4)}+Z(I-AA^{(1,4)})
\end{align*}
\end{proof}

\begin{theorem}[相容方程组的极小范数解与 1,4-逆]
设 $Ax=b$ 相容,则 $x=A^{(1,4)}b$ 为极小范数解;反之,若对于任意 $b\in R(A)$,$x=Xb$ 都是极小范数解,则 $X\in A\{1,4\}$.
\end{theorem}
\begin{proof}
由第一节定理知 $x=A^{(1,4)}b$ 一定是解。设 $Au=b$,则 $x=A^{(1,4)}b=A^{(1,4)}Au=(A^{(1,4)}A)^Hu=A^H(A^{(1,4)})^Hu\in R(A^H)$,于是根据本节引理知 $x$ 为唯一极小范数解。

反之,考虑 $Ax=a_i$,由于 $x=Xa_i$ 是方程的极小范数解,所以 $Xa_i=A^{(1,4)}a_i$,故 $XA=A^{(1,4)}A$,根据引理知 $X\in A\{1,4\}$.
\end{proof}

\begin{lemma}
集合 $A\{1,3\}$ 由矩阵方程 $AX=AA^{(1,3)}$ 的所有解组成,其中 $A^{(1,3)}\in A\{1,3\}$.
\end{lemma}
\begin{proof}
$AXA=AA^{(1,3)}A=A$,故 $X\in A\{1\}$;$(AX)^H=(AA^{(1,3)})^H=AA^{(1,3)}=AX$,故 $X\in A\{3\}$. 综上 $X\in A\{1,3\}$.

另一方面,若 $X\in A\{1,3\}$,则:
\begin{align*}
    AA^{(1,3)}&=AXAA^{(1,3)}=(AX)^H(AA^{(1,3)})^H=X^HA^H(A^{(1,3)})^HA^H\\
    &=X^H(AA^{(1,3)}A)^H=X^HA^H=AX
\end{align*}
即 $X$ 是方程的解。
\end{proof}

\begin{remark}
该定理说明尽管 $A^{(1,3)}$ 不唯一,但是 $AA^{(1,3)}$ 唯一。
\end{remark}

\begin{corollary}
$AA^{(1,3)}=P_{R(A)}$.
\end{corollary}

\begin{theorem}
设 $A\in\mathbb C^{m\times n},\,A^{(1,3)}\in A\{1,3\}$,则:
\[
    A\{1,3\}=\{A^{(1,3)}+(I-A^{(1,3)}A)Z\mid Z\in\mathbb C^{n\times m}\}
\]
\end{theorem}
\begin{proof}
根据引理,任意 $X\in A\{1,3\}$ 满足方程 $AX=AA^{(1,3)}$,代入通解形式得:
\begin{align*}
    X&=A^{(1,3)}AA^{(1,3)}+Y-A^{(1,3)}AY\\
    &=A^{(1,3)}AA^{(1,3)}+A^{(1,3)}+Z-A^{(1,3)}A(A^{(1,3)}+Z)&Y=A^{(1,3)}+Z\\
    &=A^{(1,3)}+Z+A^{(1,3)}AA^{(1,3)}-A^{(1,3)}A(A^{(1,3)}+Z)\\
    &=A^{(1,3)}+(I-A^{(1,3)}A)Z
\end{align*}
\end{proof}

\begin{theorem}[矛盾方程组的最小二乘解与 1,3-逆]
设有方程 $Ax=b$,则 $x=A^{(1,3)}b$ 为最小二乘解;反之,若对于任意 $b$,$x=Xb$ 都是最小二乘解,则 $X\in A\{1,3\}$.
\end{theorem}

\begin{theorem}[法方程]
$x$ 是方程组 $Ax=b$ 的最小二乘解的充要条件为:
\[
    A^HAx=A^Hb
\]
\end{theorem}

\begin{theorem}[矛盾方程组的极小范数最小二乘解与 $A^+$]
$x=A^+b$ 是方程组 $Ax=b$ 的唯一极小范数最小二乘解。反之,若对所有 $b$,$x=Xb$ 都是方程 $Ax=b$ 的极小范数最小二乘解,则 $X=A^+$.
\end{theorem}

\begin{theorem}
若矩阵方程 $AXB=D$ 不相容,则其极小范数最小二乘解,即满足 $\min_{\min \Vert AXB-D\Vert}\Vert X\Vert$ 的唯一解为 $X=A^+DB^+$.
\end{theorem}
\begin{proof}
方程两边同时行拉直:
\[
    \overline{\text{vec}}(AXB)=\overline{\text{vec}}(D)\implies (A\otimes B^T)\overline{\text{vec}}(X)=\overline{\text{vec}}(D)
\]
其极小范数最小二乘解为:
\[
    \overline{\text{vec}}(X)=(A\otimes B^T)^+\overline{\text{vec}}(D)=(A^+\otimes (B^T)^+)\overline{\text{vec}}(D)=(A^+\otimes (B^+)^T)\overline{\text{vec}}(D)
\]
于是反过来应用拉直算子得 $X=A^+DB^+$.
\end{proof}

\begin{com}
上述过程应用了 $(A\otimes B)^+=A^+\otimes B^+$ 的结论,该结论可以通过定义验证。
\end{com}


\vskip 6pt \noindent\textbf{小结}:对于 $Ax=b$,有:
\begin{itemize}
    \item $Ax=b$ 相容的充要条件是 $AA^{(1)}b=b$
    \item 若 $Ax=b$ 相容,则通解为 $x=A^{(1)}b+(I-A^{(1)}A)y$
    \item 若 $Ax=b$ 相容,则极小范数解为 $x=A^{(1,4)}b$
    \item 若 $Ax=b$ 不相容,则最小二乘解为 $x=A^{(1,3)}b$
    \item 若 $Ax=b$ 不相容,则极小范数最小二乘解为 $x=A^+b$
\end{itemize}

对于 $AXB=D$,有:
\begin{itemize}
    \item $AXB=D$ 相容的充要条件是 $AA^{(1)}DB^{(1)}B=D$
    \item 若 $AXB=D$ 相容,则通解为 $X=A^{(1)}DB^{(1)}+Y-A^{(1)}AYBB^{(1)}$
    \item 若 $AXB=D$ 不相容,则极小范数最小二乘解为 $X=A^+DB^+$
\end{itemize}

\vskip 6pt \noindent\textbf{广义逆的集合表示}

\begin{itemize}
    \item $A\{1\}=\{X\mid AXA=A\}=\{A^{(1)}+Z-A^{(1)}AZAA^{(1)}\mid Z\in\mathbb C^{n\times m}\}$
    \item $A\{1,3\}=\{X\mid AX=AA^{(1,3)}\}=\{A^{(1,3)}+(I-A^{(1,3)}A)Z\mid Z\in\mathbb C^{n\times m}\}$
    \item $A\{1,4\}=\{X\mid XA=A^{(1,4)}A\}=\{A^{(1,4)}+Z(I-AA^{(1,4)})\mid Z\in\mathbb C^{n\times m}\}$
    \item $A\{1,2\}=\{X\mid \text{rank}(X)=\text{rank}(A),\,X\in A\{1\}\}$
\end{itemize}
 \newpage

\appendix
\section{常见随机变量}
\label{sec:random-variables}

本节总结常见的随机变量及其期望、方差、矩母函数和性质,其中离散随机变量包括伯努利、二项、泊松、几何和超几何随机变量,连续随机变量包括均匀、指数和正态随机变量。


\paragraph{伯努利随机变量}

\begin{itemize}[itemsep=1ex]
    \item 记号:$X\sim B(1,p)$
    \item 实例:抛掷一枚硬币,硬币向上概率为 $p$,$X$ 为是否向上。
    \item PMF:$p_X(k)=\begin{cases}p,&k=1\\1-p,&k=0\end{cases}$
    \item 期望:$\E X=p$
    \item 方差:$\var X=p(1-p)$
    \item 矩母函数:$M_X(s)=1-p+pe^s$
\end{itemize}

\begin{proof}[矩母函数的推导]
\[
M_X(s)=\E[e^{sX}]=(1-p)\cdot e^{0}+p\cdot e^{s}=1-p+pe^s
\]
\end{proof}

\paragraph{二项随机变量}

\begin{itemize}[itemsep=1ex]
    \item 记号:$X\sim B(n,p)$
    \item 实例:抛掷 $n$ 枚硬币,每枚硬币向上概率均为 $p$,$X$ 为向上次数。
    \item PMF:$p_X(k)=\displaystyle\binom{n}{k}p^k(1-p)^{n-k},\quad k=0,1,2,\ldots,n$
    \item 期望:$\E X=np$
    \item 方差:$\var X=np(1-p)$
    \item 矩母函数:$M_X(s)=(1-p+pe^s)^n$
\end{itemize}

\begin{proof}[期望的推导]
\begin{align*}
\E X&=\sum_{k=0}^{n}k\binom{n}{k}p^k(1-p)^{n-k}=\sum_{k=1}^nn\binom{n-1}{k-1}p^k(1-p)^{n-k}\\
&=np\sum_{k=0}^{n-1}\binom{n-1}{k}p^k(1-p)^{n-1-k}=np(p+1-p)^{n-1}=np
\end{align*}
其中用到了恒等式 $\displaystyle\binom{n}{k}=\binom{n-1}{k-1}\frac{n}{k}$ 和二项式定理。
\end{proof}
\begin{proof}[二阶矩的推导]
\begin{align*}
\E X^2&=\sum_{k=0}^{n}k^2\binom{n}{k}p^k(1-p)^{n-k}=\sum_{k=1}^nnk\binom{n-1}{k-1}p^k(1-p)^{n-k}\\
&=\sum_{k=1}^nn\binom{n-1}{k-1}p^k(1-p)^{n-k}+\sum_{k=1}^nn(k-1)\binom{n-1}{k-1}p^k(1-p)^{n-k}\\
&=np+np\sum_{k=0}^{n-1}k\binom{n-1}{k}p^{k}(1-p)^{n-1-k}=np+np(n-1)p
\end{align*}
\end{proof}
\begin{proof}[方差的推导]
\[
\var X=\E X^2-(\E X)^2=np+n(n-1)p^2-n^2p^2=np-np^2=np(1-p)
\]
\end{proof}
\begin{proof}[矩母函数的推导]
\[
M_X(s)=\E[e^{sX}]=\sum_{k=0}^n\binom{n}{k}p^k(1-p)^{n-k}e^{sk}=(1-p+pe^s)^n
\]
\end{proof}

\paragraph{泊松随机变量}

\begin{itemize}[itemsep=1ex]
    \item 记号:$X\sim P(\lambda)$
    \item 实例:一个城市一天中发生车祸次数。
    \item PMF:$p_X(k)=e^{-\lambda}\dfrac{\lambda^k}{k!},\quad k=0,1,\ldots$
    \item 期望:$\E X=\lambda$
    \item 方差:$\var X=\lambda$
    \item 矩母函数:$M_X(s)=e^{\lambda(e^s-1)}$
    \item 性质:取 $\lambda=np$,则当 $n\to\infty$ 时,泊松分布近似二项分布。
\end{itemize}

\begin{proof}[期望的推导]
\[
\E X=\sum_{k=0}^\infty ke^{-\lambda}\frac{\lambda^k}{k!}=e^{-\lambda}\sum_{k=1}^{\infty}\frac{\lambda^k}{(k-1)!}=\lambda e^{-\lambda}\sum_{k=0}^{\infty}\frac{\lambda^k}{k!}=\lambda
\]
其中用到了 $e^x$ 的泰勒展开。
\end{proof}
\begin{proof}[二阶矩的推导]
\begin{align*}
\E X^2&=\sum_{k=0}^\infty k^2e^{-\lambda}\frac{\lambda^k}{k!}=e^{-\lambda}\sum_{k=1}^{\infty}k\frac{\lambda^k}{(k-1)!}\\
&=e^{-\lambda}\sum_{k=1}^{\infty}\frac{\lambda^k}{(k-1)!}+e^{-\lambda}\sum_{k=2}^{\infty}\frac{\lambda^k}{(k-2)!}\\
&=e^{-\lambda}\lambda e^\lambda+e^{-\lambda}\lambda^2 e^\lambda=\lambda+\lambda^2
\end{align*}
\end{proof}
\begin{proof}[方差的推导]
\[
\var X=\E X^2-(\E X)^2=\lambda+\lambda^2-\lambda^2=\lambda
\]
\end{proof}
\begin{proof}[矩母函数的推导]
\[
M_X(s)=\E[e^{sX}]=\sum_{k=0}^\infty e^{-\lambda}\frac{\lambda^k}{k!}e^{sk}=e^{-\lambda}\sum_{k=0}^\infty \frac{(\lambda e^s)^k}{k!}=e^{\lambda(e^s-1)}
\]
\end{proof}
\begin{proof}[证明泊松分布近似二项分布]
取 $\lambda=np$,则:
\begin{align*}
\lim_{n\to\infty}\binom{n}{k}p^k(1-p)^{nk}&=\lim_{n\to\infty}\frac{n^{\underline{k}}}{k!}\cdot\frac{\lambda^k}{n^k}\left(1-\frac{\lambda}{n}\right)^{n-k}\\
&=\frac{\lambda^k}{k!}\lim_{n\to\infty}\frac{n^{\underline{k}}}{n^k}\cdot\left(1-\frac{\lambda}{n}\right)^{n-k}\\
&=\frac{\lambda^k}{k!}\cdot 1\cdot\lim_{n\to\infty}\left[\left(1-\frac{\lambda}{n}\right)^{-\frac{n}{\lambda}}\right]^{-\frac{\lambda(n-k)}{n}}\\
&=\frac{\lambda^k}{k!}e^{-\lambda}
\end{align*}
\end{proof}

\paragraph{几何随机变量}

\begin{itemize}[itemsep=1ex]
    \item 记号:$X\sim G(p)$
    \item 实例:抛掷一枚硬币直至向上,硬币向上概率为 $p$,$X$ 为抛掷次数。
    \item PMF:$p_X(k)=(1-p)^{k-1}p,\quad k=1,2,\ldots$
    \item 期望:$\E X=\dfrac{1}{p}$
    \item 方差:$\var X=\dfrac{1-p}{p^2}$
    \item 矩母函数:$M_X(s)=\dfrac{pe^s}{1-(1-p)e^s}$
    \item 无记忆性:$\Pb(X>n+m\vert X>n)=\Pb(X>m)$
\end{itemize}

\begin{proof}[期望的推导]
设:
\[
f(x)=\sum_{k=1}^\infty k x^{k-1}=\sum_{k=1}^\infty (x^k)'=\left(\sum_{k=1}^\infty x^k\right)'=\left(\frac{x}{1-x}\right)'=\frac{1}{(1-x)^2}
\]
则:
\[
\E X=\sum_{k=1}^\infty k(1-p)^{k-1}p=pf(1-p)=\frac{1}{p}
\]
\end{proof}
\begin{proof}[二阶矩的推导]
设:
\[
g(x)=\sum_{k=1}^\infty k^2x^{k-1}=\sum_{k=1}^\infty k(x^k)'=\left(\sum_{k=1}^\infty kx^k\right)'=(xf(x))'=\left(\frac{x}{(1-x)^2}\right)'=\frac{1+x}{(1-x)^3}
\]
则:
\[
\E X^2=\sum_{k=1}^\infty k^2(1-p)^{k-1}p=pg(1-p)=p\cdot\frac{2-p}{p^3}=\frac{2-p}{p^2}
\]
\end{proof}
\begin{proof}[方差的推导]
\[
\var X=\E X^2-(\E X)^2=\frac{2-p}{p^2}-\frac{1}{p^2}=\frac{1-p}{p^2}
\]
\end{proof}
\begin{proof}[矩母函数的推导]
\[
M_X(s)=\E[e^{sX}]=\sum_{k=1}^\infty(1-p)^{k-1}pe^{sk}=pe^s\sum_{k=0}^\infty((1-p)e^s)^{k}=\frac{pe^s}{1-(1-p)e^s}
\]
\end{proof}
\begin{proof}[证明无记忆性]
首先计算尾概率:
\[
\Pb(X>n)=\sum_{k=n+1}^\infty(1-p)^{k-1}p=p\frac{(1-p)^n}{1-(1-p)}=(1-p)^n
\]
于是:
\[
\Pb(X>n+m\vert X>n)=\frac{P(X>n+m)}{P(X>n)}=\frac{(1-p)^{n+m}}{(1-p)^n}=(1-p)^m=\Pb(X>m)
\]
\end{proof}

\paragraph{超几何随机变量}

\begin{itemize}[itemsep=1ex]
    \item 实例:一盒内有 $N$ 个球,其中 $M$ 个白球 $N-M$ 个黑球,从中无放回地取 $n$ 个球且每次取球独立,$X$ 表示取出白球个数。
    \item PMF:$p_X(k)=\dfrac{\binom{M}{k}\binom{N-M}{n-k}}{\binom{N}{n}},\quad k=0,1,\ldots,n$
    \item 期望:$\E X=\dfrac{nM}{N}$
    \item 方差:$\var X=\dfrac{nM}{N}\left(1-\dfrac{M}{N}\right)\dfrac{N-n}{N-1}$
    \item 性质:当 $N\to\infty$ 时,超几何分布近似二项分布。
\end{itemize}

\begin{proof}[期望的推导]
\[
\E X=\sum_kk\frac{\binom{M}{k}\binom{N-M}{n-k}}{\binom{N}{n}}=\frac{1}{\binom{N}{n}}\sum_k M\binom{M-1}{k-1}\binom{N-M}{n-k}=\frac{M}{\binom{N}{n}}\binom{N-1}{n-1}=\frac{nM}{N}
\]
其中用到了恒等式 $\displaystyle\binom{N}{n}=\binom{N-1}{n-1}\frac{N}{n}$ 和范德蒙德卷积式 $\displaystyle\sum_k\binom{r}{k}\binom{s}{n-k}=\binom{r+s}{n}$.
\end{proof}
\begin{proof}[二阶矩的推导]
\begin{align*}
\E X^2&=\sum_kk^2\frac{\binom{M}{k}\binom{N-M}{n-k}}{\binom{N}{n}}=\frac{M}{\binom{N}{n}}\sum_k k\binom{M-1}{k-1}\binom{N-M}{n-k}\\
&=\frac{M}{\binom{N}{n}}\sum_k\binom{M-1}{k-1}\binom{N-M}{n-k}+\frac{M}{\binom{N}{n}}\sum_k (k-1)\binom{M-1}{k-1}\binom{N-M}{n-k}\\
&=\frac{M}{\binom{N}{n}}\binom{N-1}{n-1}+\frac{M}{\binom{N}{n}}(M-1)\binom{N-2}{n-2}=\frac{nM}{N}+\frac{M(M-1)n(n-1)}{N(N-1)}
\end{align*}
\end{proof}
\begin{proof}[方差的推导]
\[
\var X=\E X^2-(\E X)^2=\frac{nM}{N}+\frac{M(M-1)n(n-1)}{N(N-1)}-\frac{n^2M^2}{N^2}=\frac{nM}{N}\left(1-\frac{M}{N}\right)\frac{N-n}{N-1}
\]
\end{proof}


\paragraph{均匀随机变量}

\begin{itemize}[itemsep=1ex]
    \item 记号:$X\sim U(a,b)$
    \item PDF:$f_X(x)=\dfrac{1}{b-a},\quad a\leqslant x\leqslant b$
    \item 期望:$\E X=\dfrac{a+b}{2}$
    \item 方差:$\var X=\dfrac{(b-a)^2}{12}$
    \item 矩母函数:$M_X(s)=\dfrac{1}{b-a}\cdot\dfrac{e^{sb}-e^{sa}}{s}$
\end{itemize}

\begin{proof}[矩母函数的推导]
\[
M_X(s)=\E[e^{sX}]=\int_a^b\frac{e^{sx}}{b-a}\mathrm dx=\frac{1}{b-a}\cdot\frac{1}{s}\int_{a/s}^{b/s}e^{sx}\mathrm d(sx)=\frac{1}{b-a}\cdot\frac{e^{sb}-e^{sa}}{s}
\]
\end{proof}

\paragraph{指数随机变量}

\begin{itemize}[itemsep=1ex]
    \item 记号:$X\sim E(\lambda)$
    \item PDF:$f_X(x)=\lambda e^{-\lambda x},\quad x\geqslant0$
    \item 期望:$\E X=\dfrac{1}{\lambda}$
    \item 方差:$\var X=\dfrac{1}{\lambda^2}$
    \item 矩母函数:$M_X(s)=\dfrac{\lambda}{\lambda-s}\;(s<\lambda)$
    \item 无记忆性:$\Pb(X>x+y\vert X>x)=\Pb(X>y)$
\end{itemize}

\begin{proof}[期望的推导]
\[
\E X=\int_0^{+\infty}x\lambda e^{-\lambda x}\mathrm dx=-\int_0^{+\infty}x\mathrm d e^{-\lambda x}=\int_0^{+\infty}e^{-\lambda x}\mathrm dx=-\frac{1}{\lambda}\left.e^{-\lambda x}\right|_{0}^{+\infty}=\frac{1}{\lambda}
\]
\end{proof}
\begin{proof}[二阶矩的推导]
\[
\E X^2=\int_0^{+\infty} x^2\lambda e^{-\lambda x}\mathrm dx=-\int_0^{+\infty}x^2\mathrm de^{-\lambda x}=2\int_0^{+\infty}x e^{-\lambda x}\mathrm dx=\frac{2}{\lambda^2}
\]
\end{proof}
\begin{proof}[方差的推导]
\[
\var X=\E X^2-(\E X)^2=\frac{2}{\lambda^2}-\frac{1}{\lambda^2}=\frac{1}{\lambda^2}
\]
\end{proof}
\begin{proof}[矩母函数的推导]
\[
M_X(s)=\E[e^{sX}]=\int_0^\infty\lambda e^{-\lambda x}e^{sx}\mathrm dx=\lambda\int_0^\infty e^{-(\lambda-s)x}\mathrm dx=\frac{\lambda}{\lambda-s}\quad(s<\lambda)
\]
\end{proof}
\begin{proof}[证明无记忆性]
首先计算尾概率:
\[
\Pb(X>x)=\int_x^{+\infty}\lambda e^{-\lambda t}\mathrm dt=\left.e^{-\lambda t}\right|_{+\infty}^x=e^{-\lambda x}
\]
于是:
\[
\Pb(X>x+y\vert X>x)=\frac{\Pb(X>x+y)}{\Pb(X>x)}=\frac{e^{-\lambda(x+y)}}{e^{-\lambda x}}=e^{-\lambda y}=\Pb(X>y)
\]
\end{proof}

\paragraph{正态随机变量}

\begin{itemize}[itemsep=1ex]
    \item 记号:$X\sim N(\mu,\sigma^2)$
    \item PDF:$f_X(x)=\dfrac{1}{\sqrt{2\pi}\sigma}\exp\left({-\dfrac{(x-\mu)^2}{2\sigma^2}}\right)$
    \item 期望:$\E X=\mu$
    \item 方差:$\var X=\sigma^2$
    \item 矩母函数:$M_X(s)=\exp\left({\dfrac{\sigma^2s^2}{2}+\mu s}\right)$
\end{itemize}

\begin{proof}[证明标准正态分布的归一性]
由于:
\begin{align*}
\left[\int_{-\infty}^{+\infty}\frac{1}{\sqrt{2\pi}}\exp\left({-\frac{x^2}{2}}\right)\mathrm dx\right]^2&=\frac{1}{2\pi}\int_{-\infty}^{+\infty}\int_{-\infty}^{+\infty}\exp\left({-\frac{x^2+y^2}{2}}\right)\mathrm dx\mathrm dy\\
&=\frac{1}{2\pi}\int_0^{2\pi}\mathrm d\theta\int_{0}^{+\infty}\exp\left({-\frac{r^2}{2}}\right)r\mathrm dr\\
&=\int_0^{+\infty}\exp\left({-\frac{r^2}{2}}\right)\mathrm d\left(\frac{r^2}{2}\right)\\
&=\left.\exp\left({-\frac{r^2}{2}}\right)\right|_{+\infty}^0=1
\end{align*}
故:
\[
\int_{-\infty}^{+\infty}\frac{1}{\sqrt{2\pi}}\exp\left({-\frac{x^2}{2}}\right)\mathrm dx=1
\]
\end{proof}
% \begin{proof}[矩母函数的推导]
% \begin{align*}
% M_X(s)=\E[e^{sX}]&=\int_{-\infty}^{+\infty}\frac{1}{\sqrt{2\pi}\sigma}e^{-\frac{(x-\mu)^2}{2\sigma^2}}e^{sx}\mathrm dx=\frac{1}{\sqrt{2\pi}\sigma}\int_{-\infty}^{+\infty}e^{-\frac{(x-\mu)^2}{2\sigma^2}+sx}\mathrm dx\\
% &=\frac{1}{\sqrt{2\pi}\sigma}\int_{-\infty}^{+\infty}e^{-\frac{(x-\mu-s\sigma^2)^2}{2\sigma^2}}\mathrm dx
% \end{align*}
% \end{proof}
 \newpage
\section{二元正态分布}
\label{sec:two-normal}

\begin{definition}[二元正态分布]
若随机变量 $X,Y$ 有如下联合概率密度函数:
\[
f_{X,Y}(x,y)=\frac{1}{2\pi\sigma_1\sigma_2\sqrt{1-\rho^2}}\exp\left[-\frac{1}{2(1-\rho^2)}\left(\frac{(x-\mu_1)^2}{\sigma_1^2}-\frac{2\rho(x-\mu_1)(y-\mu_2)}{\sigma_1\sigma_2}+\frac{(y-\mu_2)^2}{\sigma_2^2}\right)\right]
\]
则称 $X,Y$ 服从参数为 $\mu_1,\mu_2,\sigma_1,\sigma_2,\rho$ 的二元正态分布。
\end{definition}

\begin{com}
矩阵形式:设 $\mathbf X=\begin{pmatrix}X\\Y\end{pmatrix},\boldsymbol\mu=\begin{pmatrix}\mu_1\\\mu_2\end{pmatrix},\Sigma=\begin{pmatrix}\sigma_1^2&\rho\sigma_1\sigma_2\\\rho\sigma_1\sigma_2&\sigma_2^2\end{pmatrix}$,则:
\[
p_{\mathbf X}(\mathbf x)=\frac{1}{2\pi{|\Sigma|}^{1/2}}\exp\left(-\frac{1}{2}(\mathbf x-\boldsymbol\mu)^T\Sigma^{-1}(\mathbf x-\boldsymbol\mu)\right)
\]
这一形式可以推广到多元正态分布。
\end{com}

\begin{theorem}[二元正态分布的密度分解]
对定义式进行变形可以得到:
\[
f_{X,Y}(x,y)=\frac{1}{\sqrt{2\pi}\sigma_1}\exp\left(-\frac{(x-\mu_1)^2}{2\sigma_1^2}\right)\cdot\frac{1}{\sqrt{2\pi}\sigma_2\sqrt{1-\rho^2}}\exp\left(-\frac{\left[y-\left(\mu_2+\rho\frac{\sigma_2}{\sigma_1}(x-\mu_1)\right)\right]^2}{2\sigma_2^2(1-\rho^2)}\right)
\]
注意到,前一部分是 $N(\mu_1,\sigma_1)$ 的概率密度函数,后一部分是 $N\left(\mu_2+\rho\frac{\sigma_2}{\sigma_1}(x-\mu_1),\sigma_2^2(1-\rho^2)\right)$ 的概率密度函数。又由于:
\[
f_{X,Y}(x,y)=f_X(x)f_{Y|X}(y|x)
\]
所以事实上后一部分是就是 $f_{Y|X}(y|x)$. 
\end{theorem}

\begin{theorem}[二元正态分布的边缘分布]
根据密度分解容易知道,二元正态分布的边缘分布仍是正态分布,且 $X\sim N(\mu_1,\sigma_1^2),\quad Y\sim N(\mu_2,\sigma_2^2)$.
\end{theorem}

\begin{theorem}[二元正态分布的协方差与相关系数]
运用密度分解,可以计算:
\begin{align*}
\text{cov}(X,Y)&=\mathbb E[(X-\mathbb EX)(Y-\mathbb EY)]\\
&=\iint_{\R^2}(x-\mu_1)(y-\mu_2)f_{X,Y}(x,y)\mathrm dx\mathrm dy\\
&=\int_{-\infty}^{+\infty}(x-\mu_1)f_X(x)\mathrm dx\int_{-\infty}^{+\infty}(y-\mu_2)f_{Y|X}(y|x)\mathrm dy\\
&=\int_{-\infty}^{+\infty}(x-\mu_1)f_X(x)\mathrm dx\left[\int_{-\infty}^{+\infty}yf_{Y|X}(y|x)-\mu_2\right]\\
&=\int_{-\infty}^{+\infty}(x-\mu_1)f_X(x)\mathrm dx\Big[\mathbb E[Y|X=x]-\mu_2\Big]\\
&=\int_{-\infty}^{+\infty}(x-\mu_1)f_X(x)\mathrm dx\Big[\rho\frac{\sigma_2}{\sigma_1}(x-\mu_1)\Big]\\
&=\rho\frac{\sigma_2}{\sigma_1}\int_{-\infty}^{+\infty}(x-\mu_1)^2f_X(x)\mathrm dx\\
&=\rho\frac{\sigma_2}{\sigma_1}\var X\\
&=\rho\sigma_1\sigma_2
\end{align*}
由此可得相关系数:
$$
\rho(X,Y)=\frac{\cov(X,Y)}{\sqrt{\var X}\sqrt{\var Y}}=\rho
$$
也即二元正态分布定义中的参数 $\rho$ 就是其相关系数。
\end{theorem}

\begin{theorem}[二元正态分布的独立性]
设 $(X,Y)$ 服从二元正态分布,则 $X,Y$ 独立当且仅当 $\rho=0$. 
\end{theorem}
\begin{proof}
由于 $X,Y$ 独立蕴含着 $X,Y$ 不相关,而后者等价于相关系数 $\rho=0$,所以独立 $\implies\rho=0$. 又设 $\rho=0$,则:
$$
p_{X,Y}(x,y)=\frac{1}{\sqrt{2\pi}\sigma_1}\exp\left(-\frac{(x-\mu_1)^2}{2\sigma_1^2}\right)\cdot \frac{1}{\sqrt{2\pi}\sigma_2}\exp\left(-\frac{(y-\mu_2)^2}{2\sigma_2^2}\right)=p_X(x)p_Y(y)
$$
所以 $\rho=0\implies$ 独立。
\end{proof}
\begin{corollary}
对于二元正态分布而言,独立和不相关是等价的。
\end{corollary}
 \newpage
\section{正态分布的三个导出分布}
\label{sec:normal-derive}

\begin{definition}[$\chi^2$ 分布]
设 $X_1,X_2,\cdots,X_n$ 为 $n$ 个独立的服从 $N(0,1)$ 的随机变量,则称
\[
Z=\sum\limits_{i=1}^nX_i^2
\]
的分布为自由度为 $n$ 的 $\chi^2$ 分布,记作 $Z\sim \chi^2(n)$. 
\end{definition}

\noindent\textbf{期望与方差}:
\[
\E Z=n,\quad\var Z=2n
\]
\begin{proof}
由于
\[
\E X_i^2=\var X_i+(\E X_i)^2=1+0=1
\]
故
\[
\E Z=\E \left[\sum_{i=1}^nX_i^2\right]=\sum_{i=1}^n\E X_i^2=n
\]
又由于
\begin{align*}
\E X_i^4&=\int_{-\infty}^{+\infty}x^4\varphi(x)\mathrm dx\\
&=\frac{1}{\sqrt{2\pi}}\int_{-\infty}^{+\infty}x^4e^{-\frac{x^2}{2}}\mathrm dx
=-\frac{1}{\sqrt{2\pi}}\int_{-\infty}^{+\infty}x^3\mathrm d e^{-\frac{x^2}{2}}\\
&=\frac{3}{\sqrt{2\pi}}\int_{-\infty}^{+\infty}x^2e^{-\frac{x^2}{2}}\mathrm dx
=-\frac{3}{\sqrt{2\pi}}\int_{-\infty}^{+\infty}x\mathrm de^{-\frac{x^2}{2}}\\
&=\frac{3}{\sqrt{2\pi}}\int_{-\infty}^{+\infty}e^{-\frac{x^2}{2}}\mathrm dx=3
\end{align*}
故
\[
\var X_i^2=\E X_i^4-(\E X_i^2)^2=3-1=2
\]
故
\[
\var Z=\var \sum_{i=1}^nX_i^2=\sum_{i=1}^n\var X_i=2n
\]
\end{proof}


\begin{definition}[$t$ 分布]
设 $X\sim N(0,1)$,$Y\sim \chi^2(n)$,$X,Y$ 相互独立,则称
\[
t=\frac{X}{\sqrt{Y/n}}
\]
的分布为自由度为 $n$ 的 $t$ 分布,记作 $t\sim t(n)$. 
\end{definition}


\begin{definition}[$F$ 分布]
设 $X\sim\chi^2(n),Y\sim\chi^2(m)$,$X,Y$ 独立,则称
\[
Z=\frac{X/n}{Y/m}
\]
的分布为自由度为 $n,m$ 的 $F$ 分布,记作 $Z\sim F(n,m)$. 
\end{definition}


\begin{theorem}
设 $X_i\overset{\text{i.i.d.}}{\sim} N(\mu,\sigma^2)$,则样本均值服从期望相同、方差更小的正态分布:
\[
\bar X=\frac{1}{n}\sum\limits_{i=1}^nX_i\sim N\left(\mu,\frac{\sigma^2}{n}\right)
\]
\end{theorem}

\begin{theorem}
\label{thm:standard-normal}
设 $X_i\overset{\text{i.i.d.}}{\sim} N(\mu,\sigma^2)$,有样本均值的标准化:
\[
\frac{\bar X-\mu}{\sigma/\sqrt{n}}\sim N(0,1)
\]
\end{theorem}

\begin{theorem}
\label{thm:s2chi2}
设 $X_i\overset{\text{i.i.d.}}{\sim} N(\mu,\sigma^2)$,$S^2=\frac{1}{n-1}\sum\limits_{i=1}^n(X_i-\bar X)^2$ 为样本方差,则:
\[
\frac{(n-1)S^2}{\sigma^2}\sim \chi^2(n-1)
\]
\end{theorem}

\begin{theorem}
设 $X_i\overset{\text{i.i.d.}}{\sim} N(\mu,\sigma^2)$,$S^2=\frac{1}{n-1}\sum\limits_{i=1}^n(X_i-\bar X)^2$ 为样本方差,则 $\bar X$ 与 $S^2$ 独立。
\end{theorem}

\begin{theorem}
设 $X_i\overset{\text{i.i.d.}}{\sim} N(\mu,\sigma^2)$,$S^2=\frac{1}{n-1}\sum\limits_{i=1}^n(X_i-\bar X)^2$ 为样本方差,则:
\[
\frac{\sqrt n(\bar X-\mu)}{S}\sim t(n-1)
\]
\end{theorem}
\begin{proof}
根据定理 \ref{thm:standard-normal} 和定理 \ref{thm:s2chi2} 可知;
\[
\frac{\bar X-\mu}{\sigma/\sqrt{n}}\sim N(0,1),\quad
\frac{(n-1)S^2}{\sigma^2}\sim \chi^2(n-1)
\]
于是,根据 $t$ 分布的定义有:
\[
\cfrac{\cfrac{\bar X-\mu}{\sigma/\sqrt{n}}}{\sqrt{\frac{(n-1)S^2}{\sigma^2}/(n-1)}}=\frac{\sqrt n(\bar X-\mu)}{S}\sim t(n-1)
\]
\end{proof}

\begin{theorem}
设 $X_i\overset{\text{i.i.d.}}{\sim} N(\mu_1,\sigma_1^2)$,$i=1,2,\cdots,n$,样本方差为 $S_1^2$;$Y_i\overset{\text{i.i.d.}}{\sim} N(\mu_2,\sigma_2^2)$,$i=1,2,\cdots,m$,样本方差为 $S_2^2$,且 $X_i,Y_i$ 相互独立,则:
\[
\frac{S_1^2/\sigma_1^2}{S_2^2/\sigma_2^2}\sim F(n-1,m-1)
\]
\end{theorem}
\begin{proof}
由定理 \ref{thm:s2chi2} 知:
\[
\frac{(n-1)S_1^2}{\sigma_1^2}\sim\chi^2(n-1),\quad
\frac{(m-1)S_2^2}{\sigma_2^2}\sim \chi^2(m-1)
\]
于是根据 $F$ 分布的定义有:
\[
\frac{\dfrac{(n-1)S_1^2}{\sigma_1^2}/(n-1)}{\dfrac{(m-1)S_2^2}{\sigma_2^2}/(m-1)}=\frac{S_1^2/\sigma_1^2}{S_2^2/\sigma_2^2}\sim F(n-1,m-1)
\]
\end{proof}
 \newpage

\bibliographystyle{unsrt}
\bibliography{references}

\end{document}
